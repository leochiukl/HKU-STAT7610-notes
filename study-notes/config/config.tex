\usepackage[margin=1in]{geometry}
\usepackage{enumitem}
\usepackage{mathtools}
\usepackage{amssymb}
\usepackage{diagbox}
\usepackage{makecell}
\usepackage{aligned-overset}
\usepackage{derivative}
\usepackage{mleftright}
\usepackage{centernot}
\usepackage{cancel}
\usepackage{fontawesome5}
\usepackage[svgnames]{xcolor}
\usepackage{datetime2}
\usepackage{amsthm}
\usepackage{bm}
\usepackage{xfp}
\usepackage{pgfplots}
\usepackage{booktabs}
\usepackage[type={CC},modifier={by},version={4.0}]{doclicense}
\usepackage{imakeidx}
\usepackage[style=apa]{biblatex}
\usepackage{tikz}

\usepackage[colorlinks=true, allcolors=blue]{hyperref}
\usepackage{cleveref}

\addbibresource{bib/references.bib}

\usetikzlibrary{arrows.meta}
\usetikzlibrary{decorations.pathreplacing}
\usetikzlibrary{decorations.shapes}
\usetikzlibrary{calligraphy}
\usetikzlibrary{shapes}
\usetikzlibrary{patterns}
\usetikzlibrary{patterns.meta}
\usetikzlibrary{external}
\usepgfplotslibrary{fillbetween}

\pgfplotsset{compat=1.18}

\setenumerate[1]{label=\thesubsection.\arabic*~, ref=[\thesubsection.\arabic*]}
\mleftright

%Proper spacing when coloring math symbols
%https://tex.stackexchange.com/questions/21598/how-to-color-math-symbols
\makeatletter
\def\mcolor#1#{\@mcolor{#1}}
\def\@mcolor#1#2#3{%
  \protect\leavevmode
  \begingroup
    \color#1{#2}#3%
  \endgroup
}
\makeatother

\newcommand*{\prob}[1]{\mathbb{P}\left(#1\right)}
\newcommand*{\sprob}[1]{\mathbb{P}^{*}\left(#1\right)}
\newcommand*{\expv}[1]{\mathbb{E}\left[#1\right]}
\newcommand*{\expvmu}[2]{\mathbb{E}_{#1}\left[#2\right]}
\newcommand*{\sd}[1]{\operatorname{SD}\left(#1\right)}
\newcommand*{\vari}[1]{\operatorname{Var}\left(#1\right)}
\newcommand*{\cov}[1]{\operatorname{Cov}\left(#1\right)}
\newcommand*{\corr}[1]{\operatorname{Corr}\left(#1\right)}
\newcommand*{\pr}{\mathbb{P}}
\newcommand*{\spr}{\mathbb{P}^{*}}
\newcommand*{\indic}{\mathbf{1}}
\newcommand*{\indicset}[1]{\mathbf{1}_{\left\{#1\right\}}}
\newcommand*{\eqd}{\overset{\text{d}}=}
\newcommand*{\eqas}{\overset{\text{a.s.}}=}
\newcommand*{\eqae}{\overset{\text{a.e.}}=}
\newcommand*{\apxsim}{\overset{\text{approx.}}\sim}
\newcommand*{\iid}{\overset{\text{iid}}\sim}
\newcommand*{\ndist}[1]{\operatorname{N}\left(#1\right)}
\newcommand*{\ndistd}[1]{\operatorname{N}_{d}\left(#1\right)}
\newcommand*{\unif}[2]{\operatorname{U}\left(#1,#2\right)}
\newcommand*{\supp}[1]{\operatorname{supp}\left(#1\right)}
\newcommand*{\tocc}{\underset{n\to\infty}{\overset{\text{c.c.}}{\longrightarrow}}}
\newcommand*{\toas}{\underset{n\to\infty}{\overset{\text{a.s.}}{\longrightarrow}}}
\newcommand*{\topr}{\underset{n\to\infty}{\overset{\text{p}}{\longrightarrow}}}
\newcommand*{\netopr}{\underset{n\to\infty}{\overset{\text{p}}{\nearrow}}}
\newcommand*{\setopr}{\underset{n\to\infty}{\overset{\text{p}}{\searrow}}}
\newcommand*{\tod}{\underset{n\to\infty}{\overset{\text{d}}{\longrightarrow}}}
\newcommand*{\tolp}{\underset{n\to\infty}{\overset{L^p}{\longrightarrow}}}
\newcommand*{\tox}[2]{\underset{#1\to\infty}{\overset{#2}{\longrightarrow}}}

\newcommand*{\N}{\mathbb{N}}
\newcommand*{\R}{\mathbb{R}}
\newcommand*{\C}{\mathbb{C}}
\newcommand*{\Q}{\mathbb{Q}}
\newcommand*{\eR}{\bar{\mathbb{R}}}
\newcommand*{\vect}[1]{\bm{#1}}
\newcommand*{\pow}[1]{\mathcal{P}\left(#1\right)}
\newcommand*{\ran}[1]{\operatorname{ran}\left(#1\right)}

\newcommand*{\re}[1]{\operatorname{Re}\left(#1\right)}
\newcommand*{\im}[1]{\operatorname{Im}\left(#1\right)}

\newcommand*{\inner}[2]{\left\langle #1,#2\right\rangle}
\newcommand*{\rk}[1]{\operatorname{rank}\left(#1\right)}
\newcommand*{\nul}[1]{\operatorname{null}\left(#1\right)}

\newcommand*{\defn}[1]{\index[def]{#1}\textcolor{ForestGreen}{\textbf{#1}}}
\newcommand*{\bu}[1]{\textbf{\underline{#1}}}
\newcommand*{\cmark}{\gc{\faIcon{check}}}
\newcommand{\xmark}{\textcolor{red}{\faIcon{times}}}
\newcommand*{\ystar}{\textcolor{yellow}{\faIcon{star}}}
\newcommand*{\warn}{\textcolor{red}{\faIcon{exclamation-triangle}}}

\newcommand*{\vc}[1]{\mcolor{violet}{#1}}
\newcommand*{\mgc}[1]{\mcolor{magenta}{#1}}
\newcommand*{\gc}[1]{\mcolor{ForestGreen}{#1}}
\newcommand*{\blc}[1]{\mcolor{blue}{#1}}
\newcommand*{\brc}[1]{\mcolor{brown!70!black}{#1}}
\newcommand*{\orc}[1]{\mcolor{orange!90!black}{#1}}
\newcommand*{\rc}[1]{\mcolor{red}{#1}}
\newcommand*{\tec}[1]{\mcolor{teal}{#1}}

\DeclareMathOperator*{\esssup}{ess\,sup}
\DeclareMathOperator*{\argmin}{arg\,min}
\newenvironment{remark}{\sffamily \underline{Remarks}:\begin{itemize}}{\end{itemize}}
\newenvironment{note}{\sffamily\lbrack Note:}{\!\!\rbrack}
\newenvironment{mnemonic}{\sffamily\lbrack Mnemonic {\color{pink!90!black}\faIcon{brain}}:}{\!\!\rbrack}
\newenvironment{warning}{\sffamily\lbrack \warn\;\textcolor{red}{Warning}:}{\!\!\rbrack}
\newenvironment{intuition}{\sffamily\lbrack Intuition \textcolor{yellow!80!black}{\faIcon{lightbulb}}:}{\!\!\rbrack}
\newenvironment{pf}{\textit{Proof.}}{\qed}

\theoremstyle{definition}
\newtheorem{proposition}{Proposition}[subsection]
\renewcommand*\theproposition{\thesubsection.\alph{proposition}}
\newtheorem{corollary}[proposition]{Corollary}
\newtheorem{theorem}[proposition]{Theorem}
\newtheorem{lemma}[proposition]{Lemma}

\urlstyle{tt}

%For links in index page to link to line
%https://tex.stackexchange.com/questions/405813/link-to-line-instead-of-page
\makeatletter
\AtBeginDocument{
  \LetLtxMacro\egregs@index\index
  \RenewDocumentCommand{\index}{o+m}{%
    \begingroup
    \phantomsection%
    \IfValueTF{#1}{%
      \imki@wrindexentry{#1}{#2|hyperlink{\@currentHref}}{\thepage}%
    }{%
      \imki@wrindexentry{\jobname}{#2|hyperlink{\@currentHref}}{\thepage}%
    }%
    \endgroup
  }
}
\makeatother
\makeindex[name=def, title=Concepts and Terminologies]
\tikzexternalize[prefix=figures/]
