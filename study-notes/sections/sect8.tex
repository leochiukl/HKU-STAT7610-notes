\section{Conditional Expectation}
\label{sect:cond-exp}
\begin{enumerate}
\item We have finally reached the last section of this notes, which is about
the concept of \emph{conditional expectation}. This notion plays a fundamental
role in the study of probability theory, which formalizes the idea of computing
probabilistic quantities with some \emph{additional information} given
\emph{(conditioning)}. While you should have already learnt basic ideas about
conditional expectation before, here we will analyze it in full mathematical
details and introduce a \emph{measure-theoretic} way to define conditional
expectation. Such definition is quite general and frees us from many
restrictions, but is also rather abstract unfortunately \faIcon[regular]{dizzy}.
\end{enumerate}
\subsection{Ordinary Conditioning}
\begin{enumerate}
\item Before providing the measure-theoretic definition of conditional
expectation, we first have a glance on how ``ordinary'' conditioning (i.e., the
one you have seen before) works, and its potential restrictions, as a
motivation of the measure-theoretic study of conditional expectation.
\item \textbf{Ordinary conditional probability and implied conditional
expectation.} Recall from \Cref{subsect:ordinary-cond-prob} that, on a
probability space \((\Omega,\mathcal{F},\pr)\), for an event
\(B\in\mathcal{F}\) with \(\prob{B}>0\), the \emph{(ordinary) conditional
probability} of \(A\) given \(B\) is defined by \(\prob{A|B}=\prob{A\cap
B}/\prob{B}\) for every \(A\in\mathcal{F}\). Based on this idea, we can then
define \emph{conditional distribution function} (with some restrictions) as
follows. Consider a random vector \(\vect{X}_2:\Omega\to\R^{d_{2}}\). For every
event \(B\in\mathcal{F}\) with \(\prob{B}>0\), we define the \defn{(ordinary)
conditional distribution function of \(\vect{X}_2\) given \(B\)} by
\[
F_{\vect{X}_2|B}(\vect{x}_2):=\prob{\vect{X}_2\le\vect{x}_2|B}
=\frac{\prob{\{\vect{X}_2\le\vect{x}_2\}\cap B}}{\prob{B}}
\]
for all \(\vect{x}_2\in\R^{d_2}\). The mean of \(F_{\vect{X}_2|B}\) (i.e., mean
of a random variable with distribution function being \(F_{\vect{X}_2|B}\)), if
exists, is called the \defn{(ordinary) conditional expectation of
\(\vect{X}_2\) given \(B\)}.  Often, we are handling the case with
\(B=\{\vect{X}_1=\vect{x}_1\}\) for some random variable \(\vect{X}_1\).
However, the definition above only works if
\(\prob{B}=\prob{\vect{X}_1=\vect{x}_1}>0\) (e.g., discrete case), and breaks
down if \(\prob{\vect{X}_1=\vect{x}_1}=0\) (e.g., continuous case).
\item \textbf{How to deal with conditioning on zero-probability
event?}
If \(F_{\vect{X}_1}\) is discrete, we know that
\(B=\{\vect{X}_1=\vect{x}_1\}\) has positive probability for every
\(\vect{x}_1\in\supp{F_{\vect{X}_1}}\), and zero probability for every
\(\vect{x}_1\) outside the support. In this case, we may handle the potential
conditioning on zero probability through, e.g., setting the conditional
distribution function as zero for every \(\vect{x}_1\) outside the support:
\[
F_{\vect{X}_2|\vect{X}_1}(\vect{x}_1)=
\begin{cases}
\prob{\vect{X}_2\le\vect{x}_2|\vect{X}_1=\vect{x}_1}&\text{if \(\vect{x}_1\in\supp{F_{\vect{X}_1}}\),} \\
0&\text{otherwise.}
\end{cases}
\]
This is still meaningful as we have plenty of
\(\vect{x}_1\in\supp{F_{\vect{X}_1}}\) to work with, and we can just
``neglect'' the behaviour of \(F_{\vect{X}_2|\vect{X}_1}\) at \(\vect{x}_1\)
outside the support.

However, in case \(F_{\vect{X}_1}\) is \emph{continuous}, the event
\(B=\{\vect{X}_1=\vect{x}_1\}\) would have zero probability for \emph{every}
\(\vect{x}_1\), making the method suggested above obsolete, as we would need to
artificially define the value taken by \(F_{\vect{X}_2|\vect{X}_1}\) at
\emph{every} point, making this concept not meaningful anymore.

This leads us to the idea of considering the following limiting argument.
Suppose we have \((\vect{X}_1,\vect{X}_2)\sim F\) with a joint density \(f\).
Recall from your first probability course that the \emph{conditional density}
is defined by
\[
f_{\vect{X}_2|\vect{X}_1}(\vect{x}_2|\vect{x}_1)
=\frac{f(\vect{x}_1,\vect{x}_2)}{f_{\vect{X}_1}(\vect{x}_1)}
\]
if \(f_{\vect{X}_1}(\vect{x}_1)>0\). It is then natural to define
\[
F_{\vect{X}_2|\vect{X}_1}(\vect{x}_2|\vect{x}_1)
:=\prob{\vect{X}_2\le\vect{x}_2|\vect{X}_1=\vect{x}_1}
=\int_{\vect{-\infty}}^{\vect{x}_2}
f_{\vect{X}_2|\vect{X}_1}(\widetilde{\vect{x}}_2|\vect{x}_1)
\odif{\widetilde{\vect{x}}_2}
\]
provided that \(f_{\vect{X}_1}(\vect{x}_1)>0\). Nevertheless, such
``definition'' may actually lead to potential ill-definedness, as the
\emph{Borel-Kolmogorov paradox} illustrates.
\item \textbf{Borel-Kolmogorov paradox.} Let
\((X_1,X_2)\sim\operatorname{U}(D)\) with \(D=\{(x_1,x_2):-1\le x_1\le 1, 0\le
x_2\le \sqrt{1-x_{1}^{2}}\}\) being the upper half of unit disk centered at
\((0,0)\). Here, \(\operatorname{U}(D)\) refers to the uniform distribution
on \(D\) (i.e., constant density over \(D\)). We then consider the conditional
distribution using two coordinates.

\begin{enumerate}
\item \emph{(Cartesian coordinates)} Using the Cartesian coordinates, the joint
density is \[f(x_1,x_2)=\frac{1}{\pi/2}\indicset{(x_1,x_2)\in D}
=\frac{2}{\pi}\indicset{(x_1,x_2)\in D},\] and we have
\[f_{X_1}(x_1)
=\int_{0}^{\sqrt{1-x_{1}^{2}}}f(x_1,x_2)\odif{x_2}
=\frac{2}{\pi}\indicset{-1\le x_1\le 1}\int_{0}^{\sqrt{1-x_{1}^{2}}}\odif{x_2}
=\frac{2\sqrt{1-x_{1}^{2}}}{\pi}\indicset{-1\le x_1\le 1}.
\]
Hence, for every \(x_1\in (-1,1)\), we get
\[
f_{X_2|X_1}(x_2|x_1)=\frac{f(x_1,x_2)}{f_{X_1}(x_1)}
=\frac{1}{\sqrt{1-x_{1}^{2}}}\indicset{(x_1,x_2)\in D},
\]
Particularly, we have
\[\prob{X_2\in [0,1/2]|X_1=0}=\int_{0}^{1/2}1\odif{x_2}=\frac{1}{2}.\]

\item \emph{(Polar coordinates)} We introduce the polar coordinates with
\(X_1=R\cos(\Theta)\) and \(X_2=R\sin(\Theta)\). In polar coordinates, \(D\)
can be expressed as \(\{(r,\theta):0\le r\le 1, 0\le \theta\le \pi\}
=[0,1]\times[0,\pi]\). Applying the change-of-variables formula for density,
we get
\begin{align*}
f_{R,\Theta}(r,\theta)&=f(x_{1},x_{2})
\begin{vmatrix}
\pdv{r\cos\theta}/{r}&\pdv{r\cos\theta}/{\theta} \\
\pdv{r\sin\theta}/{r}&\pdv{r\sin\theta}/{\theta}
\end{vmatrix}
=f(x_1,x_2)\begin{vmatrix}
\cos\theta&-r\sin\theta \\
\sin\theta&r\cos\theta
\end{vmatrix} \\
&=\frac{2r}{\pi}\indicset{(x_1,x_2)\in D}
=\frac{2r}{\pi}\indicset{(r,\theta)\in [0,1]\times [0,\pi]}.
\end{align*}
Therefore, we have \[f_{\Theta}(\theta)
=\indicset{\theta\in[0,\pi]}\int_{0}^{1}\frac{2r}{\pi}\odif{r}
=\frac{1}{\pi}\indicset{\theta\in[0,\pi]}.\]
Thus, for every \(\theta\in [0,\pi]\) we get
\[
f_{R|\Theta}(r|\theta)=\frac{2r/\pi}{1/\pi}\indicset{0\le r\le 1}
=2r\indicset{0\le r\le 1}.
\]
Particularly, we have
\[
\prob{R\in[0,1/2]|\Theta=\pi/2}=\int_{0}^{1/2}2r\odif{r}=\frac{1}{4}.
\]
\end{enumerate}
In this case, while both \(\prob{X_2\in [0,1/2]|X_1=0}\) and
\(\prob{R\in[0,1/2]|\Theta=\pi/2}\) should just be describing the same
probability expressed in different coordinates system, their values turn out to
differ.
\begin{center}
\begin{tikzpicture}
\begin{axis}[domain=-1:1, axis lines=middle, ymax=1.1, ymin=-0.1, xmax=1.1,
xmin=-1.1, axis equal image]
\addplot[blue, name path=A]{sqrt{1-x^2}};
\addplot[draw=none, name path=B]{0};
\tikzfillbetween[of=A and B]{blue, opacity=0.1};
\draw[yellow, opacity=0.5, line width=0.5mm] (0,0) -- (0,0.5);
\end{axis}
\end{tikzpicture}
\end{center}
Intuitively, this phenomenon appears since the two coordinates system lead to
different ways to ``approach'' this probability as a limit. This is best
understood through the following pictures:
\begin{center}
\begin{tikzpicture}
\begin{axis}[domain=-1:1, axis lines=middle, ymax=1.1, ymin=-0.1, xmax=1.1,
xmin=-1.1, title={Cartesian}, axis equal image]
\addplot[blue, name path=A]{sqrt{1-x^2}};
\addplot[draw=none, name path=B]{0};
\tikzfillbetween[of=A and B]{blue, opacity=0.1};
\draw[green, opacity=0.3, line width=2mm] (0,0) -- (0,1);
\draw[yellow, opacity=0.5, line width=0.5mm] (0,0) -- (0,0.5);
\draw[->, violet] (-0.2,0.3) -- (-0.05,0.3);
\draw[->, violet] (0.2,0.3) -- (0.05,0.3);
\draw[densely dotted, magenta] (-0.03,0.5) -- (0.03,0.5);
\end{axis}
\end{tikzpicture}
\begin{tikzpicture}
\begin{axis}[domain=-1:1, axis lines=middle, ymax=1.1, ymin=-0.1, xmax=1.1,
xmin=-1.1, title={Polar}, axis equal image]
\addplot[blue, name path=A]{sqrt{1-x^2}};
\addplot[draw=none, name path=B]{0};
\tikzfillbetween[of=A and B]{blue, opacity=0.1};
\draw[yellow, opacity=0.5, line width=0.5mm] (0,0) -- (0,0.5);
\addplot[draw=none, name path=C, domain=-0.05:0]{sqrt{1-x^2}};
\addplot[draw=none, name path=D, domain=-0.05:0]{-19.975*x};
\addplot[draw=none, name path=E, domain=0:0.05]{sqrt{1-x^2}};
\addplot[draw=none, name path=F, domain=0:0.05]{19.975*x};
\tikzfillbetween[of=C and D]{green, opacity=0.3};
\tikzfillbetween[of=E and F]{green, opacity=0.3};
\draw[->, violet] (-0.2,0.35) -- (-0.05,0.4);
\draw[->, violet] (0.2,0.35) -- (0.05,0.4);
\draw[dashed, magenta] (-0.025,0.49875) -- (0,1);
\draw[dashed, magenta] (0.025,0.49875) -- (0,1);
\draw[dashed, magenta] (-0.025,0.49875) -- (0.025,0.49875);
\end{axis}
\end{tikzpicture}
\end{center}
These pictures also intuitively illustrate why the probability is \(1/2\) for
Cartesian coordinates and is \(1/4\) for polar coordinates. For Cartesian
coordinates, the ``region of interest'' occupies a half of the ``limiting''
region (approached with ``rectangles'' having shrinking widths). On the other
hand, for polar coordinates, it only occupies a quarter of the ``limiting
region'' (approached with ``circular sectors'' having shrinking central angles).
The dependency of the probability calculation on the way of how the limit is
approached is an undesirable trait, and can lead to ambiguities. This issue
can be avoided by following the \emph{measure-theoretic} approach instead.
\end{enumerate}
\subsection{Conditioning in a Measure-Theoretic Framework}
\label{subsect:meas-theoretic-cond}
\begin{enumerate}
\item In view of the potential issues from the ``ordinary'' way of conditioning
suggested above, we are then motivated to study the conditioning in a
\emph{measure-theoretic} framework. Let us start with the measure-theoretic
definition of conditional expectation.

Let \((\Omega,\mathcal{F},\pr)\) be a probability space, \(\mathcal{G}\subseteq
\mathcal{F}\) be a \(\sigma\)-algebra on \(\Omega\), and \(X\in
L^{1}(\Omega,\mathcal{F},\pr)\). The \defn{conditional expectation of \(X\)
given \(\mathcal{G}\)}, denoted by \(\expv{X|\mathcal{G}}\), is any function
\(Y:\Omega\to\R\) satisfying:
\begin{enumerate}[label={(\arabic*)}]
\item \emph{(Measurability)} \(Y\) is \(\mathcal{G}\)-measurable.
\item \emph{(Partial averaging)} \(\expv{Y\indic_{A}}=\expv{X\indic_{A}}\) for
every \(A\in\mathcal{G}\).

\begin{note}
Due to the presence of the indicator function \(\indic_{A}\), the ``averaging''
(mean) is only taken over ``part'' of \(\Omega\), hence the name ``partial
averaging''.
\end{note}
\end{enumerate}
Any such \(Y\) is said to be a \defn{version} of \(\expv{X|\mathcal{G}}\). We
also have the following shorthand notations about conditional expectation:
\begin{itemize}
\item \(\expv{X|\mathcal{A}}:=\expv{X|\sigma(\mathcal{A})}\) with
\(\mathcal{A}\subseteq \mathcal{F}\) being a collection of events.
\item \(\expv{X|Z}:=\expv{X|\sigma(Z)}\) with \(Z:\Omega\to\Omega'\) being a
function (e.g., random variable, random vector, etc.).
\item \(\vari{X|\mathcal{G}}=\expv{(X-\expv{X|\mathcal{G}})^{2}|\mathcal{G}}\)
is the \defn{conditional variance of \(X\) given \(\mathcal{G}\)}.
\end{itemize}
With the measurability and partial averaging properties satisfied, the function
\(Y\) carries the characteristics of a random variable and an expectation
``with respect to \(\mathcal{G}\)'' (i.e., we only require
\(\mathcal{G}\)-measurability for random variable, and consider only set in
\(\mathcal{G}\) for the partial averaging).

\item \textbf{Integrability, existence, and uniqueness.} With this
nonconstructive definition of conditional expectation, the first thing we
should do is to examine whether the object defined ``makes sense'', by
investigating some core properties about it, namely \emph{integrability},
\emph{existence}, and \emph{uniqueness}.

\begin{proposition}[Integrability]
\label{prp:cond-exp-int}
Every version \(Y\) of \(\expv{X|\mathcal{G}}\) is integrable, i.e., in
\(L^{1}(\Omega,\vc{\mathcal{G}},\pr)\).
\end{proposition}
\begin{pf}
Let \(A=\{Y\ge 0\}=Y^{-1}([0,\infty))\in\mathcal{G}\). Then, we have
\(A^{c}\in\mathcal{G}\), and hence
\begin{align*}
\expv{|Y|}&=\expv{|Y|\indic_{A}}+\expv{|Y|\indic_{A^{c}}}
=\expv{Y\indic_{A}}+\expv{(-Y)\indic_{A^{c}}}
=\int_{A}^{}Y\odif{\pr}+\int_{A^{c}}^{}(-Y)\odif{\pr} \\
\overset{\text{(partial averaging)}}&{=}
\int_{A}^{}X\odif{\pr}+\int_{A^{c}}^{}(-X)\odif{\pr}
\overset{\text{(monotonicity)}}{\le}
\int_{A}^{}|X|\odif{\pr}+\int_{A^{c}}^{}|X|\odif{\pr}
=\int_{\Omega}^{}|X|\odif{\pr}=\expv{|X|}\overset{(X\in L^{1})}{<}\infty.
\end{align*}
\end{pf}

\begin{theorem}[Existence and a.s.\ uniqueness]
\label{thm:cond-exp-exist-unique}
The conditional expectation \(\expv{X|\mathcal{G}}\) exists and is unique a.s.
\end{theorem}
\begin{pf}
We first prove the existence. Consider first the case where \(X\ge 0\). By
\labelcref{it:nonneg-leb-int-prop}, the function \(\nu\) given by
\(\nu(A)=\expv{X\indic_{A}}=\int_{A}^{}X\odif{\pr}\) for all
\(A\in\mathcal{G}\) is a measure on \((\Omega,\mathcal{G})\). Also, by
\labelcref{it:quasi-int-prop} we have \(\nu(A)=0\) for every \(A\in\mathcal{G}\)
with \(\prob{A}=0\), i.e., \(\nu\ll\pr\), with \(\pr\) being the original
probability measure restricted on \(\mathcal{G}\) (we keep this notation for
convenience). Applying \Cref{thm:leb-rn} then gives, for all
\(A\in\mathcal{G}\),
\[
\int_{A}^{}X\odif{\pr}=\nu(A)=\int_{A}^{}\odv{\nu}{\pr}\odif{\pr}
\]
where \(\odv{\nu}/{\pr}\) is the \(\pr\)-a.s.\ unique and
\(\mathcal{G}\)-measurable Radon-Nikodym derivative. By
definition of conditional expectation, \(\odv{\nu}/{\pr}\) is a version of
\(\expv{X|\mathcal{G}}\), establishing the existence in this case.

Next, consider the general case where \(X\) may not be nonnegative. We write
\(X=X^{+}-X^{-}\), and let \(Y^{+}=\expv{X^{+}|\mathcal{G}}\) and
\(Y^{-}=\expv{X^{-}|\mathcal{G}}\) (which exist by the case above). By
\Cref{prp:cond-exp-int}, both \(Y^{+}\) and \(Y^{-}\) are integrable, which
implies that \(Y:=Y^{+}-Y^{-}\) is also integrable (and hence
\(\mathcal{G}\)-measurable). Furthermore, we have
\[
\int_{A}^{}X\odif{\pr}=
\int_{A}^{}X^{+}\odif{\pr}
-\int_{A}^{}X^{-}\odif{\pr}
\overset{\text{(partial averaging)}}{=}
\int_{A}^{}Y^{+}\odif{\pr}
-\int_{A}^{}Y^{-}\odif{\pr}
=\int_{A}^{}Y\odif{\pr}
\]
for all \(A\in\mathcal{G}\). Therefore, \(Y\) is a version of
\(\expv{X|\mathcal{G}}\). This establishes the existence.

Now, we prove the a.s.\ uniqueness. Suppose both \(Y\) and \(\widetilde{Y}\)
are versions of \(\expv{X|\mathcal{G}}\). Then, by the partial averaging
property we have
\(\expv{Y\indic_{A}}=\expv{X\indic_{A}}=\expv{\widetilde{Y}\indic_{A}}\) for
every \(A\in\mathcal{G}\). Furthermore, by \Cref{prp:cond-exp-int}, \(Y\) and
\(\widetilde{Y}\) are integrable. Hence, by
\labelcref{it:equiv-ae-equal-int-fn} we have \(Y\eqas\widetilde{Y}\),
establishing the a.s.\ uniqueness.
\end{pf}

As \Cref{thm:cond-exp-exist-unique} suggests, the conditional expectation is
only unique \emph{almost surely}.  Consequently, equalities/inequalities
involving conditional expectation like \(Y=\expv{X|\mathcal{G}}\) should all be
understood to hold \emph{almost surely} only. Nonetheless, to avoid making the
notations cumbersome, we often just write ``\(=\)'' to mean ``\(\eqas\)''; this
should be clear from context.  When considering conditional expectation in the
measure-theoretic framework here, we are actually working with a specific
version of \(\expv{X|\mathcal{G}}\) implicitly. But since it equals every other
version a.s., we usually do not care much about the choice of version.
\item \textbf{Regular conditional probability measures.} Having the conditional
expectation \(\expv{X|\mathcal{G}}\) defined, it is then tempting to define the
\emph{conditional probability measure} \(\prob{\cdot|\mathcal{G}}\) by
\(\prob{A|\mathcal{G}}=\expv{\indic_{A}|\mathcal{G}}\) for every
\(A\in\mathcal{F}\). However, since \(\expv{\indic_{A}|\mathcal{G}}\) is only
unique a.s.\ in general, such function may not be well-defined.

In view of this, the first thing to be done is to specify a certain version of
\(\expv{\indic_{A}|\mathcal{G}}\) for each \(A\in\mathcal{F}\) in the
definition, so that the function is well-defined. Afterwards, we need to
ensure that \(\prob{\cdot|\mathcal{G}}\) is a valid probability measure, by
requiring the conditions there to be satisfied. This discussion leads to
the definition of \emph{regular conditional probability measure}.

Let \((\Omega,\mathcal{F})\) and \((\Omega',\mathcal{F}')\) be measurable
spaces, and \(\mathcal{G}\subseteq \mathcal{F}\) be a \(\sigma\)-algebra.  The
function \(\spr:\mathcal{F}\times \Omega\to [0,\infty]\) is called a \defn{regular
conditional probability measure given \(\mathcal{G}\)} if
\begin{enumerate}[label={(\arabic*)}]
\item \emph{(Specifying versions)} For every fixed \(A\in\mathcal{F}\), the
function \(\vc{\omega}\mapsto\sprob{A,\vc{\omega}}\) is a specific version of
\(\prob{A|\mathcal{G}}:=\expv{\indic_{A}|\mathcal{G}}\).
\item \emph{(Qualifying as probability measure)} For every
fixed \(\omega\in\Omega\), the function \(\vc{A}\mapsto\sprob{\vc{A},\omega}\)
is a probability measure on \((\Omega,\mathcal{F})\).
\end{enumerate}
In a similar way, we call the function
\(F_{\vect{X}|\mathcal{G}}^{*}:\R^{d}\times \Omega \to [0,1]\) a \defn{regular
conditional distribution function of \(\vect{X}\) given \(\mathcal{G}\)} if
\begin{enumerate}[label={(\arabic*)}]
\item \emph{(Specifying versions)} For every fixed \(\vect{x}\in\R^{d}\),
the function \(\vc{\omega}\mapsto F_{\vect{X}|\mathcal{G}}^{*}(\vect{x},\vc{\omega})\)
is a specific version of \(\prob{\vect{X}\le\vect{x}|\mathcal{G}}\).
\item \emph{(Qualifying as distribution function)} For every fixed
\(\omega\in\Omega\), the function \(\vc{\vect{x}}\mapsto
F_{\vect{X}|\mathcal{G}}(\vc{\vect{x}},\omega)\) is a distribution function.
\end{enumerate}

The definitions of regular conditional probability measure and distribution
function here are again \emph{nonconstructive}, so it is not immediately clear
whether they actually exist (just like the case for conditional expectation).
It turns out that as long as there is a bijective function
\(\varphi:\Omega'\to\R^{d}\) such that both \(\varphi\) and \(\varphi^{-1}\)
are measurable \emph{(\((\Omega',\mathcal{F}')\) is nice)}, then both of them
exist. In many cases of practical interest, such ``niceness'' is satisfied,
e.g., \((\Omega',\mathcal{F}')=(\R^{d},\mathcal{B}(\R^{d}))\) is nice
(by taking \(\varphi\) to be the identity function). So, henceforth we will
assume the niceness and always work with the regular ones (implicitly) when
considering the notations like \(\prob{A|\mathcal{G}}\) and
\(\prob{\vect{X}\le\vect{x}|\mathcal{G}}\).
\item \textbf{Formula of conditional expectation
for \(\mathcal{G}\) generated by a countable partition.} Generally, we do not
have an explicit formula for finding out the conditional expectation.  However,
in the special case where \(\mathcal{G}\) is generated by a countable
partition, we do have an explicit formula for conditional expectation (up to
a.s.\ equality), which involves expressions that should have appeared in your
first probability course.

\begin{proposition}
\label{prp:count-part-cond-exp}
Let \(X\in L^{1}(\Omega,\mathcal{F},\pr)\),
\(\mathcal{A}=\{A_n:n\in\N\}\subseteq \mathcal{F}\) be a partition of
\(\Omega\), and \(\mathcal{G}=\sigma(\mathcal{A})
\overset{\text{(\Cref{lma:sig-alg-gen-part})}}{=}
\{\biguplus_{i\in
I}^{}A_i:A_i\in\mathcal{A}~\forall i\in I \text{ and } I\subseteq \N\}\subseteq
\mathcal{F}\). Then, we have
\[
\expv{X|\mathcal{G}}=\sum_{n=1}^{\infty}\expvmu{A_n}{X}\indic_{A_n},\quad
\text{where }\expvmu{A_n}{X}:=\begin{cases}
\expv{X\indic_{A_n}}/\prob{A_n}&\text{if \(\prob{A_n}>0\),} \\
0&\text{if \(\prob{A_n}=0\).}
\end{cases}
\]
\end{proposition}

\begin{remark}
\item The value assigned for \(\expvmu{A_n}{X}\) in the case \(\prob{A_n}=0\) can
indeed be replaced by other real numbers (not necessarily \(0\)); this also
explains why the equality only holds a.s.\

\item \emph{(More explicit form of the formula)} For convenience, suppose that
there are only \(k\) nonempty sets \(A_1,\dotsc,A_k\) in the partition, each
with positive probability. Then, we can write
\[
\expv{X|\mathcal{G}}(\omega)
=\begin{cases}
\expv{X\indic_{A_1}}/\prob{A_1}&\text{if \(\omega\in A_1\),} \\
\expv{X\indic_{A_2}}/\prob{A_2}&\text{if \(\omega\in A_2\),} \\
\vdots \\
\expv{X\indic_{A_k}}/\prob{A_k}&\text{if \(\omega\in A_k\).}
\end{cases}
\]
From this expression, it is transparent that the conditional expectation in
this case is a ``piecewise constant'' function in \(\omega\) (the output stays
the same within a certain \(A_j\)).
\end{remark}

\begin{pf}
Let \(Y:=\sum_{n=1}^{\infty}\expvmu{A_n}{X}\indic_{A_n}:\Omega\to\R\). We now
verify that \(Y\) satisfies both the measurability and partial averaging
properties.

\begin{enumerate}[label={(\arabic*)}]
\item Let \(Y_m:=\sum_{n=1}^{m}\expvmu{A_n}{X}\indic_{A_n}
=\sum_{n=1}^{m}\expvmu{A_n}{X}\indic_{A_n} +0\cdot
\indic_{\biguplus_{k=m+1}^{\infty}A_k}\) for every \(m\in\N\).  Since each
\(Y_m\) is simple with all the indicator sets being in \(\mathcal{G}\), we know
\(Y_m\in\mathcal{G}\) for all \(m\in\N\). Hence, by \labelcref{it:seq-rvs-meas}
we have \(Y=\lim_{m\to\infty}Y_m\in\mathcal{G}\).
\item \textbf{Showing that \(Y\indic_{A}\in L^{1}(\Omega,\mathcal{F},\pr)\) for
all \(A\in\mathcal{G}\).}
Note that \(|Y|\overset{\text{(triangle)}}{\le}
\sum_{n=1}^{\infty}|\expvmu{A_n}{X}|\indic_{A_n}
\overset{\text{(triangle)}}{\le}
\sum_{n=1}^{\infty}\expvmu{A_n}{|X|}\indic_{A_n}\). Hence,
\begin{align*}
\expv{|Y|}\overset{\text{(monotonicity)}}&{\le}
\expv{\sum_{n=1}^{\infty}\expvmu{A_n}{|X|}\indic_{A_n}}
\overset{\text{(MCT)}}{=}
\lim_{N\to\infty}\expv{\sum_{n=1}^{N}\expvmu{A_n}{|X|}\indic_{A_n}} \\
\overset{\text{(linearity)}}&{=}
\lim_{N\to\infty}\sum_{n=1}^{N}\expvmu{A_n}{|X|}\prob{A_n}
=\lim_{N\to\infty}\sum_{n=1}^{N}\expv{|X|\indic_{A_n}} \\
\overset{\text{(linearity)}}&{=}
\lim_{N\to\infty}\expv{\vc{\sum_{n=1}^{N}}|X|\indic_{\vc{A_n}}}
=\lim_{N\to\infty}\expv{|X|\indic_{\vc{\bigcup_{n=1}^{N}A_n}}}
\overset{\text{(MCT)}}{=}\expv{|X|}<\infty.
\end{align*}
Therefore, for every \(A\in\mathcal{G}\), we have
\(\expv{|Y\indic_{A}|}\overset{\text{(monotonicity)}}{\le} \expv{|Y|}<\infty\),
and hence \(Y\indic_{A}\in L^{1}(\Omega,\mathcal{F},\pr)\).

\textbf{Showing the partial averaging property.}
Fix any \(A\in\mathcal{G}\). We can then write \(A=\biguplus_{i\in I}^{}A_i\)
for some countable \(I\subseteq \N\). Therefore,
\begin{align*}
\expv{Y\indic_{A}}\overset{\text{(\(Y\indic_{A}\in L^{1}\), 
\labelcref{it:leb-int-sig-add})}}&{=}
\sum_{i\in I}^{}\expv{\vc{Y}\indic_{A_i}}
=\sum_{i\in I}^{}\expv{\vc{\expvmu{A_i}{X}}\indic_{A_i}} \\
\overset{\text{(\(\vc{\expvmu{A_i}{X}}\) deterministic)}}&{=}
\sum_{i\in I}^{}\vc{\expvmu{A_i}{X}}\expv{\indic_{A_i}}
=\sum_{i\in I}^{}\expvmu{A_i}{X}\prob{A_i}.
\end{align*}
Noting that for each \(i\in I\),
\begin{align*}
\expvmu{A_i}{X}\prob{A_i}
&=\begin{cases}
\frac{\expv{X\indic_{A_i}}}{\prob{A_i}}\prob{A_i}=\expv{X\indic_{A_i}}
&\text{if \(\prob{A_i}>0\),} \\
0\cdot 0&\text{if \(\prob{A_i}=0\),}
\end{cases} \\
\overset{\text{\labelcref{it:quasi-int-prop}}}&{=}
\expv{X\indic_{A_i}},
\end{align*}
we have
\[
\expv{Y\indic_{A}}=\sum_{i\in I}^{}\expv{X\indic_{A_i}}
\overset{\text{\labelcref{it:leb-int-sig-add}}}{=}\expv{X\indic_{A}}.
\]
\end{enumerate}
\end{pf}

\begin{remark}
\item \emph{(Interpretation of formula)} The formula of conditional expectation
in \Cref{prp:count-part-cond-exp} can be more intuitively understood as
follows. From an ``information'' perspective, we can interpret ``given
\(\mathcal{G}\)'' as suggesting that, for the unknown outcome
\(\omega\in\Omega\), the given information is enough for us to determine
whether each set in \(\mathcal{G}\) contains \(\omega\). In the case here, we
have \(\mathcal{G}=\{\biguplus_{i\in I}^{}A_i:A_i\in\mathcal{A}~\forall i\in I
\text{ and } I\subseteq \N\}\), so based on the information we have
\(\omega\in A_n\) for some known \(n\in\N\) (but we still do not know what
\(\omega\) is exactly). With this piece of information, the ``expectation'' or
``best prediction'' for \(X=X(\omega)\) should naturally be the mean of \(X\)
over \(A_n\) (containing all the possible candidates of \(\omega\)), scaled by
for the probability of \(A_n\), i.e., \(\expv{X|\mathcal{G}}(\omega)
\overset{(\omega\in A_n)}{=}\expv{X\indic_{A_n}}/\prob{A_n}=\expvmu{A_n}{X}\).
\item \emph{(Conditional probability)} \Cref{prp:count-part-cond-exp} also
gives rise to formulas of conditional probability. By taking \(X=\indic_{A}\)
with \(A\in\mathcal{F}\), we have
\[
\prob{A|\mathcal{G}}=\expv{\indic_{A}|\mathcal{G}}
=\sum_{n=1}^{\infty}\frac{\expv{\indic_{A}\indic_{A_n}}}{\prob{A_n}}\indic_{A_n}
=\sum_{n=1}^{\infty}\frac{\expv{\indic_{A\cap A_n}}}{\prob{A_n}}\indic_{A_n}
=\sum_{n=1}^{\infty}\frac{\prob{A\cap A_n}}{\prob{A_n}}\indic_{A_n}
=\sum_{n=1}^{\infty}\prob{A|A_n}\indic_{A_n}
\]
with \(\prob{A|A_n}:=0\) if \(\prob{A_n}=0\). This formula suggests that
\(\prob{A|\mathcal{G}}(\omega)=\prob{A|A_n}\) whenever \(\omega\in A_n\), which
is quite intuitive.

\item \emph{(Conditional probability given discrete \(\vect{X}\))}
The formula above can be reduced to a more familiar one by taking \(\mathcal{G}=
\sigma(\vect{X})\) where \(\vect{X}\) has support \(\supp{\vect{X}}=\{\vect{x}_n:n\in\N\}\).
In such case, we can write \(\mathcal{G}=\{\vect{X}^{-1}(B):B\in\mathcal{B}(\R^{d})\}
=\sigma(\mathcal{A})\) where \(\mathcal{A}\) is a partition of \(\Omega\),
including every set
\(\{\vect{X}=\vect{x}_n\}\) (and also the set
\(\{\vect{X}\notin\supp{\vect{X}}\}\), if nonempty, which always has
probability zero and thus can be omitted in the sum below). Hence, by the
formula above we have
\[
\prob{A|\vect{X}}:=\prob{A|\sigma(\vect{X})}
=\sum_{n=1}^{\infty}\prob{A|\vect{X}=\vect{x}_n}\indicset{\vect{X}=\vect{x}_n},
\]
meaning that if we know that \(\vect{X}=\vect{x}_n\) (i.e.,
\(\omega\in\{\vect{X}=\vect{x}_n\}\)), the conditional probability
\(\prob{A|\vect{X}}\) is given by \(\prob{A|\vect{X}=\vect{x}_n}\) (very
natural).
\end{remark}
\item \textbf{Properties of conditional expectation.} Based on the definition
of conditional expectation here, we can indeed deduce many properties that are
perhaps somewhat familiar to you, and also allow us to work with conditional
expectations more efficiently.

\begin{proposition}
\label{prp:cond-exp-prop}
Let \((\Omega,\mathcal{F},\pr)\) be a probability space, \(\mathcal{H},\mathcal{G}\)
be \(\sigma\)-algebras such that \(\mathcal{H}\subseteq \mathcal{G}\subseteq
\mathcal{F}\), and \(X,Y\in L^{1}(\Omega,\mathcal{F},\pr)\).

\begin{enumerate}
\item \emph{(No information)} \(\expv{X|\{\varnothing,\Omega\}}=\expv{X}\).
\item \emph{(No relevant information)} If \(X\) is independent of
\(\mathcal{G}\) (i.e., \(\sigma(X)\) and \(\mathcal{G}\) are independent), then
\(\expv{X|\mathcal{G}}=\expv{X}\).
\item \emph{(Full information)} If \(X\in\mathcal{G}\), then \(\expv{X|\mathcal{G}}=X\).
Particularly, \(\expv{c|\mathcal{G}}=c\) for every \(c\in\R\).
\item \emph{(Linearity)}
\(\expv{aX+bY|\mathcal{G}}=a\expv{X|\mathcal{G}}+b\expv{Y|\mathcal{G}}\) for
all \(a,b\in\R\).
\item \emph{(Monotonicity)} If \(X\le Y\), then
\(\expv{X|\mathcal{G}}\le\expv{Y|\mathcal{G}}\).
\item \emph{(Triangle inequality)} \(|\expv{X|\mathcal{G}}|\le\expv{|X||\mathcal{G}}\).
\item \emph{(Tower property)} \(\expv{\expv{X|\mathcal{H}}|\mathcal{G}}
=\expv{X|\mathcal{H}}=\expv{\expv{X|\mathcal{G}}|\mathcal{H}}\).
Particularly, \(\expv{X}=\expv{\expv{X|\mathcal{G}}}\) \emph{(law of total
expectation)}.
\end{enumerate}
\end{proposition}
\begin{pf}
\begin{enumerate}
\item \begin{enumerate}[label={(\arabic*)}]
\item Since \(\expv{X}\) is constant, it is
\(\{\varnothing,\Omega\}\)-measurable.
\item We have
\(
\expv{\expv{X}\indic_{\varnothing}}
\overset{\text{\labelcref{it:quasi-int-prop}}}{=}
0\overset{\text{\labelcref{it:quasi-int-prop}}}{=}
\expv{X\indic_{\varnothing}}
\)
and
\(
\expv{\expv{X}\indic_{\Omega}}
=\expv{\expv{X}}
=\expv{X}
=\expv{X\indic_{\Omega}}
\).
\end{enumerate}
\item \begin{enumerate}[label={(\arabic*)}]
\item Since \(\expv{X}\) is constant, it is \(\mathcal{G}\)-measurable.
\item For all \(A\in\mathcal{G}\), we have
\(
\expv{\vc{\expv{X}}\indic_{A}}
\overset{\text{(linearity)}}{=}\vc{\expv{X}}\expv{\indic_{A}}
\overset{\text{(independence)}}{=}\expv{X\indic_{A}}
\).
\end{enumerate}
\item Let \(Y=X\). We have:
\begin{enumerate}[label={(\arabic*)}]
\item \(Y\) is \(\mathcal{G}\)-measurable by definition.
\item For all \(A\in\mathcal{G}\), we have
\(\expv{Y\indic_{A}}=\expv{X\indic_{A}}\).
\end{enumerate}
Hence, \(\expv{X|\mathcal{G}}=Y=X\). In particular, the random variable
(constant function) \(c\) is always \(\mathcal{G}\)-measurable, and so
\(\expv{c|\mathcal{G}}=c\).
\item \begin{enumerate}[label={(\arabic*)}]
\item Since \(\expv{X|\mathcal{G}}\) and \(\expv{Y|\mathcal{G}}\) are
\(\mathcal{G}\)-measurable, \(a\expv{X|\mathcal{G}}+b\expv{Y|\mathcal{G}}\)
is also \(\mathcal{G}\)-measurable.
\item For all \(A\in\mathcal{G}\), we have
\begin{align*}
\expv{(a\expv{X|\mathcal{G}}+b\expv{Y|\mathcal{G}})\indic_{A}}
\overset{\text{(linearity)}}&{=}
a\expv{\expv{X|\mathcal{G}}\indic_{A}}+b\expv{\expv{Y|\mathcal{G}}\indic_{A}} \\
\overset{\text{(partial averaging)}}&{=}
a\expv{X\indic_{A}}+b\expv{Y\indic_{A}}
\overset{\text{(linearity)}}{=}
\expv{(aX+bY)\indic_{A}}.
\end{align*}
\end{enumerate}
\item
For all \(A\in\mathcal{G}\), we have \[\expv{\expv{X|\mathcal{G}}\indic_{A}}
\overset{\text{(partial averaging)}}{=}\expv{X\indic_{A}}
\overset{(X\indic_{A}\le Y\indic_{A})}{\le}\expv{Y\indic_{A}}
\overset{\text{(partial averaging)}}{=}\expv{\expv{Y|\mathcal{G}}\indic_{A}}.\]
Hence, we have
\[\expv{(\expv{X|\mathcal{G}}-\expv{Y|\mathcal{G}})\indic_{A}}\le
0.\]

Now fix any \(\varepsilon>0\) and let \(A_{\varepsilon}:=\{
\expv{X|\mathcal{G}}-\expv{Y|\mathcal{G}}\ge\varepsilon
\}\in\mathcal{G}\) (as \(\expv{X|\mathcal{G}}-\expv{Y|\mathcal{G}}\) is
\(\mathcal{G}\)-measurable). Applying the inequality above on
\(A_{\varepsilon}\) gives
\[
0\ge\expv{(\expv{X|\mathcal{G}}-\expv{Y|\mathcal{G}})\indic_{A_{\varepsilon}}}
\overset{\text{(monotonicity)}}{\ge}\expv{\varepsilon\indic_{A_{\varepsilon}}}
=\varepsilon\prob{A_{\varepsilon}}\ge 0.
\]
This implies that \(\prob{A_{\varepsilon}}=0\) for all \(\varepsilon>0\),
and hence
\[
0\le 
\prob{\expv{X|\mathcal{G}}>\expv{Y|\mathcal{G}}}
=\prob{\bigcup_{n=1}^{\infty}A_{1/n}}
\overset{\text{(subadditivity)}}{\le}
\sum_{n=1}^{\infty}\prob{A_{1/n}}
=0,
\]
meaning that \(\prob{\expv{X|\mathcal{G}}>\expv{Y|\mathcal{G}}}=0\). Therefore,
we have \(\expv{X|\mathcal{G}}\le\expv{Y|\mathcal{G}}\) (a.s.).
\item Consider
\begin{align*}
|\expv{X|\mathcal{G}}|&=\left|\expv{X^{+}-X^{-}|\mathcal{G}}\right|
\overset{\text{(linearity)}}{=}
\left|\expv{X^{+}|\mathcal{G}}-\expv{X^{-}|\mathcal{G}}\right| \\
\overset{\text{(triangle)}}&{\le}
\big|\underbrace{\expv{X^{+}|\mathcal{G}}}
_{\mathclap{\underset{\text{(monotonicity)}}{\ge}0}}\big|
+\big|\underbrace{\expv{X^{-}|\mathcal{G}}}
_{\mathclap{\underset{\text{(monotonicity)}}{\ge}0}}\big|
=\expv{X^{+}|\mathcal{G}}+\expv{X^{-}|\mathcal{G}}
=\expv{|X||\mathcal{G}}.
\end{align*}
\item Since \(\expv{X|\mathcal{H}}\in\mathcal{H}\) and \(\mathcal{H}\subseteq
\mathcal{G}\), we have \(\expv{X|\mathcal{H}}\in\mathcal{G}\). Hence, by (c)
we have \(\expv{\expv{X|\mathcal{H}}|\mathcal{G}}=\expv{X|\mathcal{H}}\).
This proves the first equality. Now, consider the second equality:
\begin{enumerate}[label={(\arabic*)}]
\item By definition, \(\expv{\expv{X|\mathcal{G}}|\mathcal{H}}\) is
\(\mathcal{H}\)-measurable.
\item For all \(A\in\mathcal{H}\subseteq \mathcal{G}\), we have
\[
\expv{\expv{\expv{X|\mathcal{G}}|\mathcal{H}}\indic_{A}}
\underset{(A\in\mathcal{H})}{\overset{\text{(partial averaging)}}{=}}
\expv{\expv{X|\mathcal{G}}\indic_{A}}
\underset{(A\in\mathcal{G})}{\overset{\text{(partial averaging)}}{=}}
\expv{X\indic_{A}}
\underset{(A\in\mathcal{H})}{\overset{\text{(partial averaging)}}{=}}
\expv{\expv{X|\mathcal{H}}\indic_{A}}.
\]
\end{enumerate}
To get the law of total expectation, take
\(\mathcal{H}=\{\varnothing,\Omega\}\subseteq \mathcal{G}\). Then, we have
\[
\expv{X}\overset{\text{(a)}}{=}\expv{X|\{\varnothing,\Omega\}}
\overset{\text{(tower)}}{=}
\expv{\expv{X|\mathcal{G}}|\{\varnothing,\Omega\}}
\overset{\text{(a)}}{=}\expv{\expv{X|\mathcal{G}}}.
\]
\end{enumerate}
\end{pf}

\begin{remark}
\item \emph{(``Tower'' in tower property)} In the tower property, we have
\(\expv{\expv{X|\mathcal{H}}|\mathcal{G}}
=\expv{X|\mathcal{H}}=\expv{\expv{X|\mathcal{G}}|\mathcal{H}}\). There
are several ``layers'' of conditioning involved, which looks like a ``tower'',
hence the name ``tower property''.
\item \emph{(Interpretation of tower property)} To interpret the tower property
more intuitively, we can consider the impact of the
coarseness/fineness\footnote{We say that \(\mathcal{H}\) is coarser than
\(\mathcal{G}\) (or \(\mathcal{G}\) is finer than \(\mathcal{H}\)) if
\(\mathcal{H}\subseteq \mathcal{G}\), i.e., \(\mathcal{H}\) contains fewer
``pieces'' and hence they are generally ``larger''/``rougher''.} of
\(\mathcal{G}\) on the conditional expectation (averaging): For a coarser
(finer) \(\mathcal{G}\), generally ``larger'' (``smaller'') pieces are
contained. Hence, ``stronger'' (``weaker'') averaging takes place over the
``pieces'', making \(\expv{X|\mathcal{G}}\) retain less (more) information
about \(X\). Parts (a) and (c) provide two extreme examples: (i) For
\(\mathcal{G}=\{\varnothing,\Omega\}\), it is so coarse that no information is
retained except just the mean of \(X\): \(\expv{X|\mathcal{G}}=\expv{X}\), and
(ii) For \(\mathcal{G}\) with \(X\in\mathcal{G}\), it is so fine that full
information about \(X\) is retained: \(\expv{X|\mathcal{G}}=X\) (there is no
``averaging'' taking place).

Armed with this idea, the tower property can then be interpreted as saying that
\emph{the coarser \(\sigma\)-algebra remains} (\(\mathcal{H}\) in the case with
\(\mathcal{G}\subseteq \mathcal{H}\)), which can be intuitively understood as
follows:
\begin{itemize}
\item \(\expv{\expv{X|\mathcal{H}}|\mathcal{G}}=\expv{X|\mathcal{H}}\):
Conditioning on the coarser \(\mathcal{H}\) already retains only a little
information about \(X\), and conditioning on the finer \(\mathcal{G}\)
afterwards cannot ``bring back'' the information lost.
\item \(\expv{\expv{X|\mathcal{G}}|\mathcal{H}}=\expv{X|\mathcal{H}}\)
Conditioning on the finer \(\mathcal{G}\) first allows us to retain more
information about \(X\), but conditioning on the coarser \(\mathcal{H}\)
afterwards just leads to more loss in information, and so at the end, still a
little information about \(X\) is retained.
\end{itemize}
\end{remark}
\item \textbf{Inequalities about conditional expectation.}
In \labelcref{it:holder-cs-ineq,it:minkowski-ineq,it:jensen-ineq}, we have
studied various inequalities about expectation. It turns out that there is an
analogous versions of them for \emph{conditional} expectation. But before
stating them, we first need to define a ``conditional version'' of \(L^{p}\)
norm. We define \(\|X|\mathcal{G}\|_{p}:=\expv{|X|^{p}|\mathcal{G}}^{1/p}\)
for every \(p\in (0,\infty]\) (though we typically only consider \(p\in[1,\infty]\)).
The inequalities are as follows. Let \((\Omega,\mathcal{F},\pr)\) be a
probability space.
\begin{enumerate}
\item \label{it:cond-holder-cs-ineq} \emph{(Conditional H\"older's inequality)}
Let \(p,q\in [1,\infty]\) with \(1/p+1/q=1\) (with the convention that
\(1/\infty:=0\)) be conjugate indices. Then, \(\|XY\vc{|\mathcal{G}}\|_{1}\le
\|X\vc{|\mathcal{G}}\|_{p}\|Y\vc{|\mathcal{G}}\|_{q}\) for all \(X\in
L^{p}(\Omega,\mathcal{F},\pr)\) and \(Y\in L^{q}(\Omega,\mathcal{F},\pr)\)
(which can then be shown to imply \(XY\in L^{1}(\Omega,\mathcal{F},\pr)\)).

\begin{note}
The special case with \(p=q=2\) is known as the \emph{conditional
Cauchy-Schwarz inequality}.
\end{note}
\item \label{it:cond-minkowski-ineq} \emph{(Conditional Minkowski's inequality)}
Let \(p\in [1,\infty]\). Then \(\|X+Y\vc{|\mathcal{G}}\|_{p}\le \|X\vc{|\mathcal{G}}\|_{p}+\|Y\vc{|\mathcal{G}}\|_{p}\) for all
\(X,Y\in L^{p}(\Omega,\mathcal{F},\pr)\).
\item \label{it:cond-jensen-ineq} \emph{(Conditional Jensen's inequality)}
Let \(X:\Omega\to\R\) be a function in \(L^{1}(\Omega,\mathcal{F},\pr)\), and
\(\varphi\) be a convex function on \(\R\) such that \(\varphi(X)\in
L^{1}(\Omega,\mathcal{F},\pr)\). Then,
\(\varphi(\expv{X\vc{|\mathcal{G}}})\le\expv{\varphi(X)\vc{|\mathcal{G}}}\).

\begin{note}
Like \labelcref{it:jensen-ineq}, if \(\varphi\) is concave (with
\(\varphi(X)\in L^{1}(\Omega,\mathcal{F},\pr)\) still), then we have
\(\varphi(\expv{X|\mathcal{G}})\ge\expv{\varphi(X)|\mathcal{G}}\).
\end{note}
\end{enumerate}
\begin{pf}
Omitted.
\end{pf}
\item \textbf{Contraction and convergence in \(L^{p}\) for conditional expectation.}
\begin{enumerate}
\item\label{it:cond-lp-contract} \emph{(Contraction in \(L^{p}\))}
If \(X\in L^{p}(\Omega,\mathcal{F},\pr)\) and \(p\in [1,\infty)\), then
\(\|\expv{X|\mathcal{G}}\|_{p}\le\|X\|_{p}\).

\item \label{it:cond-lp-conv} \emph{(Convergence in \(L^{p}\))}
Let \(X,X_1,X_2,\dotsc\) be in \(L^{p}(\Omega,\mathcal{F},\pr)\), with
\(p\in[1,\infty)\).  If \(X_n\tolp X\), then
\(\expv{X_n|\mathcal{G}}\tolp\expv{X|\mathcal{G}}\).
\end{enumerate}
\begin{pf}
\begin{enumerate}
\item Note that
\[
\|\orc{\expv{X|\mathcal{G}}}\|_{p}=\expv{|\orc{\expv{X|\mathcal{G}}}|^{p}}^{1/p}
\overset{\text{(conditional Jensen)}}{\le}
\expv{\expv{|X|^{p}|\mathcal{G}}}^{1/p}
\overset{\text{(total expectation)}}{=}
\expv{|X|^{p}}^{1/p}=\|X\|_{p}.
\]
\item Note that
\[
\|\expv{X_n|\mathcal{G}}-\expv{X|\mathcal{G}}\|_{p}
\overset{\text{(linearity)}}{=}
\|\expv{X_n-X|\mathcal{G}}\|_{p}
\overset{\text{(a)}}{\le}\|X_n-X\|_{p}\tox{n}{\text{(assumption)}}0.
\]
Hence, by definition we have \(\expv{X_n|\mathcal{G}}\tolp\expv{X|\mathcal{G}}\).
\end{enumerate}
\end{pf}
\item \textbf{Conditional versions of convergence results.}
In \Cref{thm:mct,lma:fatou,thm:dct}, we have studied three notable convergence
results for expectation, namely MCT, Fatou's lemma, and DCT. Like the
inequalities, there are also conditional versions for them, as follows.

Let \((\Omega,\mathcal{F},\pr)\) be a probability space.
\begin{enumerate}
\item \emph{(Conditional monotone convergence theorem)}
Let \(X_n\) be an \vc{integrable} function in \(L_{+}\) for every \(n\in\N\).
If we have \(X_n\nearrow X\) pointwisely, then \(X\) is in \(L_{+}\) and
is \vc{integrable}, and \(\expv{X_n|\mathcal{G}}\nearrow\expv{X|\mathcal{G}}\) \vc{a.s.}
\item \emph{(Conditional Fatou's lemma)} Let \(X_n\) be an \vc{integrable}
function in \(L_{+}\) for every \(n\in\N\). Then, \(\expv{\liminf_{n\to
\infty}X_n|\mathcal{G}}\le\liminf_{n\to \infty}\expv{X_n|\mathcal{G}}\)
\vc{a.s.}
\item \emph{(Conditional dominated convergence theorem)}
Let \(X_n\in L^{1}\) for every \(n\in\N\), and \(X:\Omega\to\R\) be measurable.
If \(X_n\toas X\) and \(|X_n|\le Y\) a.s.\ for all \(n\in\N\), for some
\(Y\in L^{1}\) \emph{(domination)}, then \(X\in L^{1}\) and
\(\lim_{n\to\infty}\expv{X_n|\mathcal{G}}=\expv{X|\mathcal{G}}\) \vc{a.s.}
\end{enumerate}
\begin{pf}
\begin{enumerate}
\item By monotonicity, \(X_n\nearrow X\) pointwisely implies
\(\expv{X_n|\mathcal{G}}\nearrow\) pointwisely (a.s.), and so we can let
\(Y:=\lim_{n\to\infty}\expv{X_n|\mathcal{G}}\) (a.s.). By
\labelcref{it:seq-rvs-meas}, \(Y\) is \(\mathcal{G}\)-measurable. Also, for all
\(A\in\mathcal{G}\) we have
\[
\expv{Y\indic_{A}}=\expv{\lim_{n\to\infty}(\expv{X_n|\mathcal{G}}\indic_{A})}
\overset{\text{(MCT)}}{=}\lim_{n\to\infty}\expv{\expv{X_n|\mathcal{G}}\indic_{A}}
\overset{\text{(partial averaging)}}{=}\lim_{n\to\infty}\expv{X_n\indic_{A}}
\overset{\text{(MCT)}}{=}\expv{X\indic_{A}}.
\]
By \Cref{prp:cond-exp-int}, we know that \(Y\in L^{1}\). Hence, by taking
\(A=\Omega\in\mathcal{G}\) we know \(X\in L_{+}\) is integrable (MCT implies
that \(X\in L_{+}\)).  Furthermore, we have
\(\expv{X|\mathcal{G}}=Y=\lim_{n\to\infty}\expv{X_n|\mathcal{G}}\).
\item Like the proof of Fatou's lemma, let \(Y_n:=\inf_{k\ge n}X_k\) for every
\(n\in\N\). Since \(0\le Y_n\le X_n\) for each \(n\in\N\), each \(Y_n\) is in
\(L_{+}\) and is integrable. Also, \(Y_n\nearrow\) pointwisely by construction,
thus we can let \(Y:=\lim_{n\to\infty}Y_n=\liminf_{n\to \infty}X_n\).
Hence,
\begin{align*}
\expv{\liminf_{n\to \infty}X_n|\mathcal{G}}
&=\expv{Y|\mathcal{G}}\overset{\text{(conditional MCT)}}{=}
\lim_{n\to\infty}\expv{Y_n|\mathcal{G}}
=\liminf_{n\to \infty}\expv{\vc{Y_n}|\mathcal{G}} \\
&=\liminf_{n\to \infty}\expv{\left.\vc{\inf_{k\ge n}X_k}\right|\mathcal{G}}
\overset{(\vc{\inf_{k\ge n}X_k}\le X_n)}{=}
\liminf_{n\to \infty}\expv{X_n|\mathcal{G}}.
\end{align*}
\item By DCT, we immediately have \(X\in L^{1}\). So it remains to show that
\(\lim_{n\to\infty}\expv{X_n|\mathcal{G}}=\expv{X|\mathcal{G}}\) a.s. Noting
that
\[
0\le|\expv{X_n|\mathcal{G}}-\expv{X|\mathcal{G}}|
\overset{\text{(linearity)}}{=}|\expv{X_n-X|\mathcal{G}}|
\overset{\text{(triangle)}}{\le}\expv{|X_n-X||\mathcal{G}}
\overset{\text{(monotonicity)}}{\le}
\expv{\left.\sup_{k\ge n}|X_k-X|\right|\mathcal{G}},
\]
it suffices to show that \(\lim_{n\to\infty}\expv{Z_n}=0\) a.s., with
\(Z_{\vc{n}}:= \sup_{k\ge \vc{n}}|X_k-X|\) for every \(n\in\N\).
By \Cref{thm:as-conv-char-ip}, we have \(Z_n\topr 0\). Also, for each
\(n\in\N\), we have
\[
0\le Z_n
\overset{\text{(triangle)}}{\le}
\sup_{k\ge n}(\underbrace{|X_n|}_{\underset{\text{a.s.}}{\le} Y}+
\underbrace{|X|}_{\underset{\text{a.s.}}{\le} Y})
\overset{\text{a.s.}}{\le}2Y \in L^{1}.
\]
It then follows by \Cref{cor:dct-ip-version} that
\(\lim_{n\to\infty}\expv{Z_n}=\expv{0}=0\).

Since \(Z_n\ge 0\) and \(Z_n\searrow\), by monotonicity we know
\(\expv{Z_n|\mathcal{G}}\ge 0\) and \(\expv{Z_n|\mathcal{G}}\searrow\).
Therefore, we can let \(Z:=\lim_{n\to\infty}\expv{Z_n|\mathcal{G}}\), which is
\(\mathcal{G}\)-measurable by \labelcref{it:seq-rvs-meas}. Also, since
\(0\le Z\le
\expv{Z_n|\mathcal{G}}\overset{\text{(\Cref{prp:cond-exp-int})}}{\in} L^{1}\)
for each \(n\in\N\), we know \(Z\) is in \(L_{+}\) and is integrable, and also
\[
0\overset{\text{(monotonicity)}}{\le}
\expv{Z}\overset{\text{(monotonicity)}}{\le}
\expv{\expv{Z_n|\mathcal{G}}}\overset{\text{(total expectation)}}{=}
\expv{Z_n}\tox{n}{\text{(above)}}0,
\]
which implies that \(\expv{Z}=0\). Hence,
\(\lim_{n\to\infty}\expv{Z_n|\mathcal{G}}\overset{\text{(definition)}}{=}Z
\underset{\labelcref{it:nonneg-leb-int-prop}}{\eqas} 0\).
\end{enumerate}
\end{pf}
\item \textbf{``Removing'' \(\sigma\)-algebra in conditional expectation based
on independence.} By the \emph{no relevant information} property in
\Cref{prp:cond-exp-prop}, we know that \(\expv{X|\mathcal{G}}=\expv{X}\) if
\(\mathcal{G}\) is independent of \vc{\(\sigma(X)\)}. Here, the \(\sigma\)-algebra
\(\mathcal{G}\) can be ``removed'' from the conditional expectation due to the
independence. This property is generalized by the following result, which
suggests when such ``removal'' can take place with two \(\sigma\)-algebras
involved.

\begin{proposition}
\label{prp:cond-exp-ind-remove-sig-alg}
Let \((\Omega,\mathcal{F},\pr)\) be a probability space,
\(\mathcal{G},\mathcal{H}\subseteq \mathcal{F}\) be \(\sigma\)-algebras, and
\(X\in L^{1}(\Omega,\mathcal{F},\pr)\). If \(\mathcal{H}\) is independent of
\vc{\(\sigma(\sigma(X),\mathcal{G})\)}, then
\(\expv{X|\sigma(\mathcal{G},\mathcal{H})} =\expv{X|\mathcal{G}}\).
\end{proposition}
\begin{pf}
Consider first the case with \(X\ge 0\). Let \(Y\) be a version of
\(\expv{X|\mathcal{G}}\). Then \(Y\) is \(\mathcal{G}\)-measurable by definition,
and \(Y\ge 0\) a.s.\ by monotonicity. By \labelcref{it:nonneg-leb-int-prop},
the functions \(\mu\) and \(\nu\) defined by \(\mu(A)=\expv{Y\indic_{A}}\)
and \(\nu(A)=\expv{X\indic_{A}}\) for all \(A\in\mathcal{F}\) are measures on
\((\Omega,\mathcal{F})\). Note that \(\mu(\Omega)=\expv{Y}=\expv{\expv{X|\mathcal{G}}}
\overset{\text{(total expectation)}}{=}\expv{X}<\infty\)
and \(\nu(\Omega)=\expv{X}<\infty\) also. Hence both measures are finite, thus
also \(\sigma\)-finite.

For all \(G\in\mathcal{G}\) and \(H\in\mathcal{H}\), we have
\begin{align*}
\mu(G\cap H)&=\expv{Y\indic_{G\cap H}}=\expv{Y\indic_{G}\indic_{H}}
\overset{\text{(\(Y\in\mathcal{G}\), \(\mathcal{H}\) independent of
\(\mathcal{G}\))}}{=}\expv{Y\indic_{G}}\expv{\indic_{H}} \\
\overset{\text{(partial averaging)}}&{=}
\expv{X\indic_{G}}\expv{\indic_{H}}
\overset{\text{(\(\mathcal{H}\) independent of
\(\sigma(\sigma(X),\mathcal{G})\))}}{=}
\expv{X\indic_{G}\indic_{H}}
=\expv{X\indic_{G\cap H}}=\nu(G\cap H).
\end{align*}
This means that \(\mu\) and \(\nu\) are \(\sigma\)-finite measures that
coincide on the \(\pi\)-system \(\{G\cap H:G\in\mathcal{G},
H\in\mathcal{H}\}\). Thus, by \Cref{prp:meas-unique}, \(\mu\) and \(\nu\)
coincide on \(\sigma(\{G\cap H:G\in\mathcal{G},H\in\mathcal{H}\}
\overset{\text{\labelcref{it:sig-alg-union-two-expr}}}{=}
\sigma(\mathcal{G},\mathcal{H})\). In other words, for every
\(A\in\sigma(\mathcal{G},\mathcal{H})\), we have
\(\mu(A)=\expv{Y\indic_{A}}=\expv{X\indic_{A}}=\nu(A)\). This shows the partial
averaging property.

Therefore, \(Y=\expv{X|\mathcal{G}}\) is also a version of
\(\expv{X|\sigma(\mathcal{G},\mathcal{H})}\). It then follows by the a.s.\
uniqueness of conditional expectation (\Cref{thm:cond-exp-exist-unique})
that \(\expv{X|\mathcal{G}}\eqas\expv{X|\sigma(\mathcal{G},\mathcal{H})}\).

Now, consider the general case where \(X\) may not be nonnegative. We write
\(X=X^{+}-X^{-}\). Then, by the proven case and linearity, we get
\[
\expv{X|\mathcal{G}}\eqas \expv{X^{+}|\mathcal{G}}-\expv{X^{-}|\mathcal{G}}
\eqas\expv{X^{+}|\sigma(\mathcal{G},\mathcal{H})}
-\expv{X^{-}|\sigma(\mathcal{G},\mathcal{H})}
\eqas\expv{X|\sigma(\mathcal{G},\mathcal{H})}.
\]
\end{pf}

To see how \Cref{prp:cond-exp-ind-remove-sig-alg} generalizes the \emph{no
relevant information} property (i.e., includes it as a special case), take
\(\mathcal{G}=\{\varnothing,\Omega\}\). Then, we have
\[\sigma(\sigma(X),\mathcal{G})\overset{\text{(consider
definition)}}{=}\sigma(\sigma(X))\overset{\text{(\(\sigma(X)\) is the smallest
\(\sigma\)-algebra containing itself)}}{=}\sigma(X)\]
and 
\[\sigma(\mathcal{G},\mathcal{H})\overset{\text{(consider
definition)}}{=}\sigma(\mathcal{H})\overset{\text{(\(\mathcal{H}\) is the smallest
\(\sigma\)-algebra containing itself)}}{=}\mathcal{H}.\]
Therefore, in this case \Cref{prp:cond-exp-ind-remove-sig-alg} is just
asserting that
\(\expv{X|\mathcal{H}}=\expv{X|\{\varnothing,\Omega\}}\overset{\text{(no
information)}}{=}\expv{X}\), which is the same as the \emph{no relevant
information} property.
\item \textbf{Taking out what is known.} The next property of conditional
expectation to be discussed is \emph{taking out what is known (TOWIK)}, which
is frequently used. In its proof, we will utilize the following lemma, which
can simplify the verification of the \emph{partial averaging} property for
conditional expectation.
\begin{lemma}
\label{lma:part-avg-on-pi-sys}
Let \((\Omega,\mathcal{F},\pr)\) be a probability space, \(\mathcal{G}\subseteq
\mathcal{F}\) be a \(\sigma\)-algebra on \(\Omega\), and \(X\in
L^{1}(\Omega,\mathcal{F},\pr)\). If \(\mathcal{A}\subseteq \mathcal{F}\) is a
\(\pi\)-system such that (i) \(\sigma(\mathcal{A})=\mathcal{G}\)
\emph{(generating \(\mathcal{G}\))} and (ii) there exists
\(\{A_i\}_{i\in\N}\subseteq \mathcal{A}\) such that
\(\bigcup_{i=1}^{\infty}A_i=\Omega\) \emph{(containing a cover of \(\Omega\))},
then the function \(Y:\Omega\to\R\) is a version of \(\expv{X|\mathcal{G}}\) if
it satisfies:
\begin{enumerate}[label={(\arabic*)}]
\item \emph{(Measurability)} \(Y\in\mathcal{G}\).
\item \emph{(Partial averaging on \(\mathcal{A}\))}
\(\expv{Y\indic_{A}}=\expv{X\indic_{A}}\) for all \(A\in\mathcal{A}\).
\end{enumerate}
\end{lemma}
\begin{pf}
First, for all \(A\in\mathcal{A}\), we have
\(\expv{|X\indic_{A}|}\overset{\text{(monotonicity)}}{\le} \expv{|X|}<\infty\),
and so \(X\indic_{A}\) is integrable. Now, by the partial averaging on
\(\mathcal{A}\), we have \(\expv{Y\indic_{A}}=\expv{X\indic_{A}}\),
which means that \(Y\indic_{A}\) is integrable also. Therefore, by the
integrability of \(X\indic_{A}\) and \(Y\indic_{A}\) for all \(A\in\mathcal{A}\),
we get the following chain of equivalences:
\begin{align*}
\quad\expv{Y\indic_{A}}&=\expv{X\indic_{A}}\quad\forall A\in\mathcal{A} \\
\iff \expv{Y^{+}\indic_{A}}-\expv{Y^{-}\indic_{A}}
&=\expv{X^{+}\indic_{A}}-\expv{X^{-}\indic_{A}}\quad \forall A\in\mathcal{A} \\
\iff \mu(A):=\expv{(Y^{+}+X^{-})\indic_{A}}
&=\expv{(X^{+}+Y^{-})\indic_{A}}=:\nu(A)\quad \forall A\in\mathcal{A}.
\end{align*}
Since \(X\in\mathcal{F}\) and \(Y\in\mathcal{G}\) (hence also
\(Y\in\mathcal{F}\)), we know \(Y^{+}+X^{-}\in\mathcal{F}\) and
\(X^{+}+Y^{-}\in\mathcal{F}\). Therefore, by \labelcref{it:nonneg-leb-int-prop}
we know \(\mu\) and \(\nu\) are measures on \(\mathcal{F}\) (hence also on
\(\mathcal{G}\)).

With \(\mu|_{\mathcal{A}}=\nu|_{\mathcal{A}}\) and the integrability of \(X,Y\)
(thus also \(X^{+},X^{-},Y^{+},Y^{-}\)) on \(\mathcal{A}\subseteq
\sigma(\mathcal{A})=\mathcal{G}\), we get for each \(i\in\N\),
\(\mu(A_i)\overset{(A_i\in\mathcal{A})}{=}\nu(A_i)
\overset{\text{(integrability)}}{<}\infty\). Since
\(\bigcup_{i=1}^{\infty}A_i=\Omega\) by assumption, we have
\[
\mu|_{\mathcal{G}}=\mu|_{\sigma(\mathcal{A})}
\overset{\text{(\Cref{prp:meas-unique})}}{=}
\nu|_{\sigma(\mathcal{A})}=\nu|_{\mathcal{G}}.
\]
This means that
\[
\mu(A)=\expv{(Y^{+}+X^{-})\indic_{A}}
=\expv{(X^{+}+Y^{-})\indic_{A}}=\nu(A)\quad \forall A\in\mathcal{G},
\]
which is equivalent to
\[
\expv{Y\indic_{A}}=\expv{X\indic_{A}}\quad\forall A\in\mathcal{G},
\]
i.e., the partial averaging property for conditional expectation. Hence, by
definition, \(Y\) is a version of \(\expv{X|\mathcal{G}}\).
\end{pf}

Now we are ready to prove the \emph{taking out what is known} property.

\begin{proposition}[Taking out what is known (TOWIK)]
\label{prp:towik}
Let \((\Omega,\mathcal{F},\pr)\) be a probability space, and
\(\mathcal{G}\subseteq \mathcal{F}\) be a \(\sigma\)-algebra on \(\Omega\). If
\(XY\) and \(X\) are integrable and \(Y\in\mathcal{G}\), then \(\expv{XY|\mathcal{G}}
=Y\expv{X|\mathcal{G}}\) \emph{(taking out the ``known'' \(Y\))}.
\end{proposition}
\begin{pf}
\begin{enumerate}[label={(\arabic*)}]
\item Since \(Y\in\mathcal{G}\) and \(\expv{X|\mathcal{G}}\in\mathcal{G}\),
we have \(Y\expv{X|\mathcal{G}}\in\mathcal{G}\).
\item We apply the standard argument on \(Y\).
\begin{enumerate}
\item Fix any indicator function \(Y=\indic_{B}\) with \(B\in\mathcal{G}\).
For all \(A\in\mathcal{G}\),
\[
\expv{\vc{Y\expv{X|\mathcal{G}}}\indic_{A}}
=\expv{\indic_{B}\expv{X|\mathcal{G}}\indic_{A}}
=\expv{\expv{X|\mathcal{G}}\indic_{A\cap B}}
\overset{(A\cap B\in\mathcal{G})}{=}
\expv{X\indic_{A\cap B}}
=\expv{\indic_{B}X\indic_{A}}
=\expv{\vc{XY}\indic_{A}}.
\]
By linearity, the equality also holds for every simple \(Y\in\orc{\mathcal{G}}\)
(i.e., those with the form \(Y=\sum_{i=1}^{n}y_i\indic_{A_i}\) where
\(y_1,\dotsc,y_n\in\R\) and \(A_1,\dotsc,A_n\in\orc{\mathcal{G}}\) are pairwise
disjoint).
\item Fix any nonnegative \(Y\in\mathcal{G}\). Consider first the case
where \(X\ge 0\).

By \Cref{prp:apx-seq}, we know there exists a sequence \(\{Y_n\}\) of
nonnegative simple functions such that \(Y_n\nearrow Y\). Hence, for all
\(A\in\mathcal{G}\), we have \(Y_n\expv{X|\mathcal{G}}\indic_{A}\nearrow
Y\expv{X|\mathcal{G}}\indic_{A}\) and \(XY_n\indic_{A}\nearrow XY\indic_{A}\),
both pointwisely, which imply that
\[
\expv{\vc{Y\expv{X|\mathcal{G}}}\indic_{A}}
\overset{\text{(MCT)}}{=}\lim_{n\to\infty}\expv{Y_n\expv{X|\mathcal{G}}\indic_{A}}
\underset{\text{(partial averaging)}}{\overset{\text{(simple case)}}{=}}
\lim_{n\to\infty}\expv{XY_n\indic_{A}}
\overset{\text{(MCT)}}{=}
\expv{\vc{XY}\indic_{A}}.
\]
Next, consider the general case where \(X\) may not be nonnegative. Write
\(X=X^{+}-X^{-}\), and then we get
\begin{align*}
\expv{\vc{Y\expv{X|\mathcal{G}}}\indic_{A}}
&=\expv{Y\expv{X^{+}|\mathcal{G}}\indic_{A}}-\expv{Y\expv{X^{-}|\mathcal{G}}\indic_{A}}
\underset{\text{(partial averaging)}}{\overset{\text{(above case)}}{=}}
\expv{X^{+}Y\indic_{A}}-\expv{X^{-}Y\indic_{A}} \\
&=\expv{\vc{XY}\indic_{A}}
\end{align*}
for all \(A\in\mathcal{G}\).
\item Fix any \(Y\in\mathcal{G}\). Write \(Y=Y^{+}-Y^{-}\).
Then, for all \(A\in\mathcal{G}\) we have
\begin{align*}
\expv{\vc{Y\expv{X|\mathcal{G}}}\indic_{A}}
&=\expv{Y^{+}\expv{X|\mathcal{G}}\indic_{A}}
-\expv{Y^{-}\expv{X|\mathcal{G}}\indic_{A}}
\underset{\text{(partial averaging)}}{\overset{\text{(nonnegative case)}}{=}}
\expv{XY^{+}\indic_{A}}
-\expv{XY^{-}\indic_{A}} \\
&=\expv{\vc{XY}\indic_{A}}.
\end{align*}
\end{enumerate}
\end{enumerate}
\end{pf}

A corollary of the TOWIK property is a ``conditional version'' of
\Cref{prp:expv-prod-indp} (expectation of product is product of expectations),
with two random variables.
\begin{corollary}
\label{cor:cond-expv-prod-indp}
Let \((\Omega,\mathcal{F},\pr)\) be a probability space, and
\(\mathcal{G}\subseteq \mathcal{F}\) be a \(\sigma\)-algebra on \(\Omega\).  If
\(X\), \(Y\), and \(XY\) are integrable, and \(Y\) is independent of
\(\sigma(\sigma(X),\mathcal{G})\), then
\(\expv{XY|\mathcal{G}}=\expv{Y}\expv{X|\mathcal{G}}\).
\end{corollary}
\begin{pf}
Note that
\begin{align*}
\expv{XY|\mathcal{G}}\overset{\text{(tower, \(\mathcal{G}\subseteq
\sigma(\sigma(X),\mathcal{G}))\))}}&{=}
\expv{\expv{\vc{X}Y|\sigma(\vc{\sigma(X)},\mathcal{G})}|\mathcal{G}}
\overset{\text{(TOWIK)}}{=}
\expv{\vc{X}\expv{Y|\sigma(\sigma(X),\mathcal{G})}|\mathcal{G}} \\
\overset{\text{(\Cref{prp:cond-exp-prop})}}&{=}
\expv{X\expv{Y}|\mathcal{G}}=\expv{Y}\expv{X|\mathcal{G}}.
\end{align*}
\end{pf}
\item \textbf{Factorization/Doob-Dynkin lemma.} Here, we are going to study an
important result, known as \emph{factorization} or \emph{Doob-Dynkin lemma},
that justifies the method of finding conditional expectation learnt in your
first probability course. It turns out that, while the definition of
conditional expectation given here is rather abstract, it is possible to be
reduced to some more ``comprehensible'' forms that you have previously seen.

\begin{theorem}[Factorization/Doob-Dynkin lemma]
\label{thm:factorization}
Let \((\Omega,\mathcal{F},\pr)\) be a probability space,
\((\Omega',\mathcal{F}')\) be a measurable space, and \(Z:\Omega\to\Omega'\).
The function \(Y:\Omega\to\R\) is \((\sigma(Z),\mathcal{B}(\R))\)-measurable
iff there exists a \((\mathcal{F}',\mathcal{B}(\R))\)-measurable function
\vc{\(h:\Omega'\to\R\)} such that \(Y=h(Z)\).
\end{theorem}
\begin{center}
\begin{tikzpicture}
\node[scale=2] (om) at (0,0) {\(\Omega\)};
\node[scale=2] (omp) at (5,0) {\(\Omega'\)};
\node[scale=2] (er) at (10,0) {\(\R\)};
\node[] () at (11,0) {};
\draw[-Latex] (om) to[bend left] node[auto, scale=1.5]{\(Z\)} (omp);
\draw[-Latex, violet] (omp) to[bend left] node[auto, scale=1.5]{\(h\)} (er);
\draw[-Latex] (om) to[bend right] node[auto, swap, scale=1.5]{\(Y\)} (er);
\node[draw, dashed, scale=1.5] () at (0,-1) {\(\sigma(Z)\)};
\node[draw, dashed, scale=1.5] () at (5,-1) {\(\mathcal{F}'\)};
\node[draw, dashed, scale=1.5] () at (10,-1) {\(\mathcal{B}(\R)\)};
\draw[->, dashed, brown, thick] (5,-1.7) -- (2.5,1);
\draw[->, dashed, brown, thick] (5,-1.7) -- (7.5,1);
\node[brown] () at (5,-2.5) {``factorize''};
\end{tikzpicture}
\end{center}
\begin{pf}
``\(\Leftarrow\)'': Since \(Z\) is \((\sigma(Z),\mathcal{F}')\)-measurable
and \(h\) is \((\mathcal{F}',\mathcal{B}(\R))\)-measurable, by
\labelcref{it:meas-compo-meas} we know that \(Y=h(Z)\) is
\((\sigma(Z),\mathcal{B}(\R))\)-measurable.

``\(\Rightarrow\)'': Let \(Y\) be \((\sigma(Z),\mathcal{B}(\R))\)-measurable.
We then apply the standard argument on \(Y\).
\begin{enumerate}[label={(\arabic*)}]
\item Fix any simple \(Y=\sum_{i=1}^{n}y_i\indic_{A_i}\) with \(n\in\N\) and
\(A_i\in\sigma(Z)\) for every \(i=1,\dotsc,n\). Noting that \(A_i=Z^{-1}(A_i')\)
with \(A_i'\in\mathcal{F}'\) for each \(i=1,\dotsc,n\), we have
\[
Y(\omega)=\sum_{i=1}^{n}y_i\indic_{A_i}(\omega)
=\sum_{i=1}^{n}y_i\indic_{Z^{-1}(A_i')}(\omega)
\overset{(\omega\in Z^{-1}(A_i')\iff Z(\omega)\in A_i')}
{=}\sum_{i=1}^{n}y_i\indic_{A_i'}(Z(\omega))
=h(Z(\omega))
\]
where \(h(z):=\sum_{i=1}^{n}y_i\indic_{A_i'}(z)\) is
\((\mathcal{F}',\mathcal{B}(\R))\)-measurable.
\item Fix any \(Y\ge 0\) that is \((\sigma(Z),\mathcal{B}(\R))\)-measurable.
By \Cref{prp:apx-seq}, there exists a sequence \(\{Y_n\}\) of simple functions
such that \(Y_n\nearrow Y\). Now, for each \(n\in\N\), we know by (1) that
there exists a \((\mathcal{F}',\mathcal{B}(\R))\)-measurable function \(h_n\)
such that \(Y_n=h_n(Z)\). Since \(Y_n\nearrow\), we have \(h_n\nearrow\) also.
Thus, we have
\[
Y=\lim_{n\to\infty}Y_n=\lim_{n\to\infty}h_n(Z)=\sup_{n\in\N}h_n(Z)=:h(Z),
\]
where \(h(z)=\sup_{n\in\N}h_n(z)\) is
\((\mathcal{F}',\mathcal{B}(\R))\)-measurable by \labelcref{it:seq-rvs-meas}.
\item Fix any \((\sigma(Z),\mathcal{B}(\R))\)-measurable \(Y\). Write
\(Y=Y^{+}-Y^{-}\). By (2), there exist
\((\mathcal{F}',\mathcal{B}(\R))\)-measurable functions \(h^{+}\) and \(h^{-}\)
such that \(Y^{+}=h^{+}(Z)\) and \(Y^{-}=h^{-}(Z)\). Then, we have \(Y=h(Z)\)
with \(h:=h^{+}-h^{-}\) being \((\mathcal{F}',\mathcal{B}(\R))\)-measurable.
\end{enumerate}
\end{pf}

Using \Cref{thm:factorization}, we can derive the following corollary which
justifies the ``elementary'' way of computing conditional expectation.
\begin{corollary}[Factorization for conditional expectation]
\label{cor:fact-cond-exp}
Let \((\Omega,\mathcal{F},\pr)\) be a probability space. If \(X\in L^{1}\) and
\(\vect{Z}:\Omega\to\R^{d}\) is a random vector, then there
exists a measurable \(h:\R^{d}\to\R\) such that
\(\expv{X|\vect{Z}}=h(\vect{Z})\).
\end{corollary}
\begin{note}
For every \(\vect{z}\in\R^{d}\), the value \(\expv{X|\vect{Z}=\vect{z}}:=h(\vect{z})\)
is said to be the \defn{conditional expectation of \(X\) given \(\vect{Z}=\vect{z}\)}.
\end{note}

\begin{pf}
Note that \(X\) is \((\mathcal{F},\mathcal{B}(\R))\)-measurable, \(\vect{Z}\)
is \((\mathcal{F},\mathcal{B}(\R^{d}))\)-measurable, and \(\expv{X|\vect{Z}}\)
is \((\sigma(Z),\mathcal{B}(\R))\)-measurable. Therefore, the result follows by
taking \(\Omega'=\R^{d}\), \(\mathcal{F}'=\mathcal{B}(\R^{d})\), and
\(Y=\expv{X|\vect{Z}}\) in \Cref{thm:factorization}.
\end{pf}
\item \textbf{Factorization of conditional expectation under independence.}
Both \Cref{thm:factorization} and \Cref{cor:fact-cond-exp} suggest the
\emph{existence} of factorization, but do not give us an explicit ``formula''
for finding out what \(h\) is.\footnote{While the proof of
\Cref{thm:factorization} does suggest how such \(h\) can be constructed,
it is not in a very explicit form and is generally hard to obtain.} In the
following, we will consider a special case where we do have an explicit formula
for such \(h\).

\begin{proposition}
\label{prp:fact-indp}
Let \((\Omega,\mathcal{F},\pr)\) be a probability space.  If
\(\vect{X}:\Omega\to\R^{d_{\vect{x}}}\) and
\(\vect{Z}:\Omega\to\R^{d_{\vect{z}}}\) are independent random vectors, and
\(g:\R^{d_{\vect{x}}+d_{\vect{z}}}\to\R\) is a measurable function such that
\(g(\vect{X},\vect{Z})\in L^{1}\), then
\(\expv{g(\vect{X},\vect{Z})|\vect{Z}}=h(\vect{Z})\) (a.s.), where
\(h(\vect{z}):=\expv{g(\vect{X},\vect{z})}\).
\end{proposition}

\begin{pf}
\begin{enumerate}[label={(\arabic*)}]
\item It can be shown that \(h\) is
\((\mathcal{B}(\R^{d_{\vect{z}}}),\mathcal{B}(\R))\)-measurable (see
\textcite[Section 10.3]{resnick2014probability} for more details), hence
\((\mathcal{B}(\R^{d_{\vect{z}}}),\sigma(\vect{Z}))\)-measurable.  Together
with the assumption that \(\vect{Z}\) is
\((\sigma(\vect{Z}),\mathcal{B}(\R^{d}))\)-measurable, by
\labelcref{it:meas-compo-meas} we know \(h(\vect{Z})\) is
\(\sigma(\vect{Z})\)-measurable.
\item Fix any \(A\in\sigma(\vect{Z})\). Then, we can write \(A=\vect{Z}^{-1}(B)\)
for some \(B\in\mathcal{B}(\R^{d_{\vect{z}}})\). With this expression,
we note that \(\indic_{A}(\omega)=\indic_{\vect{Z}^{-1}(B)}(\omega)
=\indic_{B}(\vect{Z}(\omega))\) for all \(\omega\in\Omega\). Now, since
\(g(\vect{X},\vect{Z})\in L^{1}\), applying the Fubini's theorem yields
\begin{align*}
\int_{A}^{}g(\vect{X},\vect{Z})\odif{\pr}
&=\int_{\Omega}^{}g(\vect{X},\vect{Z})\indic_{A}\odif{\pr}
=\int_{\Omega}^{}g(\vect{X},\vect{Z})\indic_{B}(\vect{Z})\odif{\pr} \\
\overset{\text{(\Cref{thm:change-of-var})}}&{=}
\int_{\R^{d_{\vect{x}}+d_{\vect{z}}}}^{}g(\vect{x},\vect{z})\indic_{B}(\vect{z})\odif{
{F_{\vect{X},\vect{Z}}(\vect{x},\vect{z})}} \\
\overset{\text{(Fubini, independence)}}&{=}
\int_{\R^{d_{\vect{z}}}}^{}
\int_{\R^{d_{\vect{x}}}}^{}
g(\vect{x},\vect{z})\indic_{B}(\vect{z})
\odif{F_{\vect{X}}(\vect{x})}\odif{F_{\vect{Z}}(\vect{z})} \\
&=\int_{\R^{d_{\vect{z}}}}^{}
\indic_{B}(\vect{z})
\underbrace{\int_{\R^{d_{\vect{x}}}}^{}
g(\vect{x},\vect{z})\odif{F_{\vect{X}}(\vect{x})}}
_{\expv{g(\vect{X},\vect{z})}=h(\vect{z})}
\odif{F_{\vect{Z}}(\vect{z})} \\
&=\int_{\R^{d_{\vect{z}}}}^{}\indic_{B}(\vect{z})h(\vect{z})\odif{F_{\vect{Z}}(\vect{z})}
\overset{\text{(\Cref{thm:change-of-var})}}{=}
\int_{\Omega}^{}\indic_{B}(\vect{Z})h(\vect{Z})\odif{\pr} \\
&=\int_{\Omega}^{}\indic_{A}h(\vect{Z})\odif{\pr}
=\int_{A}^{}h(\vect{Z})\odif{\pr}.
\end{align*}
\end{enumerate}
\end{pf}

\begin{note}
Based on \Cref{prp:fact-indp}, we can compute conditional expectation of the
form \(\expv{g(\vect{X},\vect{Z})|\vect{Z}}\), where \(\vect{X}\) and
\(\vect{Z}\) are independent, by first computing
\(h(\vect{z})=\expv{g(\vect{X},\vect{z})}\) \emph{(replacing \(\vect{Z}\) by
\(\vect{z}\) in the expression and removing the condition)}, and then conclude
that the conditional expectation is \(h(\vect{Z})\) \emph{(replacing every
\(\vect{z}\) by \(\vect{Z}\) in the final expression)}.
\end{note}
\end{enumerate}
\subsection{Applications of Conditional Expectation}
\begin{enumerate}
\item After having a solid foundation on the measure-theoretic backbone of
conditional expectation in \Cref{subsect:meas-theoretic-cond}, we now study
some applications of conditional expectation, including deviations of formulas
about \emph{conditional distribution functions}, \emph{variances}, and
\emph{random sums}, and also about the usage of conditional expectation in
\emph{regression analysis}.
\item \textbf{Conditional distribution formula.}
Conditional expectation can be used for deriving the following versatile
formula for computing distribution functions via conditioning.

\begin{proposition}
\label{prp:cond-dist-fmla}
Let \(\vect{X}\) and \(\vect{Z}\) be random vectors. If
\((\vect{X},\vect{Z})\sim F\) for a \((d_{\vect{x}}+d_{\vect{z}})\)-dimensional
distribution function \(F\), then
\[
F(\vect{x},\vect{z})=\int_{(\vect{-\infty},\vect{z}]}^{}
F_{\vect{X}|\vect{Z}}(\vect{x}|\widetilde{\vect{z}})
\odif{F_{\vect{Z}}(\widetilde{\vect{z}})}
\quad\text{for all \((\vect{x},\vect{z})\in\R^{d_{\vect{x}}+d_{\vect{z}}}\).}
\]
\end{proposition}
\begin{pf}
Let \(h(\widetilde{\vect{z}})=\expv{\indicset{\vect{X}\le\vect{x}}
|\vect{Z}=\widetilde{\vect{z}}}=\prob{\vect{X}\le\vect{x}|\vect{Z}=\widetilde{\vect{z}}}
=F_{\vect{X}|\vect{Z}}(\vect{x}|\widetilde{\vect{z}})\). Consider
\begin{align*}
F(\vect{x},\vect{z})
&=\expv{\indicset{\vect{X}\le\vect{x},\vect{Z}\le\vect{z}}}
=\expv{\indicset{\vect{X}\le\vect{x}}\indicset{\vect{Z}\le\vect{z}}}
\overset{\text{(total expectation)}}{=}
\expv{\expv{\indicset{\vect{X}\le\vect{x}}
\vc{\indicset{\vect{Z}\le\vect{z}}}|\vect{Z}}} \\
\overset{\text{(TOWIK)}}&{=}
\expv{\vc{\indicset{\vect{Z}\le\vect{z}}}
\expv{\indicset{\vect{X}\le\vect{x}}|\vect{Z}}}
=\expv{\indicset{\vect{Z}\le\vect{z}}h(\vect{Z})}
\overset{\text{(\Cref{thm:change-of-var})}}{=}
\int_{\R^{d_{\vect{z}}}}^{}
\indicset{\widetilde{\vect{z}}\le\vect{z}}h(\widetilde{\vect{z}})
\odif{F_{\vect{Z}}(\widetilde{\vect{z}})} \\
&=\int_{(\vect{-\infty},\vect{z}]}^{}
h(\widetilde{\vect{z}})\odif{F_{\vect{Z}}(\widetilde{\vect{z}})}
=\int_{(\vect{-\infty},\vect{z}]}^{}
F_{\vect{X}|\vect{Z}}(\vect{x}|\widetilde{\vect{z}})
\odif{F_{\vect{Z}}(\widetilde{\vect{z}})}.
\end{align*}
\end{pf}

\item \textbf{Law of total variance.} Apart from the \emph{law of total
expectation} (\Cref{prp:cond-exp-prop}), there is also the \emph{law of total
variance}, but the formula is slightly more complex. The following lemma,
which gives a conditional version of the ``2nd moment minus 1st moment
squared'' formula for the variance, is helpful for establishing the law of
total variance.

\begin{lemma}
\label{lma:cond-var-mom-diff}
Let \((\Omega,\mathcal{F},\pr)\) be a probability space, \(\mathcal{G}\subseteq
\mathcal{F}\) be a \(\sigma\)-algebra, and \(X\in
L^{\vc{2}}(\Omega,\mathcal{F},\pr)\). Then, \(\vari{X|\mathcal{G}}
=\expv{X^{2}|\mathcal{G}}-\expv{X|\mathcal{G}}^{2}\).
\end{lemma}
\begin{pf}
Note that
\begin{align*}
\vari{X|\mathcal{G}}&=\expv{(X-\expv{X|\mathcal{G}})^{2}|\mathcal{G}}
\overset{\text{(linearity)}}{=}
\expv{X^{2}|\mathcal{G}}-2\expv{X\vc{\expv{X|\mathcal{G}}}|\mathcal{G}}
+\expv{\vc{(\expv{X|\mathcal{G}})^{2}}|\mathcal{G}} \\
\overset{\text{(TOWIK twice)}}&{=}
\expv{X^{2}|\mathcal{G}}-2\vc{\expv{X|\mathcal{G}}}\expv{X|\mathcal{G}}
+\vc{(\expv{X|\mathcal{G}})^{2}}
=\expv{X^{2}|\mathcal{G}}-(\expv{X|\mathcal{G}})^{2}.
\end{align*}
\end{pf}
\begin{proposition}[Law of total variance]
\label{prp:law-of-total-var}
Let \((\Omega,\mathcal{F},\pr)\) be a probability space, \(\mathcal{G}\subseteq
\mathcal{F}\) be a \(\sigma\)-algebra, and \(X\in
L^{\vc{2}}(\Omega,\mathcal{F},\pr)\). Then, \(\vari{X}=\expv{\vari{X|\mathcal{G}}}
+\vari{\expv{X|\mathcal{G}}}\).
\end{proposition}
\begin{pf}
We have
\begin{align*}
\vari{X}&=\expv{X^{2}}-\expv{X}^{2}
\overset{\text{(total expectation twice)}}{=}
\expv{\vc{\expv{X^{2}|\mathcal{G}}}}-\expv{\expv{X|\mathcal{G}}}^{2} \\
\overset{\text{(\Cref{lma:cond-var-mom-diff})}}&{=}
\expv{\vc{\vari{X|\mathcal{G}}+\expv{X|\mathcal{G}}^{2}}}
-\expv{\expv{X|\mathcal{G}}}^{2}
\overset{\text{(linearity)}}{=}
\expv{\vari{X|\mathcal{G}}}+\expv{\orc{\expv{X|\mathcal{G}}}^{2}}
-\expv{\orc{\expv{X|\mathcal{G}}}}^{2} \\
&=\expv{\vari{X|\mathcal{G}}}+\vari{\orc{\expv{X|\mathcal{G}}}}.
\end{align*}
\end{pf}
\item \textbf{Formulas of expectation and variance of random sum.} In
\labelcref{it:mixture-dist}, we have seen the appearance of random variable
like \(S=\sum_{i=1}^{N}X_i\), which is a \emph{random sum} with both the number
of summands and the summands themselves being random variables. To compute the
expectation and variance of such random sum, the following formulas are often
used.

\begin{proposition}
\label{prp:random-sum-fmla}
Let \(N,X_1,X_2,\dotsc\) be independent, \(N\in\N_0\), \(X_1,X_2,\dotsc\) be
identically distributed, and \(S=\sum_{i=1}^{N}X_i\).
\begin{enumerate}
\item \emph{(Wald's equation)} If \(N,X_1\in L^{1}\), then \(S\in L^{1}\) and
\(\expv{S}=\expv{N}\expv{X_1}\).
\item \emph{(Blackwell-Girshick equation)} If \(N,X_1\in L^{2}\), then \(S\in
L^{2}\) and \(\vari{S}=\expv{N}\vari{X_1}+\vari{N}\expv{X_1}^{2}\).
\end{enumerate}
\end{proposition}
\begin{pf}
\begin{enumerate}
\item We first show \(S\in L^{1}\). Let \(S_n=\sum_{i=1}^{n}X_i\) for each
\(n\in\N_0\). Then, we have \(S=\sum_{n=1}^{\infty}S_n\indicset{N=n}\),
and thus
\begin{align*}
\expv{|S|}\overset{\text{(triangle)}}&{\le}\sum_{n=1}^{\infty}\expv{|S_n|\indicset{N=n}}
\overset{\text{(independence)}}{=}
\sum_{n=1}^{\infty}\vc{\expv{|S_n|}}\expv{\indicset{N=n}} \\
\overset{\text{(triangle)}}&{\le}
\sum_{n=1}^{\infty}\vc{n\expv{|X_1|}}\prob{N=n}
=\expv{|X_1|}\sum_{n=1}^{\infty}n\prob{N=n}
=\expv{|X_1|}\expv{N}<\infty.
\end{align*}
Repeating the same argument without the absolute value \(|\cdot|\)
(inequalities becomes equalities as there is no absolute value) yields
\(\expv{S}=\expv{N}\expv{X_1}\). Alternatively, we can apply
\Cref{prp:fact-indp} as follows. Noting that
\(S=\sum_{i=1}^{N}X_i=g(\vect{X},N)\in L^{1}\), we have
\[
\expv{S|N}=\expv{g(\vect{X},N)|N}\overset{\text{(\Cref{prp:fact-indp})}}{=}
h(N),
\]
where
\(h(n)=\expv{g(\vect{X},n)}=\expv{\sum_{i=1}^{n}X_i}=\sum_{i=1}^{n}\expv{X_i}
=n\expv{X_1}\).
Hence, we get
\[
\expv{S}\overset{\text{(total expectation)}}{=}
\expv{\expv{S|N}}=\expv{h(N)}=\expv{N\expv{X_1}}=\expv{N}\expv{X_1}.
\]
\item Again, we first show \(S\in L^{2}\). Let \(S_n=\sum_{i=1}^{n}X_i\) for
each \(n\in\N_0\). Then, we have
\(S^{2}=\sum_{n=1}^{\infty}S_n^{2}\indicset{N=n}\), and thus
\begin{align*}
\expv{S^{2}}&=\sum_{n=1}^{\infty}\expv{S_n^{2}\indicset{N=n}}
\overset{\text{(independence)}}{=}\sum_{n=1}^{\infty}\expv{S_n^{2}}\prob{N=n} \\
&=\sum_{n=1}^{\infty}(\vari{S_n}+\expv{S_n}^{2})\prob{N=n}
\underset{\text{\orc{(linearity)}}}{\overset{\text{\vc{(independence)}}}{=}}
\sum_{n=1}^{\infty}(\vc{n\vari{X_1}}+(\orc{n\expv{X_1}})^{2})\prob{N=n} \\
&=\expv{N}\vari{X_1}+\expv{N^{2}}\expv{X_1}^{2}<\infty,
\end{align*}
which implies that \(S\in L^{2}\).

Next, to show the equation, there are again two approaches:
\begin{itemize}
\item \emph{Method 1: Using the relationship \(\vari{S}=\expv{S^{2}}-\expv{S}^{2}\).}
Note that
\[
\vari{S}=\expv{N}\vari{X_1}+\expv{N^{2}}\expv{X_1}^{2}-(\expv{N}\expv{X_1})^{2}
=\expv{N}\vari{X_1}+\expv{N^{2}}\vari{X_1}.
\]
\item \emph{Method 2: Using \Cref{prp:fact-indp}.}
First, we have
\begin{align*}
\vari{S|N}&=\expv{(S-\expv{S|N})^{2}|N}
\overset{\text{(see (a))}}{=}\expv{(S-N\expv{X_1})^{2}|N} \\
\overset{(g(\vect{X},N)=(S-N\expv{X_1})^{2})}&{=}\expv{g(\vect{X},N)|N}
\overset{\text{(\Cref{prp:fact-indp})}}{=}h(N)
\end{align*}
where
\(h(n)=\expv{g(\vect{X},\mgc{n})}=\expv{(\sum_{i=1}^{\mgc{n}}X_i-\mgc{n}\expv{X_1})^{2}}=
\vari{\sum_{i=1}^{n}X_i}\overset{\text{(independence)}}{=}n\vari{X_1}\). Hence,
\(\vari{S|N}=N\vari{X_1}\). Therefore, by the law of total variance,
\begin{align*}
\vari{S}&=\expv{\vari{S|N}}+\vari{\expv{S|N}}
\overset{\text{(see (a))}}{=}\expv{N\vari{X_1}}+\vari{N\expv{X_1}} \\
&=\expv{N}\vari{X_1}+\vari{N}\expv{X_1}^{2}.
\end{align*}
\end{itemize}
\end{enumerate}
\end{pf}
\item \textbf{Role of conditional expectation in regression.} In statistics
(regression analysis), the function
\(\vect{z}\mapsto\expv{g(\vect{X},\vect{Z})|\vect{Z}=\vect{z}}:=h(\vect{z})\)
is called the \defn{regression function} of \(g(\vect{X},\vect{Z})\) on
\(\vect{Z}\) at \(\vect{z}\). If \(g(\vect{X},\vect{Z})\in L^{2}\), then
\(h(\vect{z})\) is the best \(L^{2}\)-approximation of \(g(\vect{X},\vect{Z})\)
with \(\vect{Z}=\vect{z}\) observed, in terms of mean squared error, as the
following result illustrates.

\begin{proposition}
\label{prp:cond-exp-best-l2}
Let \((\Omega,\mathcal{F},\pr)\) be a probability space, \(\mathcal{G}\subseteq
\mathcal{F}\) be a \(\sigma\)-algebra, and \(X\in
L^{2}(\Omega,\mathcal{F},\pr)\). Then, \(\expv{X|\mathcal{G}}\) is the
\(Y\in L^{2}(\Omega,\vc{\mathcal{G}},\pr)\) which minimizes
\(\expv{(X-Y)^{2}}\), i.e.,
\[
\expv{(X-Y)^{2}}=\min_{Z\in L^{2}(\Omega,\vc{\mathcal{G}},\pr)}\expv{(X-Z)^{2}}
\quad\text{iff}\quad
Y\eqas \expv{X|\mathcal{G}}.
\]
\end{proposition}
\begin{pf}
Note first that
\begin{align*}
\expv{(X-Y)^{2}}&=\expv{(X-\expv{X|\mathcal{G}}+\expv{X|\mathcal{G}}-Y)^{2}} \\
&=\underbrace{\expv{(X-\expv{X|\mathcal{G}})^{2}}}_{\text{does not depend on \(Y\)}}
+2\expv{(X-\expv{X|\mathcal{G}})(\expv{X|\mathcal{G}}-Y)}
+\underbrace{\expv{(\expv{X|\mathcal{G}}-Y)^{2}}}
_{\mathclap{\text{smallest \((=0)\) iff \(Y\eqas \expv{X|\mathcal{G}}\)
by \labelcref{it:nonneg-leb-int-prop}}}}.
\end{align*}
Therefore, it suffices to show that
\(\expv{(X-\expv{X|\mathcal{G}})(\expv{X|\mathcal{G}}-Y)}=0\) for every \(Y\in
L^{2}(\Omega,\vc{\mathcal{G}},\pr)\):
\begin{align*}
&\quad\expv{(X-\expv{X|\mathcal{G}})(\expv{X|\mathcal{G}}-Y)}
\overset{\text{(total expectation)}}{=}
\expv{\expv{(X-\expv{X|\mathcal{G}})\vc{(\expv{X|\mathcal{G}}-Y)}|\mathcal{G}}} \\
\overset{\text{(TOWIK)}}&{=}
\expv{\vc{(\expv{X|\mathcal{G}}-Y)}
\expv{(X-\expv{X|\mathcal{G}})|\mathcal{G}}}
\overset{\text{(linearity)}}{=}
\expv{(\expv{X|\mathcal{G}}-Y)
(\expv{X|\mathcal{G}}-\expv{\orc{\expv{X|\mathcal{G}}}|\mathcal{G}})} \\
\overset{\text{(TOWIK)}}&{=}
\expv{(\expv{X|\mathcal{G}}-Y)
(\expv{X|\mathcal{G}}-\orc{\expv{X|\mathcal{G}}})}
=\expv{0}=0.
\end{align*}
\end{pf}

More geometrically, the conditional expectation \(\expv{X|\mathcal{G}}\) can
actually be interpreted as the \emph{orthogonal projection} of \(X\) onto the
closed subspace \(L^{2}(\Omega,\mathcal{G},\pr)\) of the Hilbert space
\(L^{2}(\Omega,\mathcal{F},\pr)\) equipped with the inner product \(\inner{X}{Y}:=
\expv{XY}\). Here, such orthogonal projection is given by
\[
\argmin_{Y\in L^{2}(\Omega,\mathcal{G},\pr)}\expv{(X-Y)^{2}}
\overset{\text{(\Cref{prp:cond-exp-best-l2})}}{=}
\expv{X|\mathcal{G}}.
\]
\begin{center}
\begin{tikzpicture}
\draw[-Latex, blue] (0,0) -- (3,2) node[pos=1.1]{\(X\)};
\draw[green, opacity=0.2, fill] (-5,-3) rectangle (5,3);
\node[] () at (-3,1.5) {\(L^{2}(\Omega,\mathcal{F},\pr)\)};
\draw[violet, thick] (-5,0) -- (5,0);
\node[] () at (-3,-0.5) {\(L^{2}(\Omega,\vc{\mathcal{G}},\pr)\)};
\draw[-Latex, magenta] (0,0) --node[auto, swap]{\(Y=\expv{X|\mathcal{G}}\)} (3,0);
\draw[-Latex, dashed] (3,2) --node[auto]{\(Y-X\)} (3,0);
\end{tikzpicture}
\end{center}
We can verify the orthogonality as follows:
\begin{align*}
\inner{Y}{Y-X}&=\expv{Y(Y-X)}
=\expv{\expv{X|\mathcal{G}}(\expv{X|\mathcal{G}}-X)}
=\expv{\expv{X|\mathcal{G}}^{2}-X\expv{X|\mathcal{G}}}
=\expv{\expv{X|\mathcal{G}}^{2}}-\expv{X\expv{X|\mathcal{G}}} \\
\overset{\text{(total expectation)}}&{=}
\expv{\expv{X|\mathcal{G}}^{2}}-\expv{\expv{X\vc{\expv{X|\mathcal{G}}}|\mathcal{G}}}
\overset{\text{(TOWIK)}}{=}
\expv{\expv{X|\mathcal{G}}^{2}}-\expv{\vc{\expv{X|\mathcal{G}}}\expv{X|\mathcal{G}}} \\
&=\expv{\expv{X|\mathcal{G}}^{2}}-\expv{\expv{X|\mathcal{G}}^{2}}
=0.
\end{align*}
\end{enumerate}
