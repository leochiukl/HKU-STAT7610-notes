\section{Integration and Expectation}
\label{sect:integration-expectation}
\begin{enumerate}
\item Apart from working with probability measures, often we would like to
compute \emph{expectations} in probability theory. In your first probability
course, you should have learnt some formulas for computing expectations like
below:
\[
\expv{X}=\begin{cases}
\sum_{k}^{}x_kf(x_k)&\text{if \(X\) is discrete, taking countably many values \(x_1,x_2,\dotsc\)}, \\
\int_{-\infty}^{\infty}xf(x)\odif{x}&\text{if \(X\) is continuous}
\end{cases}
\]
where \(f\) denotes the mass function (for discrete case) or the density
function (for continuous case). However, we will face difficulties when \(X\)
is \emph{neither discrete nor continuous} (this is possible; the definition
for \(X\) to be continuous is not ``\(X\) is not discrete''\warn{}).

While these formulas are relatively easy to work with, they are not flexible
enough for working with general random variables. Hence, in
\Cref{sect:integration-expectation} we will introduce a more general (yet more
abstract) definition of expectation, which is closely related to the notion of
\emph{Lebesgue integral}; let us start by studying the construction of Lebesgue
integrals in \Cref{subsect:construct-lebesgue-int}, to better understand this
concept.
\end{enumerate}
\subsection{Construction of Lebesgue Integrals}
\label{subsect:construct-lebesgue-int}
\begin{enumerate}
\item \textbf{Motivation.} The formulas of expectations mentioned above are
both capturing the idea of ``weighted average''; the expectation is considered
to be an average weighted according to the probabilities. This idea is
preserved in the general and abstract definition of expectation.

Consider the special case where the sample space \(\Omega\) is countable, which
forces all random variables defined on \(\Omega\) to be discrete. Hence, we can
express the expectation of a random variable \(X\) as follows:
\[
\expv{X}=\sum_{\omega\in\Omega}^{}X(\omega)\prob{\{\omega\}}
\overset{\text{(in words)}}{=}
\sum_{\text{all outcomes}}^{}\text{outcome}\times \text{probability of having that outcome}
\]
which is an alternative expression of the discrete expectation formula above.
To extend this idea to the general case, we need to make sense of ``weighted
average'' in the case where \(\Omega\) is \emph{uncountable} also. One natural
idea is to replace ``\(\sum\)'' by ``\(\int\)'' in such case, and consider
something like ``\(\int_{\Omega}^{}X(\omega)\odif{\prob{\omega}}\)'' or
``\(\int_{\Omega}^{}X\odif{\pr}\)'', just like the development of Riemann
integral.  Our goal here is to make sense of this kind of integral in a more
general setting, where a general measure \(\mu\) is considered.
\item \textbf{Steps for the construction.} Let \((\Omega,\mathcal{F},\mu)\) be
a measure space and \(X:\Omega\to\R\) be a measurable function. Write
\(\bar{\R}_{+}:=[0,\infty]\) and \(\bar{\R}:=[-\infty,\infty]\).  Then, the
\emph{Lebesgue integral of \(X\) with respect to \(\mu\)}, denoted by
\(\int_{\Omega}^{}X(\omega)\odif{\mu(\omega)}\) or
\(\int_{\Omega}^{}X\odif{\mu}\), can be constructed in the following three-step
process:
\begin{enumerate}[label={(\arabic*)}]
\item Define \(\int_{\Omega}^{}X\odif{\mu}\) for every \emph{simple function} \(X:\Omega\to\R\).
\item Extends the notion of \(\int_{\Omega}^{}X\odif{\mu}\) to every
\emph{nonnegative} \((\mathcal{F},\mathcal{B}(\bar{\R}_{+}))\)-measurable function
\(X:\Omega\to\bar{\R}_{+}\).
\item Extends the notion of \(\int_{\Omega}^{}X\odif{\mu}\) to every
\((\mathcal{F},\mathcal{B}(\bar{\R}))\)-measurable function
\(X:\Omega\to\bar{\R}\).

\end{enumerate}
More generally, in many measure theoretic proofs, this kind of procedure is
also utilized, which is known as the \defn{standard argument}:
\begin{enumerate}[label={(\arabic*)}]
\item Show that the target result holds for every \emph{simple function} \(X:\Omega\to\R\).
\item Show that the target result holds for every
\emph{nonnegative} \((\mathcal{F},\mathcal{B}(\bar{\R}_{+}))\)-measurable function
\(X:\Omega\to\bar{\R}_{+}\).
\item Show that the target result holds for every
\((\mathcal{F},\mathcal{B}(\bar{\R}))\)-measurable function
\(X:\Omega\to\bar{\R}\).
\end{enumerate}
Very often, step 2 is the step that takes the most work. The steps 1 and 3 are
usually quite easy.
\begin{remark}
\item Sometimes we may replace \(\eR_{+}\) and \(\eR\) by \(\R_{+}\) and \(\R\)
respectively, depending on the result we are trying to prove.
\item For an illustration of the standard argument, see the proof of \Cref{thm:change-of-var}.
\end{remark}
\subsubsection*{Lebesgue integrals of simple functions}
\item \textbf{Definition.} Let
(\(\Omega,\mathcal{F},\mu)\) be a measure space. Recall that a function
\(X:\Omega\to\R\) is said to be simple if it takes the form
\(X=\sum_{i=1}^{n}x_i\indic_{A_i}\), where \(x_1,\dotsc,x_n\in\R\), \(n\in\N\),
and \(A_1,\dotsc,A_n\in\mathcal{F}\) are pairwise disjoint.

Now consider a simple function \(X=\sum_{i=1}^{n}x_i\indic_{A_i}\). The
\defn{Lebesgue integral of \(X\) with respect to \(\mu\)} is
\[
\int_{\Omega}^{}X\odif{\mu}:=\sum_{i=1}^{n}x_i\mu(A_i),
\]
with the convention that \(0\cdot \infty=0\) (i.e., \(x_i\mu(A_i)=0\) if
\(x_i=0\) and \(\mu(A_i)=\infty\) for some \(i\)). Also, we write
\(\int_{A}^{}X\odif{\mu}:=\int_{\Omega}^{}X\indic_{A}\odif{\mu}\) for every \(A\in\mathcal{F}\).

\begin{center}
\begin{tikzpicture}
\begin{axis}[domain=0:5, xmin=-0.5, xmax=4.5, ymax=19, ymin=-5.5, axis y
line=middle, axis x line=none, xlabel={\(x\)}, ylabel={\(X(\omega)\)}, xlabel
style={anchor=west}, ylabel style={anchor=south}, xtick={1,4},
xticklabels={\(a\),\(b\)}, ytick={2,5,10,14}, yticklabels={\(x_1\), \(x_2\),
\(x_3\), \(x_4\)}, samples=50]
\addplot[blue, thick, domain=0:0.7]{2};
\addplot[blue, thick, domain=1.7:2]{2};
\draw[green, fill, opacity=0.2] (0,2) rectangle (0.7,0);
\draw[green, fill, opacity=0.2] (1.7,2) rectangle (2,0);
\draw[-Latex, ForestGreen] (1,-2) -- (0.35,0);
\draw[-Latex, ForestGreen] (1,-2) -- (1.85,0);
\node[ForestGreen] () at (1,-3) {\(x_1\mu(A_1)\)};
\addplot[blue, thick, domain=0.7:1.7]{5};
\draw[orange, fill, opacity=0.2] (0.7,5) rectangle (1.7,0);
\draw[-Latex, orange] (2,-2) -- (1.2,0);
\node[orange] () at (2,-3) {\(x_2\mu(A_2)\)};
\addplot[blue, thick, domain=2:2.5]{10};
\addplot[blue, thick, domain=3.5:5]{10};
\draw[violet, fill, opacity=0.2] (2,10) rectangle (2.5,0);
\draw[violet, fill, opacity=0.2] (3.5,10) rectangle (5,0);
\draw[-Latex, violet] (3.5,-2) -- (2.25,0);
\draw[-Latex, violet] (3.5,-2) -- (4.25,0);
\node[violet] () at (3.5,-3) {\(x_3\mu(A_3)\)};
\addplot[blue, thick, domain=2.5:3.5]{14};
\draw[magenta, fill, opacity=0.2] (2.5,14) rectangle (3.5,0);
\draw[-Latex, magenta] (2.75,-3) -- (3,0);
\node[magenta] () at (2.75,-4.5) {\(x_4\mu(A_4)\)};
\end{axis}
\end{tikzpicture}
\end{center}
Writing \(\sum_{i=1}^{n}x_i\mu(A_i)\) as \(\sum_{i=1}^{n}x_i\mu(X=x_i)\), we
can see that it aligns with the formula of expectation of discrete random
variable \(X\) taking values \(x_1,\dotsc,x_n\) (when \(\mu\) is a probability
measure).  Furthermore, from a geometric point of view, the integral
\(\int_{\Omega}^{}X\odif{\mu}\) also captures the notion of ``area under
curve'', like the Riemann integral.

\item \label{it:simp-leb-int-prop} \textbf{Properties.} Let
\(X,Y:\Omega\to\R\) be simple functions, say \(X=\sum_{i=1}^{n}x_i\mu(A_i)\)
and \(Y=\sum_{j=1}^{m}y_j\mu(B_j)\). Then:
\begin{enumerate}
\item \emph{(nonnegativity)} \(X\ge 0\implies \int_{\Omega}^{}X\odif{\mu}\ge 0\).
\item \emph{(linearity)}
\(\int_{\Omega}^{}aX+bY\odif{\mu}=a\int_{\Omega}^{}X\odif{\mu}+b\int_{\Omega}^{}Y\odif{\mu}\),
for all \(a,b\in\R\).
\item \emph{(monotonicity)} \(X\le Y\implies
\int_{\Omega}^{}X\odif{\mu}\le\int_{\Omega}^{}Y\odif{\mu}\).
\item If \(X\ge 0\), then the function \(\nu\) given by
\(\nu(A):=\int_{A}^{}X\odif{\mu}~\forall A\in\mathcal{F}\) is a measure on \((\Omega,\mathcal{F})\).
\item \(\int_{\Omega}^{}\indic_{A}\odif{\mu}=\mu(A)\) for every \(A\in\mathcal{F}\).
\end{enumerate}
\begin{pf}
\begin{enumerate}
\item Since \(X\ge 0\), we have \(x_i\ge 0\) for every \(i=1,\dotsc,n\), and hence
\(\int_{\Omega}^{}X\odif{\mu}=\sum_{i=1}^{n}x_i\mu(A_i)\ge 0\).
\item Note that \(aX+bY=\sum_{i,j}^{}(ax_i+by_j)\indic_{A_i\cap B_j}\) is still
simple, so
\begin{align*}
\int_{\Omega}^{}aX+bY\odif{\mu}
&=\sum_{i,j}^{}(ax_i+by_j)\mu(A_i\cap B_j)
=a\sum_{i=1}^{n}x_i\vc{\sum_{j=1}^{m}\mu(A_i\cap B_j)}
+b\sum_{j=1}^{m}y_j\orc{\sum_{i=1}^{n}\mu(A_i\cap B_j)} \\
\overset{\text{(law of total measure)}}&{=}a\sum_{i=1}^{n}x_i\vc{\mu(A_i)}
+b\sum_{j=1}^{m}y_j\orc{\mu(B_j)}=a\int_{\Omega}^{}X\odif{\mu}+b\int_{\Omega}^{}Y\odif{\mu}.
\end{align*}
\item Note that \(Y-X\ge 0\) is still simple, and thus
\[
\int_{\Omega}^{}Y\odif{\mu}
\overset{\text{(linearity)}}{=}\int_{\Omega}^{}Y-X\odif{\mu}+\int_{\Omega}^{}X\odif{\mu}
\overset{\text{(monotonicity)}}{\ge}\int_{\Omega}^{}X\odif{\mu}.
\]
\item \begin{enumerate}[label={(\arabic*)}]
\item For every \(A\in\mathcal{F}\), we have \(X\ge 0\implies X\indic_{A}\ge
0\implies \mu(A)\ge 0\).
\item Note that \(\nu(\varnothing)=\int_{\Omega}^{}X\indic_{\varnothing}\odif{\mu}
=\int_{\Omega}^{}0\odif{\mu}\overset{\text{(simple integrand)}}{=}0\mu(\Omega)=0\)
(even if \(\mu(\Omega)=\infty\)).
\item Fix any pairwise disjoint \(C_1,C_2,\dotsc\in\mathcal{F}\). Then,
\begin{align*}
\mu\left(\biguplus_{k=1}^{\infty}C_k\right)
&=\int_{\Omega}^{}X\indic_{\biguplus_{k=1}^{\infty}C_k}\odif{\mu}
=\int_{\Omega}^{}\sum_{i=1}^{n}x_i\indic_{A_i\cap \biguplus_{k=1}^{\infty}C_k}\odif{\mu}
\overset{\text{(simple integrand)}}{=}
\sum_{i=1}^{n}x_i\mu\left(A_i\cap \biguplus_{k=1}^{\infty}C_k\right) \\
\overset{\text{(\(\sigma\)-additivity)}}&{=}
\sum_{i=1}^{n}x_i\sum_{k=1}^{\infty}\mu(A_i\cap c_k)
\underset{\text{(Riemann series theorem)}}{\overset{\text{(reordering)}}{=}}
=\sum_{k=1}^{\infty}\sum_{i=1}^{n}x_i\mu(A_i\cap C_k) \\
&=\sum_{i=1}^{\infty}\int_{\Omega}^{}X\indic_{C_k}\odif{\mu}=\sum_{i=1}^{\infty}\nu(C_i).
\end{align*}
\end{enumerate}
\item Note that \(\int_{\Omega}^{}\indic_{A}\odif{\mu}
\underset{(\indic_{A}=1\cdot \indic_{A}+0\cdot
\indic_{A^c})}{\overset{\text{(simple integrand)}}{=}}
1\cdot \mu(A)+0\cdot \mu(A^c) =\mu(A)\).
\end{enumerate}
\end{pf}
\subsubsection*{Lebesgue integrals of nonnegative measurable functions}
\item \textbf{Definition.} The key idea to extend the notion of
\(\int_{\Omega}^{}X\odif{\mu}\) to every \emph{nonnegative} measurable function
\(X\) is to approach the nonnegative measurable function \(X\) ``from below'',
through nonnegative \emph{simple functions} (for which the Lebesgue integral
has been defined). We define the Lebesgue integral by taking the supremum over
the Lebesgue integrals of all the nonnegative simple functions that ``lie
below'' the nonnegative measurable function.

Let \(X:\Omega\to\bar{\R}_{+}\) be a
\((\mathcal{F},\mathcal{B}(\bar{\R}_{+}))\)-measurable function. Then the
\defn{Lebesgue integral of \(X\) with respect to \(\mu\)} is
\[
\int_{\Omega}^{}X\odif{\mu}:=
\sup_{\substack{0\le Y\le X,\\ \text{\(Y\) simple}}}
\int_{\Omega}^{}Y\odif{\mu}.
\]
Also, we write
\(\int_{A}^{}X\odif{\mu}:=\int_{\Omega}^{}X\indic_{A}\odif{\mu}\) for every
\(A\in\mathcal{F}\).  The set of all
\((\mathcal{F},\mathcal{B}(\bar{\R}_{+}))\)-measurable functions on \(\Omega\)
(i.e., all functions for which this kind of Lebesgue integral is defined) is
denoted by \(L_{+}\) or \(L_{+}(\Omega,\mathcal{F},\mu)\).

\begin{note}
In case \(X\) is itself simple, say \(X=\sum_{i=1}^{n}x_i\indic_{A_i}\), we
would have \(\int_{\Omega}^{}Y\odif{\mu}\le\sum_{i=1}^{n}x_i\mu(A_i)\) for
every simple function \(Y\) with \(0\le Y\le X\)  by
\labelcref{it:simp-leb-int-prop}. This then implies that
\(\sup_{\substack{0\le Y\le X,\\ \text{\(Y\)
simple}}}\int_{\Omega}^{}Y\odif{\mu} =\max_{\substack{0\le Y\le X,\\
\text{\(Y\) simple}}}\int_{\Omega}^{}Y\odif{\mu} =\sum_{i=1}^{n}x_i\mu(A_i)\),
so the Lebesgue integral here indeed coincides with the one defined for simple
function in this case.
\end{note}
\item \textbf{Some lemmas.} The following lemmas are helpful for proving
properties of Lebesgue integral of this kind.
\begin{lemma}[Approximating sequence]
\label{lma:apx-seq}
A function \(X:\Omega\to\bar{\R}_{+}\) is in \(L_{+}\) iff there exists a
sequence \(\{X_n\}\), with \(X_n:\Omega\to\R_{+}:=[0,\infty)\) being
nonnegative and simple for every \(n\in\N\), such that
\(X_n\nearrow X\) pointwisely, with the convergence being also uniform on
every set where \(X\) is bounded.
\end{lemma}
\begin{remark}
\item Such sequence \(\{X_n\}\) is said to be an \defn{approximating sequence} to
\(X\).
\item The notation \(X_n\nearrow X\) means \(X_n\to X\) with the extra
emphasis that \(\{X_n\}\) is increasing (denoted by \(X_n\nearrow\)), i.e.,
\(X_1\le X_2\le\dotsb\); we shall use similar notations for real-valued
sequences also.
\end{remark}
\begin{pf}
``\(\Leftarrow\)'': Since \(X_n\) is simple and hence measurable for every
\(n\in\N\), one can show in a similar way as in \labelcref{it:seq-rvs-meas}
that \(X\) is measurable as a limit.

``\(\Rightarrow\)'': For every \(n\in\N\), let
\[
X_n:=\min\left\{\frac{\lfloor 2^nX\rfloor}{2^n},n\right\}
=\sum_{k=1}^{n2^{n}}\frac{k-1}{2^{n}}\indic_{X^{-1}\left((\frac{k-1}{2^{n}},\frac{k}{2^{n}}]\right)}
+n\indic_{X^{-1}(n,\infty]}.
\]
\begin{intuition}
The expression \(\frac{\lfloor 2^nX\rfloor}{2^n}\) gives us a value that is
``slightly'' smaller than \(X\), and taking the minimum with \(n\) avoids
\(X_n\) to be ``too large''. Also, the final expression is related to a
certain ``partition'' on the \(y\)-axis: each indicator
\(\indic_{X^{-1}\left((\frac{k-1}{2^{n}},\frac{k}{2^{n}}]\right)}\) corresponds
to a certain region in the \(y\)-axis.
\end{intuition}

Since \(X\) is measurable, the sets involved in the indicators are all in
\(\mathcal{F}\), thus each \(X_n\) is simple. It is also clear that each
\(X_n\) is nonnegative.

First, we show that \(X_n\nearrow\).
For every \(n\in\N\), note that \(\lfloor 2^nX\rfloor/2^n=
2\lfloor 2^nX\rfloor/2^{n+1}\le \lfloor 2^{n+1}X\rfloor/2^{n+1}\). Thus,
\[
X_n=\min\left\{\frac{\lfloor 2^nX\rfloor}{2^n},n\right\}
\overset{\text{(by case)}}{\le}\min\left\{\frac{\lfloor 2^{n+1}X\rfloor}{2^{n+1}},n+1\right\}
=X_{n+1}.
\]

Next, we show the uniform convergence.  On every set where \(X\) is bounded, we
would have \(X\le N\) for some \(N\in\N\). Hence, for all \(n\ge N\), we have
\(X\in (\frac{k-1}{2^{n}},\frac{k}{2^{n}}]\) and \(X_n=(k-1)/2^{n}\) for some
\(k=1,\dotsc,n2^{n}\), which means that \(|X_n-X|\le 1/2^{n}\). This would then
assert the uniform convergence on such set.

Finally, we show the pointwise convergence. This uniform convergence implies
the pointwise convergence for every \(\omega\in\Omega\) with
\(X(\omega)<\infty\) (for such \(\omega\), the singleton \(\{\omega\}\) would
be a set where \(X\) is bounded). So, it remains to show the pointwise
convergence for every \(\omega\in\Omega\) where \(X(\omega)=\infty\). For every
such \(\omega\), we have \(X_n(\omega)=n\) for all \(n\in\N\), hence
\(\{X_n(\omega)\}=\{n\}\to\infty\), establishing the pointwise convergence
(which includes the possibility of ``\(\infty\)'').
\end{pf}

\begin{lemma}[Monotonicity and scaling]
\label{lma:leb-int-nonneg-mono-scal}
Let \(X,Y\in L_{+}\). Then:
\begin{enumerate}
\item \emph{(monotonicity)} \(X\le Y\implies \int_{\Omega}^{}X\odif{\mu}\le\int_{\Omega}^{}Y\odif{\mu}\).
\item \emph{(scaling)} \(\int_{\Omega}^{}cX\odif{\mu}=c\int_{\Omega}^{}X\odif{\mu}\) for all \(c\ge 0\).
\end{enumerate}
\end{lemma}
\begin{pf}
\begin{enumerate}
\item With \(X\le Y\), the integral \(\int_{\Omega}^{}Y\odif{\mu}\) is a
supremum over a larger set than \(\int_{\Omega}^{}X\odif{\mu}\). Hence, the
former must be at least as large as the latter.
\item If \(c=0\), then the result is immediate as we have
\(\int_{\Omega}^{}0\odif{\mu}=0\). So assume henceforth that \(c>0\), and
consider:
\begin{align*}
\int_{\Omega}^{}cX\odif{\mu}
&=\sup_{\substack{0\le \vc{Y}\le cX,\\ \text{\vc{\(Y\)} simple}}}
\int_{\Omega}^{}\vc{Y}\odif{\mu}
=\sup_{\substack{0\le \vc{cZ}\le cX,\\ \text{\vc{\(cZ\)} simple}}}
\int_{\Omega}^{}\vc{cZ}\odif{\mu} \\
&=\sup_{\substack{0\le Z\le X,\\ \text{\(Z\) simple}}}
c\int_{\Omega}^{}Z\odif{\mu}
=c\cdot \sup_{\substack{0\le Z\le X,\\ \text{\(Z\) simple}}}
\int_{\Omega}^{}Z\odif{\mu}
=c\int_{\Omega}^{}X\odif{\mu}.
\end{align*}
\end{enumerate}
\end{pf}
\item \textbf{Monotone convergence theorem.} One important theorem concerning
Lebesgue integrals of nonnegative measurable functions is the \emph{monotone
convergence theorem (MCT)}, which provides conditions under which the integral
\(\int_{\Omega}^{}X\odif{\mu}\) can be computed as a \emph{limit} of
\underline{monotone} sequence instead of taking the supremum. It can also be
viewed as a result that provides a sufficient condition that permits the
\emph{interchange of limit and integral}.

Apart from being an important result on its own, it is also used for proving
many properties of Lebesgue integral of this kind.

\begin{theorem}[Monotone convergence theorem (MCT)]
\label{thm:mct}
Let \(X_n\) be a function on \(\Omega\) in \(L_{+}\) for every \(n\in\N\). If
we have \(X_n\nearrow X\) pointwisely, then \(X\in L_{+}\) and
\(\int_{\Omega}^{}X_n\odif{\mu}\nearrow\int_{\Omega}^{}X\odif{\mu}\).
\end{theorem}
\begin{note}
For the convergence of Lebesgue integrals, we can also write
\(\lim_{n\to\infty}\int_{\Omega}^{}X_n\odif{\mu}=\int_{\Omega}^{}\lim_{n\to\infty}X_n\odif{\mu}\),
from which an interchange of limit and integral can be clearly seen.
\end{note}

\begin{pf}
Since \(X\) is the limit of nonnegative measurable functions \(X_n\)'s, we know
that \(X\ge 0\) and \(X\) is measurable (like
\labelcref{it:seq-rvs-meas}), and hence \(X\in L_{+}\). Also, with \(X_n\nearrow\),
by \Cref{lma:leb-int-nonneg-mono-scal} we have
\(\int_{\Omega}^{}X_n\odif{\mu}\nearrow\), and so the limit
\(\lim_{n\to \infty}\int_{\Omega}^{}X_n\odif{\mu}\) exists (but may possibly be
\(\infty\)).

It then remains to establish that \(\lim_{n\to
\infty}\int_{\Omega}^{}X_n\odif{\mu}=\int_{\Omega}^{}X\odif{\mu}\):
\begin{itemize}
\item ``\(\le\)'': From \(X_n\nearrow X\), we have \(X=\sup_{n\in\N}X_n\ge
X_n\) for every \(n\in\N\). So, by \Cref{lma:leb-int-nonneg-mono-scal} we know
\(\int_{\Omega}^{}X_n\odif{\mu}\le\int_{\Omega}^{}X\odif{\mu}\) for every
\(n\in\N\). Thus, \(\lim_{n\to \infty}\int_{\Omega}^{}X_n\odif{\mu} \le
\int_{\Omega}^{}X\odif{\mu}\).
\item ``\(\ge\)'': Fix any simple function \(Y\) with \(0\le Y\le X\), and any
\(\alpha\in(0,1)\). Let \(A_n:=\{\omega\in\Omega:X_n(\omega)\ge \alpha
Y(\omega)\}\) for every \(n\in\N\). Since each \(X_n-\alpha Y\) is measurable,
each \(A_n\) is in \(\mathcal{F}\). Also, with \(X_n\nearrow\), we have
\(A_n\nearrow\) and
\[
\lim_{n\to\infty}A_n=\bigcup_{k=1}^{\infty}A_k
=\{\omega\in\Omega:X_k(\omega)\ge \alpha Y(\omega)\text{ for some \(k\in\N\)}\}.
\]
This set equals \(\Omega\), since we have \(\lim_{n\to\infty}X_n(\omega)=X(\omega)\ge
Y(\omega)\ge \alpha Y(\omega)\), which implies that \(X_k(\omega)\ge \alpha
Y(\omega)\) must hold for sufficiently large \(k\in\N\). So we can write
\(A_n\nearrow \Omega\).

By \Cref{lma:leb-int-nonneg-mono-scal}, we have
\(\int_{\Omega}^{}X_n\odif{\mu}\ge \int_{\Omega}^{}X_n\indic_{A_n}\odif{\mu}
\ge \alpha\int_{\Omega}^{}Y\indic_{A_n}\odif{\mu}
=\alpha\int_{A_n}^{}Y\odif{\mu}
\) for all \(n\in\N\). Taking limit then gives
\[
\lim_{n\to\infty}\int_{\Omega}^{}X_n\odif{\mu}
\ge \lim_{n\to\infty}\alpha\vc{\int_{A_n}^{}Y\odif{\mu}}
\overset{\labelcref{it:simp-leb-int-prop}}{=}
\alpha\lim_{n\to\infty}\vc{\nu(A_n)}
\overset{\text{(continuity from below)}}{=}
\alpha\nu(\Omega)
\overset{\text{(definition of \(\nu\))}}{=}
\alpha\int_{\Omega}^{}Y\odif{\mu}.
\]
Letting \(\alpha\to 1^{-}\), we have
\(\vc{\lim_{n\to\infty}\int_{\Omega}^{}X_n\odif{\mu}} \ge
\int_{\Omega}^{}Y\odif{\mu}\), so
\vc{\(\lim_{n\to\infty}\int_{\Omega}^{}X_n\odif{\mu}\)} serves as an upper bound
of \(\left\{\int_{\Omega}^{}Y\odif{\mu}:0\le Y\le X,\quad \text{\(Y\) simple}
\right\}\). Thus, by the definition of supremum, we have
\(\vc{\lim_{n\to\infty}\int_{\Omega}^{}X_n\odif{\mu}}\ge
\int_{\Omega}^{}X\odif{\mu}\).
\end{itemize}
\end{pf}
\item\label{it:nonneg-leb-int-prop} \textbf{Properties.} Now we are ready to prove more properties of
Lebesgue integral of nonnegative measurable function. Let \(X,Y\in L_{+}\) and
\(X_n\in L_{+}\) for each \(n\in\N\). Then:
\begin{enumerate}
\item \(\int_{\Omega}^{}X\odif{\mu}=0\) iff \(X\eqae 0\).
\item \emph{(linearity)} 
\(\int_{\Omega}^{}aX+bY\odif{\mu}=a\int_{\Omega}^{}X\odif{\mu}+b\int_{\Omega}^{}Y\odif{\mu}\)
for all \(a,b\ge 0\).
\item \emph{(commutativity of integral and countable
sum)} \(\sum_{n=1}^{\infty}X_n\in L^{+}\) and
\(\int_{\Omega}^{}\sum_{n=1}^{\infty}X_n\odif{\mu}=\sum_{n=1}^{\infty}
\int_{\Omega}^{}X_n\odif{\mu}\).
\item \emph{(``a.e.\ version'' of MCT)} \(X_n\nearrow X\text{ a.e.}\implies
\lim_{n\to\infty}\int_{\Omega}^{}X_n\odif{\mu}=\int_{\Omega}^{}X\odif{\mu}\).
\begin{note}
``\(X_n\nearrow X\text{ a.e.}\)'' means that we have \(X_n(\omega)\nearrow
X(\omega)\) for all \(\omega\in\Omega\setminus N\) where \(N\) is a null set.
\end{note}
\item If \(\int_{\Omega}^{}X\odif{\mu}<\infty\), then \(\mu(X=\infty)=0\),
i.e., \(X\) is finite a.e.
\item The function \(\nu\) given by \(\nu(A):=\int_{A}^{}X\odif{\mu}~\forall
A\in\mathcal{F}\) is a measure on \((\Omega,\mathcal{F})\).
\end{enumerate}
\begin{pf}
\begin{enumerate}
\item We first consider the case where \(X\) is simple. Write
\(X=\sum_{i=1}^{n}x_i\indic_{A_i}\) where \(x_1,\dotsc,x_n\ge 0\). Then, we
have
\begin{align*}
\int_{\Omega}^{}X\odif{\mu}&=\sum_{i=1}^{n}x_i\mu(A_i)=0 \\
\iff x_i&=0\text{ or }\mu(A_i)=0\text{ for all \(i=1,\dotsc,n\)} \\
\iff \mu(X>0)&=0 \\
\iff X&\eqae 0,
\end{align*}
which establishes the result for this case.

Next, consider the general case where \(X\in L_{+}\).

``\(\Leftarrow\)'': Assume \(X\eqae 0\). Fix any simple \(Y\) with \(0\le Y\le
X\). Since \(X\eqae 0\), we have \(Y\eqae 0\). Hence, by the result for simple
function, we have \(\int_{\Omega}^{}Y\odif{\mu}=0\). Therefore,
\[\int_{\Omega}^{}X\odif{\mu} =\sup_{\substack{0\le Y\le X,\\ \text{\(Y\)
simple}}} \underbrace{\int_{\Omega}^{}Y\odif{\mu}}_{0} =0.\]

``\(\Rightarrow\)'': Assume \(\int_{\Omega}^{}X\odif{\mu}=0\). Let
\(A_n:=\{\omega\in\Omega:X(\omega)\ge 1/n\}\) for every \(n\in\N\). Then,  we
have \(X\ge \indic_{A_n}/n\) for all \(n\in\N\) and
\(\mu(\bigcup_{n=1}^{\infty}A_n)=\mu(\{\omega\in\Omega:X(\omega)>0\}
=\mu(X>0)\). So, it suffices to show that \(\mu(\bigcup_{n=1}^{\infty}A_n)=0\).

Assume to the contrary that \(\mu(\bigcup_{n=1}^{\infty}A_n)>0\). Then there
exists \(n_0\in\N\) such that \vc{\(\mu(A_{n_0})>0\)}; otherwise we would have
\(\mu(A_{n})>0\) for all \(n\in\N\), so
\(\mu(\bigcup_{n=1}^{\infty}A_n)\le\sum_{n=1}^{\infty}\mu(A_n)=0\), which is
impossible. Hence, we have
\(\int_{\Omega}^{}X\odif{\mu}\ge\int_{\Omega}^{}\frac{1}{n_0}\indic_{A_{n_0}}\odif{\mu}
\overset{\text{(simple integrand)}}{=}\frac{1}{n_0}\mu(A_{n_0})\vc{>0}\), contradiction.
\item Due to the scaling property, it suffices to show the result with
\(a=b=1\). By \Cref{lma:apx-seq}, there exist sequences \(\{X_{n}\}\) and
\(\{Y_{n}\}\) of nonnegative simple functions such that \(X_n\nearrow X\) and
\(Y_n\nearrow Y\).  From this, we know \(\{X_n+Y_n\}\) is a sequence of
nonnegative simple functions with \(X_n+Y_n\nearrow X+Y\). Thus,
\begin{align*}
\int_{\Omega}^{}X+Y\odif{\mu}
\overset{\text{(MCT)}}&{=}\lim_{n\to\infty}\int_{\Omega}^{}X_n+Y_n\odif{\mu}
\overset{\text{(linearity for simple functions)}}{=}
\lim_{n\to\infty}\int_{\Omega}^{}X_n\odif{\mu}+\lim_{n\to\infty}\int_{\Omega}^{}Y_n\odif{\mu} \\
\overset{\text{(MCT)}}&{=}\int_{\Omega}^{}X\odif{\mu}+\int_{\Omega}^{}Y\odif{\mu}.
\end{align*}
\item Note that we have \(\sum_{n=1}^{\infty}X_n\in L^{+}\) by MCT, and
\[
\int_{\Omega}^{}\sum_{n=1}^{\infty}X_n\odif{\mu}
\overset{\text{(MCT)}}{=}
\lim_{N\to\infty}\int_{\Omega}^{}\underbrace{\sum_{n=1}^{N}X_n}_{\nearrow}\odif{\mu}
\overset{\text{(b)}}{=}
\lim_{N\to\infty}\sum_{n=1}^{N}\int_{\Omega}^{}X_n\odif{\mu}
=\sum_{n=1}^{\infty}\int_{\Omega}^{}X_n\odif{\mu}.
\]
\item By assumption, we have \(X_n(\omega)\nearrow X(\omega)\) for all
\(\omega\in\Omega\setminus N\) where \(N\) is a null set. Since
\(X_n-X_n\indic_{N^c}\eqae 0\) for all \(n\in\N\) and \(X-X\indic_{N^c}\eqae
0\), by (a) we have \(\int_{\Omega}^{}X_n-X_n\indic_{N^c}\odif{\mu}=0\) for all \(n\in\N\)
and \(\int_{\Omega}^{}X-X\indic_{N^c}\odif{\mu}=0\), which implies by (b) that
\(\int_{\Omega}^{}X_n\odif{\mu}=\int_{\Omega}^{}X_n\indic_{N^c}\odif{\mu}\) for
all \(n\in\N\) and \(\int_{\Omega}^{}X\odif{\mu}=\int_{\Omega}^{}X\indic_{N^c}\odif{\mu}\).

Therefore,
\[
\lim_{n\to\infty}\int_{\Omega}^{}X_n\odif{\mu}
=\lim_{n\to\infty}\int_{\Omega}^{}X_n\indic_{N^c}\odif{\mu}
\underset{(X_n\indic_{N^c}\nearrow X\indic_{N^c}\text{ pointwisely})}{\overset{\text{(MCT)}}{=}}
\int_{\Omega}^{}X\indic_{N^c}\odif{\mu}
=\int_{\Omega}^{}X\odif{\mu}.
\]
\item Let \(A_n:=\{X\ge n\}\) for all \(n\in\N\cup\{\infty\}\). Assume to the
contrary that \(\mu(X=\infty)=\mu(A_{\infty})>0\). Then, we have
\[
\int_{\Omega}^{}X\odif{\mu}=\int_{\Omega}^{}X\indic_{A_n}\odif{\mu}
\ge\int_{\Omega}^{}n\indic_{A_n}\odif{\mu}
\overset{\text{(simple integrand)}}{=}
n\mu(A_n)\overset{(A_n\supseteq A_{\infty})}{\ge}
n\mu(A_{\infty})\to\infty\text{ as \(n\to\infty\),}
\]
contradiction.
\item \begin{enumerate}[label={(\arabic*)}]
\item For every \(A\in\mathcal{F}\), we have
\(\nu(A)=\int_{\Omega}^{}\underbrace{X\indic_{A}}_{\ge
0}\odif{\mu}\overset{\text{(monotonicity)}}{\ge} 0\).
\item We have \(\nu(\varnothing)=\int_{\Omega}^{}\underbrace{X\indic_{\varnothing}}_{0}\odif{\mu}
=0\).
\item Fix any pairwise disjoint \(A_1,A_2,\dotsc\in\mathcal{F}\). Then
\(\nu(\biguplus_{i=1}^{\infty}A_i)
=\int_{\Omega}^{}X\indic_{\biguplus_{i=1}^{\infty}A_i}\odif{\mu}
=\int_{\Omega}^{}\sum_{i=1}^{\infty}X\indic_{A_i}\odif{\mu}
\overset{\text{(c)}}{=}\sum_{i=1}^{\infty}\int_{\Omega}^{}X\indic_{A_i}\odif{\mu}
=\sum_{i=1}^{\infty}\nu(A_i)\).
\end{enumerate}
\end{enumerate}
\end{pf}
\subsubsection*{Lebesgue integrals of measurable functions}
\item\label{it:leb-int-gen-def} \textbf{Definition.} The last step in the
construction of Lebesgue integral is to extend the notion of
\(\int_{\Omega}^{}X\odif{\mu}\) to every
\((\mathcal{F},\mathcal{B}(\bar{\R}))\)-measurable function
\(X:\Omega\to\bar{\R}\). The key idea is to write such function \(X\) as the
difference \(X=X^{+}-X^{-}\) where \(X^{+}:=\max\{X,0\}\ge 0\) and
\(X^{-}:=\max\{-X,0\}\ge 0\) are the \defn{positive part} and \defn{negative
part} of \(X\) respectively. Note also that we have \(|X|=X^{+}+X^{-}\), and
that \(X\) is measurable iff \(X^{+}\) and \(X^{-}\) both are (by
\labelcref{it:meas-fn-ran-vec}).


Let \(X:\Omega\to\bar{\R}\) be
\((\mathcal{F},\mathcal{B}(\bar{\R}))\)-measurable. If
\(\int_{\Omega}^{}X^{+}\odif{\mu}<\infty\) \underline{or}
\(\int_{\Omega}^{}X^{-}\odif{\mu}<\infty\), then \(X\) is said to be
\defn{quasi-integrable} and the \defn{Lebesgue integral of \(X\) with respect
to \(\mu\)} is
\[
\int_{\Omega}^{}X\odif{\mu}:=\int_{\Omega}^{}X^{+}\odif{\mu}-\int_{\Omega}^{}X^{-}\odif{\mu}.
\]
Also, we write \(\int_{A}^{}X\odif{\mu}:=\int_{\Omega}^{}X\indic_{A}\odif{\mu}\).
Furthermore, if \(\int_{\Omega}^{}X^{+}\odif{\mu}<\infty\) \underline{and}
\(\int_{\Omega}^{}X^{-}\odif{\mu}<\infty\), then \(X\) is said to be \defn{integrable};
if \(\int_{\Omega}^{}X^{+}\odif{\mu}=\infty\) \underline{and}
\(\int_{\Omega}^{}X^{-}\odif{\mu}=\infty\), then \(X\) is said to be
\defn{non-integrable}. \begin{warning}
Here, ``non-integrable'' is not the same as ``not integrable''! The latter
covers the possibility of quasi-integrable also.
\end{warning}

The set of all integrable and \((\mathcal{F},\mathcal{B}(\vc{\R}))\)-measurable
functions is denoted by \(L^1\) or \(L^1(\Omega,\mathcal{F},\mu)\); note that
here we only functions with codomain \vc{\(\R\)} rather than \(\bar{\R}\)
so that the values taken by such functions are always finite, for mathematical
tractability. (To be compatible with the definition above, we may still view
such functions as having the codomain \(\bar{\R}\) notionally, but they are not
allowed to take the values \(\pm\infty\).)

\begin{remark}
\item \emph{(more on quasi-integrability)} If \(\int_{\Omega}^{}X^{+}\odif{\mu}=\infty\) and
\(\int_{\Omega}^{}X^{+}\odif{\mu}<\infty\), then \(\int_{\Omega}^{}X\odif{\mu}=\infty\).
If \(\int_{\Omega}^{}X^{+}\odif{\mu}<\infty\) and \(\int_{\Omega}^{}X^{+}\odif{\mu}=\infty\), then
\(\int_{\Omega}^{}X\odif{\mu}=-\infty\).
\item \emph{(more on integrability)} Since \(|X|=X^{+}+X^{-}\), we have
\(\int_{\Omega}^{}|X|\odif{\mu}<\infty\iff
(\int_{\Omega}^{}X^{+}\odif{\mu}<\infty\text{ and
}\int_{\Omega}^{}X^{-}\odif{\mu}<\infty)\). Thus, we know \(X\) is integrable iff
\(\int_{\Omega}^{}|X|\odif{\mu}<\infty\), which gives us a helpful criterion
for determining integrability.

\item \emph{(Lebesgue integral of complex-valued function)}
Lebesgue integral of a \emph{complex-valued} function can be constructed by
handling its real and imaginary parts separately. For a complex-valued function
\(X:\Omega\to\C\) that is \((\mathcal{F},\mathcal{B}(\C))\)-measurable (note
that we have \(\mathcal{B}(\C)=\mathcal{B}(\R^2)\)), it is \defn{integrable} if
\(\int_{\Omega}^{}|\re{X}|\odif{\mu}<\infty\) and
\(\int_{\Omega}^{}|\im{X}|\odif{\mu}<\infty\) where \(\re{X}\) and \(\im{X}\)
denote the real and imaginary parts of \(X\) respectively. If it is integrable,
then we define
\(\int_{\Omega}^{}X\odif{\mu}:=\int_{\Omega}^{}\re{X}\odif{\mu}+i\int_{\Omega}^{}\im{X}\odif{\mu}\).

Since \(|X|\overset{\text{(triangle inequality)}}{\le}|\re{X}|+|\im{X}|
\overset{\text{(Pythagorean theorem)}}{\le}|X|+|X|=2|X|\), we have
\(\int_{\Omega}^{}|X|\odif{\mu}<\infty\) iff (\(\int_{\Omega}^{}|\re{X}|\odif{\mu}<\infty\)
and \(\int_{\Omega}^{}|\im{X}|\odif{\mu}<\infty\)), we again have \(X\) is
integrable iff \(\int_{\Omega}^{}|X|\odif{\mu}<\infty\) here (where \(|X|\)
denotes the modulus of \(X\)).
\item \emph{(special notation for Lebesgue measure)} In the case where \(\Omega=\R^d\)
and \(\mu=\lambda\) is the Lebesgue measure, we often just write
\(\int_{\Omega}^{}f(\vect{x})\odif{\vect{x}}\) instead of
\(\int_{\Omega}^{}f(\vect{x})\odif{\lambda(\vect{x})}\).
\end{remark}
\item \label{it:quasi-int-prop}\textbf{Properties of quasi-integrable functions.} Let
\(X,Y:\Omega\to\bar{\R}\) be quasi-integrable (and measurable) functions. Then:
\begin{enumerate}
\item \emph{(additivity)} If we have (\(\int_{\Omega}^{}X^{-}\odif{\mu}<\infty\) and
\(\int_{\Omega}^{}Y^{-}\odif{\mu}<\infty\)) or
(\(\int_{\Omega}^{}X^{+}\odif{\mu}<\infty\) and
\(\int_{\Omega}^{}Y^{+}\odif{\mu}<\infty\)), then \(\int_{\Omega}^{}X+Y\odif{\mu}
=\int_{\Omega}^{}X\odif{\mu}+\int_{\Omega}^{}Y\odif{\mu}\).
\item \emph{(scaling)} \(\int_{\Omega}^{}cX\odif{\mu}=c\int_{\Omega}^{}X\odif{\mu}\).
\item \emph{(nonnegativity)} \(X\ge 0\implies \int_{\Omega}^{}X\odif{\mu}\ge 0\).
\item \emph{(monotonicity)} \(X\le Y\implies \int_{\Omega}^{}X\odif{\mu}\le\int_{\Omega}^{}Y\odif{\mu}\).
\item \emph{(triangle inequality)} \(|\int_{\Omega}^{}X\odif{\mu}|\le\int_{\Omega}^{}|X|\odif{\mu}\).
\item \emph{(integral over null set is always zero)}
\(\int_{A}^{}X\odif{\mu}=0\) for every \(A\in\mathcal{F}\) with \(\mu(A)=0\).
\end{enumerate}
\begin{note}
If \(X\) and \(Y\) are \emph{integrable}, then combining (a) and (b) would
yield the \emph{linearity}:
\(\int_{\Omega}^{}aX+bY\odif{\mu}=a\int_{\Omega}^{}X\odif{\mu}+b\int_{\Omega}^{}Y\odif{\mu}\)
for all \(a,b\in\R\).
\end{note}

\begin{pf}
\begin{enumerate}
\item First note that
\[
\begin{cases}
X+Y=\vc{(X+Y)^{+}}-\orc{(X+Y)^{-}}, \\
X+Y=\orc{X^{+}}-\vc{X^{-}}+\orc{Y^{+}}-\vc{Y^{-}},
\end{cases}
\]
thus we have \(\vc{(X+Y)^{+}+X^{-}+Y^{-}}
=\orc{(X+Y)^{-}+X^{+}+Y^{+}}\). As all the functions here are nonnegative, by
\labelcref{it:nonneg-leb-int-prop} we have
\begin{equation}
\label{eq:leb-int-sum-eq}
\int_{\Omega}^{}\vc{(X+Y)^{+}}\odif{\mu}
+\int_{\Omega}^{}\vc{X^{-}}\odif{\mu}
+\int_{\Omega}^{}\vc{Y^{-}}\odif{\mu}
=\int_{\Omega}^{}\orc{(X+Y)^{-}}\odif{\mu}
+\int_{\Omega}^{}\orc{X^{+}}\odif{\mu}
+\int_{\Omega}^{}\orc{Y^{+}}\odif{\mu}.
\end{equation}
WLOG, assume we are in the case where \(\int_{\Omega}^{}\vc{X^{-}}\odif{\mu}<\infty\) and
\(\int_{\Omega}^{}\vc{Y^{-}}\odif{\mu}<\infty\). Since we have
\(\orc{(X+Y)^{-}}\le\vc{X^{-}}+\vc{Y^{-}}\), by monotonicity we get
\(\int_{\Omega}^{}\orc{(X+Y)^{-}}\odif{\mu}\le\int_{\Omega}^{}\vc{X^{-}}\odif{\mu}
+\int_{\Omega}^{}\vc{Y^{-}}\odif{\mu}<\infty\). Therefore, subtracting
\(\int_{\Omega}^{}\vc{X^{-}}\odif{\mu}+
\int_{\Omega}^{}\vc{Y^{-}}\odif{\mu}+
\int_{\Omega}^{}\orc{(X+Y)^{-}}<\infty\) from both sides of \Cref{eq:leb-int-sum-eq}
yields
\[
\int_{\Omega}^{}\vc{(X+Y)^{+}}\odif{\mu}
-\int_{\Omega}^{}\orc{(X+Y)^{-}}\odif{\mu}
=\int_{\Omega}^{}\orc{X^{+}}\odif{\mu}
-\int_{\Omega}^{}\vc{X^{-}}\odif{\mu}
+\int_{\Omega}^{}\orc{Y^{+}}\odif{\mu}
-\int_{\Omega}^{}\vc{Y^{-}}\odif{\mu},
\]
which implies that \(\int_{\Omega}^{}X+Y\odif{\mu}
=\int_{\Omega}^{}X\odif{\mu}+\int_{\Omega}^{}Y\odif{\mu}\).
\item 
\begin{itemize}
\item \emph{Case 1: \(c\ge 0\).} We have
\begin{align*}
\int_{\Omega}^{}cX\odif{\mu}
&=\int_{\Omega}^{}(cX)^{+}\odif{\mu}-\int_{\Omega}^{}(cX)^{-}\odif{\mu}
\overset{\text{(by case)}}{=}\int_{\Omega}^{}c(X^{+})\odif{\mu}-\int_{\Omega}^{}c(X^{-})\odif{\mu} \\
&=c\int_{\Omega}^{}X^{+}\odif{\mu}-c\int_{\Omega}^{}X^{-}\odif{\mu}
=c\int_{\Omega}^{}X\odif{\mu}.
\end{align*}
\item \emph{Case 2: \(c<0\).} We have
\begin{align*}
\int_{\Omega}^{}cX\odif{\mu}
&=\int_{\Omega}^{}\vc{(cX)^{+}}\odif{\mu}-\int_{\Omega}^{}\orc{(cX)^{-}}\odif{\mu}
\overset{\text{(by case)}}{=}\int_{\Omega}^{}\vc{-c(X^{-})}\odif{\mu}
-\int_{\Omega}^{}\orc{-c(X^{+})}\odif{\mu} \\
&=-c\int_{\Omega}^{}X^{-}\odif{\mu}+c\int_{\Omega}^{}X^{+}\odif{\mu}
=c\int_{\Omega}^{}X\odif{\mu}.
\end{align*}
\end{itemize}
\item With \(X\ge 0\), we know \(X=X^{+}\), so by the monotonicity for Lebesgue
integral of nonnegative measurable function, we have
\(\int_{\Omega}^{}X\odif{\mu}=\int_{\Omega}^{}X^{+}\odif{\mu}\ge 0\).
\item Since \(Y-X\ge 0\), we have
\(\int_{\Omega}^{}(Y-X)^{-}\odif{\mu}=\int_{\Omega}^{}0\odif{\mu}=0<\infty\).
Thus, \(Y-X\) is quasi-integrable. With \(X\) and \(Y-X\) being
quasi-integrable, we can apply (a) and (c) to get
\[
\int_{\Omega}^{}Y\odif{\mu}
\overset{\text{(a)}}{=}\int_{\Omega}^{}Y-X\odif{\mu}+\int_{\Omega}^{}X\odif{\mu}
=\int_{\Omega}^{}Y-X\odif{\mu}+\int_{\Omega}^{}X\odif{\mu}
\overset{\text{(c)}}{\ge}\int_{\Omega}^{}X\odif{\mu}.
\]
\item We have
\begin{align*}
\left|\int_{\Omega}^{}X\odif{\mu}\right|
&=\left|\int_{\Omega}^{}X^{+}\odif{\mu}-\int_{\Omega}^{}X^{-}\odif{\mu}\right|
\overset{\text{(triangle inequality for real numbers)}}{\le}
=\left|\int_{\Omega}^{}X^{+}\odif{\mu}\right|+\left|\int_{\Omega}^{}X^{-}\odif{\mu}\right| \\
\overset{(X^{+},X^{-}\ge 0)}&{=}\int_{\Omega}^{}X^{+}\odif{\mu}+\int_{\Omega}^{}X^{-}\odif{\mu}
=\int_{\Omega}^{}X^{+}+X^{-}\odif{\mu}
=\int_{\Omega}^{}|X|\odif{\mu}.
\end{align*}
\item First consider the special case where \(X\ge 0\), which implies
\(X\indic_{A}\ge 0\). Fix any simple function
\(Y=\sum_{i=1}^{n}y_i\indic_{B_i}\) with \(0\le Y\le X\indic_{A}\), where
\(y_1,\dotsc,y_n\ge 0\). For all \(i=1,\dotsc,n\), we have \(y_i\indic_{B_i}\le
X\indic_{A}\), and consider the following two cases:
\begin{itemize}
\item \emph{Case 1: \(B_i\subseteq A\).} By monotonicity of \(\mu\), we have
\(\mu(B_i)\le\mu(A)=0\), forcing that \(\mu(B_i)=0\).
\item \emph{Case 2: \(B_i\not\subseteq A\).} In this case, there exists
\(\omega\in B_i\) such that \(\omega\notin A\). For this \(\omega\), the
inequality asserts that \(y_i=y_i\indic_{B_i}(\omega)\le
X(\omega)\indic_{A}(\omega)=0\), which forces that \(y_i=0\).
\end{itemize}
Therefore, in either case we have \(y_i\indic_{B_i}=0\). It follows that
\(\int_{\Omega}^{}Y\odif{\mu}=\sum_{i=1}^{n}y_i\mu(B_i)=0\). Hence, we have
\(\int_{A}^{}X\odif{\mu}=\int_{\Omega}^{}X\indic_{A}\odif{\mu}
=\sup_{\substack{0\le Y\le X,\\ \text{\(Y\) simple}}}\int_{\Omega}^{}Y\odif{\mu}
=0\).

Now, consider the general case. Using the result proven for the special case
above, we get \(\int_{\Omega}^{}X\indic_{A}\odif{\mu}=\int_{\Omega}^{}X^{+}\indic_{A}\odif{\mu}
-\int_{\Omega}^{}X^{-}\indic_{A}\odif{\mu}=0-0=0\).
\end{enumerate}
\end{pf}
\item \textbf{More properties of Lebesgue integrals.}
\begin{enumerate}
\item\label{it:leb-int-sig-add} \emph{(\(\sigma\)-additivity of Lebesgue integral)} Let
\(A_1,A_2,\dotsc\in\mathcal{F}\) be pairwise disjoint and let
\(A:=\biguplus_{i=1}^{\infty}A_i\). Let \(Z:\Omega\to\bar{\R}\) be a measurable
function such that \(Z\indic_{A}\in L^{1}\). Then, we have
\[
\int_{A}^{}Z\odif{\mu}=\sum_{i=1}^{\infty}\int_{A_i}^{}Z\odif{\mu}.
\]
\item\label{it:leb-int-ae-fin} \emph{(a.e.\ finiteness of integrable functions)} If \(X\) is integrable,
then \(\mu(X=\pm\infty)=0\), i.e., \(X\) is finite a.e.
\item\label{it:equiv-ae-equal-int-fn} \emph{(equivalent criteria for a.e.\
equality of integrable functions)} Let \(X,Y\) be integrable functions. Then
the following are equivalent:
\begin{enumerate}
\item \(X\eqae Y\).
\item \(\int_{A}^{}X\odif{\mu}=\int_{A}^{}Y\odif{\mu}\) for all \(A\in\mathcal{F}\).
\item \(\int_{\Omega}^{}|X-Y|\odif{\mu}=0\).
\end{enumerate}
\end{enumerate}
\begin{pf}
\begin{enumerate}
\item First, note that \(\int_{\Omega}^{}|Z|\indic_{A}\odif{\mu}= \int_{\Omega}^{}|Z\indic_{A}|\odif{\mu}
\overset{(Z\indic_{A}\in L^{1})}{<}\infty\). Thus, \(\int_{\Omega}^{}Z^{+}\indic_{A}\odif{\mu}
\le\int_{\Omega}^{}|Z|\indic_{A}\odif{\mu}<\infty\) and
\(\int_{\Omega}^{}Z^{-}\indic_{A}\odif{\mu}
\le\int_{\Omega}^{}|Z|\indic_{A}\odif{\mu}<\infty\). So, we can write
\[
\int_{\Omega}^{}Z\indic_{A}\odif{\mu}=\int_{\Omega}^{}(Z^{+}-Z^{-})\indic_{A}\odif{\mu}
=\int_{\Omega}^{}Z^{+}\indic_{A}\odif{\mu}-\int_{\Omega}^{}Z^{-}\indic_{A}\odif{\mu}.
\]
By \labelcref{it:nonneg-leb-int-prop}, the function \(\nu\) given by
\(\nu(B)=\int_{\Omega}^{}Z^{+}\indic_{B}\odif{\mu}~\forall B\in\mathcal{F}\) is
a measure, so by its \(\sigma\)-additivity we have
\(
\int_{\Omega}^{}Z^{+}\indic_{A}\odif{\mu}
=\nu(A)=\nu(\biguplus_{i=1}^{\infty}A_i)
=\sum_{i=1}^{\infty}\nu(A_i)
=\sum_{i=1}^{\infty}\int_{\Omega}^{}Z^{+}\indic_{A_i}\odif{\mu}
\), and similarly,
\(\int_{\Omega}^{}Z^{-}\indic_{A}\odif{\mu}=\sum_{i=1}^{\infty}\int_{\Omega}^{}Z^{-}\indic_{A_i}\odif{\mu}
\). Hence,
\begin{align*}
\int_{A}^{}Z\odif{\mu}
&=\int_{\Omega}^{}Z\indic_{A}\odif{\mu}
\overset{\text{(above)}}{=}
\int_{\Omega}^{}Z^{+}\indic_{A}\odif{\mu}-\int_{\Omega}^{}Z^{-}\indic_{A}\odif{\mu}
=\sum_{i=1}^{\infty}\int_{\Omega}^{}Z^{+}\indic_{A_i}\odif{\mu}
-\sum_{i=1}^{\infty}\int_{\Omega}^{}Z^{-}\indic_{A_i}\odif{\mu} \\
&=\sum_{i=1}^{\infty}\left(\int_{\Omega}^{}Z^{+}\indic_{A_i}\odif{\mu}
-\int_{\Omega}^{}Z^{-}\indic_{A_i}\odif{\mu}\right)
=\sum_{i=1}^{\infty}\int_{\Omega}^{}(Z^{+}-Z^{-})\indic_{A_i}\odif{\mu} \\
&=\sum_{i=1}^{\infty}\int_{\Omega}^{}Z\indic_{A_i}\odif{\mu}
=\sum_{i=1}^{\infty}\int_{A_i}^{}Z\odif{\mu}.
\end{align*}
\item Since \(X\) is integrable, we have
\(\int_{\Omega}^{}X^{+}\odif{\mu}<\infty\) and
\(\int_{\Omega}^{}X^{-}\odif{\mu}<\infty\), thus
\(\mu(X^{+}=\infty)=\mu(X^{-}=\infty)=0\). This implies that
\(\mu(X=\pm\infty)=\mu(\{X^{+}=\infty\}\uplus\{X^{-}=\infty\})
=\mu(X^{+}=\infty)+\mu(X^{-}=\infty)=0+0=0\).
\item 
\begin{itemize}
\item \(\text{(i)}\implies \text{(iii)}\): From \(X\eqae Y\), we know
\(|X-Y|\eqae 0\). Hence, by \labelcref{it:nonneg-leb-int-prop} we have
\(\int_{\Omega}^{}|X-Y|\odif{\mu}=0\).
\item \(\text{(iii)}\implies \text{(ii)}\): Assume
\(\int_{\Omega}^{}|X-Y|\odif{\mu}=0\). Then, we have
\begin{align*}
0\le \left|\int_{A}^{}X\odif{\mu}-\int_{A}^{}Y\odif{\mu}\right|
&=\left|\int_{\Omega}^{}(X-Y)\indic_{A}\odif{\mu}\right|
\overset{\text{(triangle inequality)}}{\le}
\int_{\Omega}^{}|X-Y|\indic_{A}\odif{\mu} \\
&\le\int_{\Omega}^{}|X-Y|\odif{\mu}=0,
\end{align*}
so \(\int_{A}^{}X\odif{\mu}=\int_{A}^{}Y\odif{\mu}\).
\item \(\text{(ii)}\implies \text{(i)}\): Assume
\(\int_{A}^{}X\odif{\mu}=\int_{A}^{}Y\odif{\mu}\) for all \(A\in\mathcal{F}\).
Let \(Z:=X-Y\). Assume to the contrary that \(\mu(Z\ne 0)=\mu(X\ne Y)>0\).
Then, we have \(\mu(Z^{+}>0)>0\) or \(\mu(Z^{-}>0)>0\). WLOG, suppose that
the former is the case, and let \(A:=\{Z^{+}>0\}\).

On \(A\) we have \(Z>0\), so \(Z\indic_{A}=Z^{+}\indic_{A}\). Hence,
\[0=\int_{A}^{}X\odif{\mu}-\int_{A}^{}Y\odif{\mu}
=\int_{\Omega}^{}Z\indic_{A}\odif{\mu}
=\int_{\Omega}^{}Z^{+}\indic_{A}\odif{\mu},
\]
which implies by \labelcref{it:nonneg-leb-int-prop} that \(Z^{+}\indic_{A}\eqae
0\), leading to a contradiction since we have \(\mu(Z^{+}\indic_{A}>0)
=\mu(\{Z^{+}>0\}\cap A)=\mu(A)>0\).
\end{itemize}
\end{enumerate}
\end{pf}
\item \textbf{Dominated convergence theorem.} Apart from \emph{monotone
convergence theorem}, another important result about Lebesgue integral is the
\emph{dominated convergence theorem (DCT)}, which provides another sufficient
condition (involving a certain ``\underline{domination}'') that allows the
interchange of limit and integral. To prove DCT, we would need the
\emph{Fatou's lemma}, which suggests that interchanging limit inferior and
integral would lead to an inequality in general, for nonnegative measurable
functions.

\begin{lemma}[Fatou's lemma]
\label{lma:fatou}
Let \(X_n\in L_{+}\) for every \(n\in\N\). Then
\[
\int_{\Omega}^{}\liminf_{n\to \infty}X_n\odif{\mu}
\le\liminf_{n\to \infty}\int_{\Omega}^{}X_n\odif{\mu}.
\]
\begin{note}
The limit inferior \(\liminf_{n\to \infty}X_n\) can be defined as
\(\lim_{n\to\infty}\inf_{k\ge n}X_k\) (pointwise).
\end{note}
\end{lemma}
\begin{pf}
First, note that \(\inf_{k\ge n}X_k\) is in \(L_{+}\) for all \(n\in\N\).
Also, notice that the sequence \(\{\inf_{k\ge n}X_k\}_{n\in\N}\) is increasing.
Hence, by monotone convergence theorem, we have
\begin{align*}
\int_{\Omega}^{}\liminf_{n\to \infty}X_n\odif{\mu}
&=\int_{\Omega}^{}\lim_{n\to\infty}\inf_{k\ge n}X_k\odif{\mu}
\overset{\text{(MCT)}}{=}
\lim_{n\to\infty}\int_{\Omega}^{}\inf_{k\ge n}X_k\odif{\mu} \\
&=\liminf_{n\to\infty}\int_{\Omega}^{}\vc{\inf_{k\ge n}X_k}\odif{\mu}
\underset{\text{(monotonicity)}}{\overset{(\vc{\inf_{k\ge n}X_k}\le \orc{X_n})}{\le}}
\liminf_{n\to \infty}\int_{\Omega}^{}\orc{X_n}\odif{\mu}.
\end{align*}
\end{pf}
\begin{theorem}[Dominated convergence theorem (DCT)]
\label{thm:dct}
Let \(X_n\in L^{1}\) for every \(n\in\N\), and \(X:\Omega\to\R\) be measurable.
If \(X_n\to X\) a.e.\ and \(|X_n|\le Y\) a.e.\ for all \(n\in\N\), for some
\(Y\in L^{1}\) \emph{(domination)}, then \(X\in L^{1}\) and
\(\lim_{n\to\infty}\int_{\Omega}^{}X_n\odif{\mu}=\int_{\Omega}^{}X\odif{\mu}\).
\end{theorem}
\begin{pf}
By assumption, we have \(\lim_{n\to\infty}X_n(\omega)=X(\omega)\) for all
\(\omega\in\Omega\setminus N_X\), and for each \(n\in\N\), \(|X_n(\omega)|\le
Y(\omega)\) for all \(\omega\in\Omega\setminus N_n\), where \(N_X\) and
\(N_n\)'s are null sets. Now let \(N:=N_X\cup \bigcup_{n=1}^{\infty}N_n\),
which is also a null set.

By construction, we have \(|X_n|\indic_{N^c}\le Y\indic_{N^c}\) for every
\(n\in\N\), and thus letting \(n\to\infty\) gives \(|X|\indic_{N^c}\le
Y\indic_{N^c}\). Hence,
\begin{align*}
\int_{\Omega}^{}|X|\odif{\mu}
&=\underbrace{\int_{\Omega}^{}|X|\indic_{N}\odif{\mu}}_{0\text{ by \labelcref{it:quasi-int-prop}}}+\int_{\Omega}^{}|X|\indic_{N^c}\odif{\mu} \\
\overset{\text{(monotonicity)}}&{\le}
\int_{\Omega}^{}Y\indic_{N^c}\odif{\mu}
\underset{(Y\indic_{N^c}\eqae Y)}{\overset{\text{\labelcref{it:equiv-ae-equal-int-fn}}}{=}}
\int_{\Omega}^{}Y\odif{\mu}
\overset{\text{(monotonicity)}}{\le}
\int_{\Omega}^{}|Y|\odif{\mu}
<\infty,
\end{align*}
which implies that \(X\in L^1\). It then remains to show that
\(\lim_{n\to\infty}\int_{\Omega}^{}X_n\odif{\mu}=\int_{\Omega}^{}X\odif{\mu}\).

For every \(n\in\N\), since \(|X_n|\indic_{N^c}\le Y\indic_{N^c}\), we have
\((Y+X_n)\indic_{N^c}\ge 0\) and \((Y-X_n)\indic_{N^c}\ge 0\), which means that
\((Y+X_n)\indic_{N^c},(Y-X_n)\indic_{N^c}\in L_{+}\). Therefore, we get
\begin{align*}
\int_{\Omega}^{}Y\odif{\mu}+\int_{\Omega}^{}X\odif{\mu}
\overset{\text{\labelcref{it:equiv-ae-equal-int-fn}}}&{=}
\int_{\Omega}^{}(Y+X)\indic_{N^c}\odif{\mu}
=\int_{\Omega}^{}\liminf_{n\to \infty}(Y+X_n)\indic_{N^c}\odif{\mu} \\
\overset{\text{(Fatou's lemma)}}&{\le}
\liminf_{n\to \infty}\int_{\Omega}^{}(Y+X_n)\indic_{N^c}\odif{\mu}
\overset{\text{\labelcref{it:equiv-ae-equal-int-fn}}}{=}
\int_{\Omega}^{}Y\odif{\mu}+\liminf_{n\to \infty}\int_{\Omega}^{}X_n\odif{\mu},
\end{align*}
which implies that \(\int_{\Omega}^{}X\odif{\mu}\vc{\le}\liminf_{n\to
\infty}\int_{\Omega}^{}X_n\odif{\mu}\).

Similarly, we have
\begin{align*}
\int_{\Omega}^{}Y\odif{\mu}-\int_{\Omega}^{}X\odif{\mu}
\overset{\text{\labelcref{it:equiv-ae-equal-int-fn}}}&{=}
\int_{\Omega}^{}(Y-X)\indic_{N^c}\odif{\mu}
=\int_{\Omega}^{}\liminf_{n\to \infty}(Y-X_n)\indic_{N^c}\odif{\mu} \\
\overset{\text{(Fatou's lemma)}}&{\le}
\liminf_{n\to \infty}\int_{\Omega}^{}(Y-X_n)\indic_{N^c}\odif{\mu}
\overset{\text{\labelcref{it:equiv-ae-equal-int-fn}}}{=}
\int_{\Omega}^{}Y\odif{\mu}+\liminf_{n\to \infty}\left(-\int_{\Omega}^{}X_n\odif{\mu}\right) \\
\overset{(\liminf_{n\to \infty}-f_n=-\limsup_{n\to \infty}f_n)}&{=}
\int_{\Omega}^{}Y\odif{\mu}-\limsup_{n\to \infty}\int_{\Omega}^{}X_n\odif{\mu},
\end{align*}
which implies that \(\int_{\Omega}^{}X\odif{\mu}\orc{\ge}\limsup_{n\to
\infty}\int_{\Omega}^{}X_n\odif{\mu}\).

So, altogether we have
\[\limsup_{n\to \infty}\int_{\Omega}^{}X_n\odif{\mu}\ge
\liminf_{n\to \infty}\int_{\Omega}^{}X_n\odif{\mu}
\vc{\ge} \int_{\Omega}^{}X\odif{\mu}\orc{\ge} \limsup_{n\to \infty}\int_{\Omega}^{}X_n\odif{\mu}
\ge\liminf_{n\to \infty}\int_{\Omega}^{}X_n\odif{\mu},\]
which implies that \(\int_{\Omega}^{}X\odif{\mu}=\liminf_{n\to \infty}\int_{\Omega}^{}X_n\odif{\mu}
=\limsup_{n\to \infty}\int_{\Omega}^{}X_n\odif{\mu}=\lim_{n\to\infty}\int_{\Omega}^{}X_n\odif{\mu}\).
\end{pf}

\begin{remark}
\item \emph{(necessity of domination)} To illustrate the necessity of
domination, consider the following example where domination fails and the DCT
does not apply. Let \((\Omega,\mathcal{F},\mu)=([0,1],\mathcal{B}([0,1]),\lambda)\).
Let \(X_n:=n^{2}\indic_{(0,1/n)}\ge 0\) for every \(n\in\N\) and \(X:=0\). Note
that \(X_n\to X\) pointwise (so a.e.), but \(\lim_{n\to\infty}\int_{\Omega}^{}X_n\odif{\mu}
=\lim_{n\to\infty}n^{2}\mu((0,1/n))=\lim_{n\to\infty}n=\infty\), which is
definitely not equal to \(\int_{\Omega}^{}X\odif{\mu}=0\).
\item \emph{(a simple corollary)} Under the conditions in DCT, we can also
deduce that \(\lim_{n\to\infty}\int_{\Omega}^{}|X-X_n|\odif{\mu}=0\), which can
be shown by noting that \(|X-X_n|\to 0\) a.e., and \(|X-X_n|\le
|X|+|X_n|\underset{\text{(a.e.)}}{\le}|X|+|Y|\in L^{1}\).
\end{remark}
We have previously shown the commutativity of integral and countable sum for
Lebesgue integral of nonnegative measurable function. Using DCT, we can extend
this result to Lebesgue integral of general measurable function (but with some
extra condition).
\begin{corollary}[Commutativity of integral and countable sum]
\label{cor:leb-int-sum-commu}
Let \(X_n\in L^{1}\) for each \(n\in\N\). If
\(\sum_{n=1}^{\infty}\int_{\Omega}^{}|X_n|\odif{\mu}<\infty\), then
\(\sum_{n=1}^{\infty}X_n\in L^{1}\) and
\(\int_{\Omega}^{}\sum_{n=1}^{\infty}X_n\odif{\mu}=\sum_{n=1}^{\infty}\int_{\Omega}^{}X_n\odif{\mu}\).
\end{corollary}
\begin{pf}
We first verify the conditions for applying DCT: (i) \(\sum_{k=1}^{n}X_k\in L^{1}\) for each
\(n\in\N\), (ii) \(\sum_{k=1}^{n}X_k\to \sum_{k=1}^{\infty}X_k\) a.e., and
(iii) \(|\sum_{k=1}^{n}X_k|\le Y\) a.e.\ for every \(n\in\N\), for some \(Y\in L^{1}\):
\begin{enumerate}[label={(\roman*)}]
\item For each \(n\in\N\), note that \(\sum_{k=1}^{n}X_k\) is measurable and
\(\int_{\Omega}^{}|\sum_{k=1}^{n}X_k|\odif{\mu}
\le\int_{\Omega}^{}\sum_{k=1}^{n}|X_k|\odif{\mu}
=\sum_{k=1}^{n}\int_{\Omega}^{}|X_k|\odif{\mu}<\infty
\). Hence \(\sum_{k=1}^{n}X_k\in L^{1}\).
\item It suffices to show that \(\sum_{k=1}^{\infty}X_k\) converges a.e., which
would then imply by definition that \(\sum_{k=1}^{n}X_k\to \sum_{k=1}^{\infty}X_k\) a.e.

Note that \(\int_{\Omega}^{}\sum_{k=1}^{\infty}|X_k|\odif{\mu}\overset{(|X_k|\in L_{+}~\forall k\in\N)}{=}
\sum_{k=1}^{\infty}\int_{\Omega}^{}|X_k|\odif{\mu}\overset{\text{(assumption)}}{<}\infty\),
and \(\sum_{k=1}^{\infty}|X_k|\) is nonnegative.
So, we have \(Y:=\sum_{k=1}^{\infty}|X_k|\in L^{1}\). By
\labelcref{it:leb-int-ae-fin}, we have
\(Y(\omega)=\sum_{k=1}^{\infty}|X_k(\omega)|<\infty\) for all \(\omega\in
N^c\) where \(N\) is a null set. This means that \(\sum_{k=1}^{\infty}X_k\)
converges absolutely (and hence converges) a.e.
\item We consider the \(Y\in L^{1}\) above. The domination follows by noting that
for all \(n\in\N\), we have
\(|\sum_{k=1}^{n}X_k|\le \sum_{k=1}^{n}|X_k|\le Y\).
\end{enumerate}
Now, we apply the DCT to get \(\sum_{k=1}^{\infty}X_k\in L^{1}\) and
\[
\int_{\Omega}^{}\sum_{k=1}^{\infty}X_k\odif{\mu}
=\lim_{n\to\infty}\int_{\Omega}^{}\sum_{k=1}^{n}X_k\odif{\mu}
\overset{\text{(additivity)}}{=}\lim_{n\to\infty}\sum_{k=1}^{n}\int_{\Omega}^{}X_k\odif{\mu}
=\sum_{k=1}^{\infty}\int_{\Omega}^{}X_k\odif{\mu}.
\]
\end{pf}
\item \textbf{\(L^{p}\) spaces.} In the definition of Lebesgue integral of
measurable function, we have seen the notation \(L^{1}\), which denotes the set
of all integrable functions. Knowing that \(X\) is integrable iff
\(\int_{\Omega}^{}|X|\odif{\mu}<\infty\), we can express it as
\(L^{1}(\Omega,\mathcal{F},\mu)=\{\vc{X:\Omega\to\R}:\text{\(X\) is
\((\mathcal{F},\mathcal{B}(\R))\)-measurable and \(\int_{\Omega}^{}|X|\odif{\mu}<\infty\)}\}\).

More generally, we can define \emph{\(L^p\) space} in the following. For all
\(p\in (0,\infty)\), let
\(\|X\|_{p}:=(\int_{\Omega}^{}|X|^{p}\odif{\mu})^{1/p}\) be the \defn{\(L^p\)
norm}, and then the \defn{\(L^{p}\) space} is given by
\[
L^p=L^p(\Omega,\mathcal{F},\mu):=\{\vc{X:\Omega\to\R}:\text{\(X\) is
\((\mathcal{F},\mathcal{B}(\R))\)-measurable and \(\|X\|_{p}<\infty\)}\}.
\]
\begin{remark}
\item \emph{(relationship between \(L^p\) and \(L^1\) spaces)} For a
\((\mathcal{F},\mathcal{B}(\R))\)-measurable function \(X\), we have \(X\in L^{p}\) 
iff \(\left(\int_{\Omega}^{}|X|^{p}\odif{\mu}\right)^{1/p}<\infty\)
iff \(\int_{\Omega}^{}|\orc{|X|^{p}}|\odif{\mu}<\infty\)
iff \(\orc{|X|^{p}}\in L^{1}\).
\item \emph{(Banach and Hilbert spaces)} For every \(p\ge 1\), \(L^p\) can be
shown to be a \emph{Banach space}. Also, \(L^2\) can be shown to be a
\emph{Hilbert space}.  These spaces are important in the subject of
\emph{functional analysis} (but we shall not discuss them in details here).
\item \emph{(\(L^{\infty}\) space)} For \(p=\infty\), we define the
\defn{\(L^{\infty}\) norm} by \(\|X\|_{\infty}:=\esssup |X|\), where
\(\esssup |X|\) is the \defn{essential supremum} of \(|X|\) which is given by
\(\esssup |X|:=\inf\{x\ge 0:\mu(|X|>x)=0\}\). For every \(x\ge 0\) with
\(\mu(|X|>x)=0\), it is an ``essential'' upper bound for \(|X|\) in the sense
that we have \(|X|\le x\) a.e. The essential supremum is then the least
``essential'' upper bound for \(|X|\); it also has the property that
\(|X|\le\esssup{|X|}\) a.e.

The \defn{\(L^{\infty}\) space} is then defined by
\[
L^{\infty}=L^{\infty}(\Omega,\mathcal{F},\mu):=\{\vc{X:\Omega\to\R}:\text{\(X\) is
\((\mathcal{F},\mathcal{B}(\R))\)-measurable and \(\|X\|_{\infty}<\infty\)}\}.
\]
It can be shown that \(X\in L^{\infty}\) iff \(X\) is bounded a.e.
\end{remark}
\item \textbf{Properties for \(L^{p}\) spaces.}
\begin{enumerate}
\item\label{it:holder-cs-ineq} \emph{(H\"older inequality)} Let \(p,q\in
[1,\infty]\) with \(1/p+1/q=1\) (with the convention that \(1/\infty:=0\)) be
\emph{conjugate indices}. Then, \(\|XY\|_{1}\le \|X\|_{p}\|Y\|_{q}\) for all
measurable functions \(X,Y:\Omega\to\R\).

Furthermore, if we have \(p,q\in (1,\infty)\), \(X\in L^{p}\), and \(Y\in
L^{q}\), then the equality holds iff \(\alpha|X|^{p}\eqae \beta|Y|^{q}\) for
some \(\alpha,\beta\ge 0\).

\begin{note}
The special case with \(p=q=2\) is known as the \emph{Cauchy-Schwarz
inequality}.
\end{note}
\item\label{it:minkowski-ineq} \emph{(Minkowski's inequality)} Let \(p\in [1,\infty]\). Then
\(\|X+Y\|_{p}\le \|X\|_{p}+\|Y\|_{p}\) for all \(X,Y\in L^{p}\).
\item\label{it:jensen-ineq} \emph{(Jensen's inequality)} Let
\((\Omega,\mathcal{F},\mu)\) be a \vc{probability} space, \(X:\Omega\to\R\) be
a function in \(L^{1}\), and \(\varphi\) be a convex function
on \(\R\). Then, \(\varphi(\int_{\Omega}^{}X\odif{\mu})\le
\int_{\Omega}^{}\varphi(X)\odif{\mu}\).
\begin{note}
In case \(\varphi\) is concave, we can apply the Jensen's inequality to
\(-\varphi\) which is convex. This would then yield the inequality
\(\varphi(\int_{\Omega}^{}X\odif{\mu})\ge
\int_{\Omega}^{}\varphi(X)\odif{\mu}\).
\end{note}
\item\label{it:lp-sp-norm-relate} \emph{(relationship between \(L^{p}\) spaces
and norms)} If \(\mu(\Omega)<\infty\) (i.e., \(\mu\) is finite) and
\(0<p<q\le\infty\), then \(\|X\|_{p}\le
\|X\|_{q}\cdot \mu(\Omega)^{1/p-1/q}\) (and hence \(L^{q}\subseteq L^{p}\)).

\begin{remark}
\item If we have \(\mu(\Omega)=1\) (e.g., in a probability space), then this inequality
gets simplified to \(\|X\|_{p}\le\|X\|_{q}\).
\item With this inequality, we know that \(\|X\|_{q}<\infty\implies
\|X\|_{p}<\infty\) (as \(\mu(\Omega)^{1/p-1/q}\) must be finite), so
\(L^q\subseteq L^{p}\).
\end{remark}
\end{enumerate}
\begin{pf}
\begin{enumerate}
\item Omitted.
\item Omitted.
\item First, it can be shown that \(\int_{\Omega}^{}\varphi(X)\odif{\mu}\in
(-\infty,\infty]\) (see, e.g., \textcite[Theorem~7.9]{klenke2020probability}). In case it is \(\infty\), there is nothing to prove. So, we can assume henceforth that it is in \((-\infty,\infty)\), i.e., \(\varphi(X)\in L^{1}\).

By the convexity of \(\varphi\), for all \(x_0\in\R\), there
exists a \emph{supporting line} \(x\mapsto \varphi(x_0)+m(x-x_0)\) of the graph
of \(\varphi\), satisfying that \(\varphi(x)\ge\varphi(x_0)+m(x-x_0)\) for all
\(x\in\R\).

\begin{center}
\begin{tikzpicture}
\begin{axis}[domain=1:4, ymin=0, xtick={2.5}, ytick={1.25}, xticklabel={\(x_0\)}, yticklabel={\(\varphi(x_0)\)}]
\addplot[blue]{(x-2)^2+1};
\node[blue] () at (2,1.5) {\(\varphi\)};
\addplot[ForestGreen]{x-1.25};
\draw[violet, fill] (2.5,1.25) circle [radius=0.7mm];
\node[ForestGreen] () at (2.7,0.5) {\(\varphi(x_0)+m(x-x_0)\)};
\end{axis}
\end{tikzpicture}
\end{center}

Let \(\expv{X}:=\int_{\Omega}^{}X\odif{\mu}\) (this is indeed how we define
\emph{expectation}; see \Cref{subsect:expectation}).  By setting
\(x_0=\expv{X}\), we have \(\varphi(x)\ge \varphi(\expv{X})+m(x-\expv{X})\) for
all \(x\in\R\). Thus, we have \(\varphi(X)\ge
\varphi(\expv{X})+m(X-\expv{X})\). By monotonicity, we then get
\[
\int_{\Omega}^{}\varphi(X)\odif{\mu}\ge \int_{\Omega}^{}\varphi(\expv{X})
+m(X-\expv{X})\odif{\mu}
=\varphi(\expv{X})\cdot \underbrace{\mu(\Omega)}_{1}
+m\underbrace{\left(\int_{\Omega}^{}X\odif{\mu}-\expv{X}\right)}_{0}
=\varphi\left(\int_{\Omega}^{}X\odif{\mu}\right).
\]
\item For the case where \(q=\infty\), the inequality
translates into \(\|X\|_{p}\le\|X\|_{\infty}\cdot \mu(\Omega)^{1/p}\), which can
be proven as follows:
\[
\|X\|_{p}=\left(\int_{\Omega}^{}|X|^{p}\odif{\mu}\right)^{1/p}
\underset{\text{\labelcref{it:equiv-ae-equal-int-fn}}}{\overset{\text{(monotonicity)}}{\le}}
\left(\int_{\Omega}^{}\|X\|_{\infty}^{p}\odif{\mu}\right)^{1/p}
=\|X\|_{\infty}\cdot \mu(\Omega)^{1/p}.
\]
Next, for the case where \(q<\infty\), we apply H\"older inequality with
conjugate indices \(q/p\) and \(q/(q-p)\) to get
\begin{align*}
\|X\|_{p}&=\left(\int_{\Omega}^{}\vc{|X|^{p}}\cdot \orc{1}\odif{\mu}\right)^{1/p}
=\|\vc{|X|^{p}}\cdot \orc{1}\|_{1}^{1/p}
\overset{\text{(H\"older)}}{\le}
\left(\|\vc{|X|^{p}}\|_{q/p}\|\orc{1}\|_{q/(q-p)}\right)^{1/p} \\
&=\left(\int_{\Omega}^{}|X|^{q}\odif{\mu}\right)^{p/q\cdot 1/p}
\left(\int_{\Omega}^{}1\odif{\mu}\right)^{(q-p)/q\cdot 1/p}
=\|X\|_{q}\cdot \mu(\Omega)^{1/p-1/q}.
\end{align*}
\end{enumerate}
\end{pf}
\end{enumerate}
\subsection{Expectation}
\label{subsect:expectation}
\begin{enumerate}
\item After studying Lebesgue integrals in
\Cref{subsect:construct-lebesgue-int}, we now apply the concept in defining and
computing expectations. Let \(X:\Omega\to\bar{\R}\) be
\((\mathcal{F},\mathcal{B}(\bar{\R}))\)-measurable. Then, the
\defn{expectation} or \defn{mean} of \(X\) is given by \(\expv{X}:=\int_{\Omega}^{}X\odif{\mu}\),
provided that the Lebesgue integral is defined (being finite or \(\pm\infty\)).
\begin{note}
If we want to emphasize the underlying measure \(\mu\), we may write
\(\expvmu{\mu}{X}\) instead.
\end{note}

\item \textbf{Change of variables formula.} Although many general properties of
expectations are derived based on this abstract form of expectation, in
practice we rarely use such abstract formula for computing expectations. Rather
than, we often utilize formulas involving the \emph{mass function} or
\emph{density function}, which can be seen as special cases of the
following \emph{change of variables} formula.

\begin{theorem}[Change of variables]
\label{thm:change-of-var}
Let \((\Omega,\mathcal{F},\mu)\) be a measure space, \((\Omega',\mathcal{F}')\)
be a measurable space, \(X:\Omega\to\Omega'\) be measurable, and
\(h:\Omega'\to\R\) be measurable. If \(\expv{|h(X)|}<\infty\) (i.e., \(h(X)\)
is integrable), then we have
\[
\int_{\Omega}^{}h(X)\odif{\mu}=\int_{\Omega'}^{}h\odif{\mu_{X}}
\]
where \(\mu_X\) denotes the push-forward measure of \(\mu\) with respect to \(X\).
\end{theorem}
\begin{pf}
We apply the standard argument to \(h\) here.
\begin{enumerate}[label={(\arabic*)}]
\item Fix any indicator function \(h(\omega')=\indic_{A'}(\omega')\),
where \(A'\in\mathcal{F}\). Since \(\indic_{A'}(X(\omega))=1\iff X(\omega)\in
A'\iff \omega\in X^{-1}(A')\iff \indic_{X^{-1}(A')}(\omega)=1\), we have
\vc{\(\indic_{A'}(X)=\indic_{X^{-1}(A')}\)}. Hence,
\[
\int_{\Omega}^{}h(X)\odif{\mu}
=\int_{\Omega}^{}\vc{\indic_{A'}(X)}\odif{\mu}
\vc{=}\int_{\Omega}^{}\vc{\indic_{X^{-1}(A')}}\odif{\mu}
=\mu(X^{-1}(A'))
=\mu_X(A')
=\int_{\Omega'}^{}\indic_{A'}\odif{\mu_X}
=\int_{\Omega'}^{}h\odif{\mu_X}.
\]
Then, by linearity, the equality also holds for every simple function \(h\).

\item Fix any \(h\in L_{+}\). By \Cref{lma:apx-seq}, there exists a sequence
\(\{h_n\}\) of nonnegative simple functions on \(\Omega'\) such that
\(h_n\nearrow h\) (pointwisely), which implies \(h_n(X)\nearrow h(X)\). Thus,
\[
\int_{\Omega}^{}h(X)\odif{\mu}
\overset{\text{(MCT)}}{=}\lim_{n\to\infty}\int_{\Omega}^{}h_n(X)\odif{\mu}
\overset{(1)}{=}\lim_{n\to\infty}\int_{\Omega'}^{}h_n\odif{\mu_X}
\overset{\text{(MCT)}}{=}
\int_{\Omega}^{}h\odif{\mu_X}.
\]
\item Fix any \(h\in L^{1}\), and we have
\[
\int_{\Omega}^{}h(X)\odif{\mu}=
\int_{\Omega}^{}h^{+}(X)\odif{\mu}
-\int_{\Omega}^{}h^{-}(X)\odif{\mu}
\overset{(2)}{=}
\int_{\Omega'}^{}h^{+}\odif{\mu_X}
-\int_{\Omega'}^{}h^{-}\odif{\mu_X}
=\int_{\Omega'}^{}h\odif{\mu_X}.
\]
\end{enumerate}
\end{pf}
\item \textbf{Implications of change of variables formula.}
\begin{enumerate}
\item \emph{(expectation as a Lebesgue-Stieltjes integral)}
\label{it:lebesgue-stieljes-int} When applying the change of variables formula,
often we are in the case where \(\mu=\pr\) is a \emph{probability measure} and
\((\Omega',\mathcal{F}')=(\R^{d},\mathcal{B}(\R^d))\) (we then change the
notation \(X\to\vect{X}\), as it is a random vector). Based on
\Cref{thm:dist-fn-char-dist}, sometimes we write \(\odif{F}\) in place of
\(\odif{\pr_{\vect{X}}}\), so the change of variables formula in this case
would become
\[
\expv{h(\vect{X})}=\int_{\R^d}^{}h\odif{F}\overset{\text{(notation)}}{=}\int_{\R^d}^{}h(\vect{x})\odif{F(\vect{x})},
\]
which takes a form that is more familiar to us. The integral
\(\int_{\R^d}^{}h(\vect{x})\odif{F(\vect{x})}\) is known as the
\defn{Lebesgue-Stieltjes integral} of \(h\) with respect to \(F\).
In the special case with \(F(\vect{x})=\prod_{j=1}^{d}x_j\) for all
\(\vect{x}\in[\vect{a},\vect{b}]\), it becomes the Lebesgue integral of \(h\)
with respect to the Lebesgue measure \(\lambda\).

From this expression, it is also clear that if we have \(\vect{X}_1\eqd
\vect{X}_2\) (so both share the same distribution function \(F\)), then
\(\expv{h(\vect{X}_1)}=\expv{h(\vect{X}_2)}\) for every measurable function
\(h:\R^d\to\R\).
\item \emph{(interpretations of Lebesgue-Stieltjes integral)}
Based on the standard argument, one can derive the following expressions of
\(\expv{h(\vect{X})}\):
\begin{itemize}
\item \emph{(in terms of mass function)} If \(F\) is discrete with mass
function \(f\) and countable support
\(\{\vect{x}_1,\vect{x}_2,\dotsc\}\subseteq \R^d\), then
\[
\expv{h(\vect{X})}=\sum_{i=1}^{\infty}h(\vect{x}_i)f(\vect{x}_i).
\]
\item \emph{(in terms of density function)} If \(F\) is absolutely continuous
with density function \(f\), then
\[
\expv{h(\vect{X})}=\int_{\R^d}^{}h(\vect{x})f(\vect{x})\odif{\vect{x}}.
\]
\end{itemize}
These suggest the interpretations of the Lebesgue-Stieltjes integral
\(\int_{\R^d}^{}h(\vect{x})\odif{F(\vect{x})}\) in the cases above.
\end{enumerate}
\item \textbf{Relationship between Riemann-Stieltjes and Lebesgue-Stieltjes integrals.}
The notation \(\int_{[\vect{a},\vect{b}]}^{}h(\vect{x})\odif{F(\vect{x})}\) is
also used for the \emph{Riemann-Stieltjes integral} with integrand
\(h:[\vect{a},\vect{b}]\to\R\) and integrator \(F:[\vect{a},\vect{b}]\to\R\).
This is because such Riemann-Stieltjes integral indeed always coincides with the
corresponding Lebesgue-Stieltjes integral, provided that the former exists.

The \defn{Riemann-Stieltjes integral}
\(\int_{[\vect{a},\vect{b}]}^{}h(\vect{x})\odif{F(\vect{x})}\) is defined as
the limit of the \emph{Riemann-Stieltjes sums}:
\[
\int_{[\vect{a},\vect{b}]}^{}h(\vect{x})\odif{F(\vect{x})}
:=\lim_{n_1,,\dotsc,n_d\to\infty}\sum_{i_d=1}^{n_d}\dotsb\sum_{i_1=1}^{n_1}h(\xi_{1i_1},\dotsc,\xi_{di_{d}})
\Delta_{(\vect{x}_{\vect{i}-{}_{d}1},\vect{x}_{\vect{i}}]}F
\]
where \((\vect{x}_{\vect{i}-{}_{d}1},\vect{x}_{\vect{i}}]\) denotes \(\left(
\begin{bmatrix}x_{1,i_1-1}\\\vdots\\ x_{d,i_d-1}\end{bmatrix},
\begin{bmatrix}x_{1i_1}\\\vdots\\ x_{di_d}\end{bmatrix}
\right]\), with \(a_j=x_{j0}<x_{j1}<\dotsb<x_{jn}=b_j\) and
\(\xi_{ji_j}\in[x_{j,i_j-1},x_{j,i_j}]\) for all \(j=1,\dotsc,d\).
Here, the measure associated to the \(F\)-volume is the Lebesgue-Stieltjes
measure \(\lambda_{F}\).

\begin{note}
In the special case with \(F(\vect{x})=\prod_{j=1}^{d}x_j\) for all
\(\vect{x}\in[\vect{a},\vect{b}]\), it reduces to the \defn{Riemann integral}
\(\int_{[\vect{a},\vect{b}]}^{}h(\vect{x})\odif{\vect{x}}\). Here, the measure
associated to the \(F\)-volume is the Lebesgue measure \(\lambda\).
\end{note}

The following relates the Riemann-Stieltjes and Lebesgue-Stieltjes integrals:
\begin{proposition}
\label{prp:rs-ls-int-relate}
Let \(h:[\vect{a},\vect{b}]\to\R\) be a function.
\begin{enumerate}
\item If \(h\) is bounded, then \(h\) is Riemann-Stieltjes integrable with
respect to \(F\) on \([\vect{a},\vect{b}]\) iff \(h\) is continuous
\(\lambda_{F}\)-a.e.\ on \([\vect{a},\vect{b}]\).
\item If \(h\) is Riemann-Stieltjes integrable with respect to \(F\) on
\([\vect{a},\vect{b}]\), then \(h\) is also Lebesgue-Stieltjes integrable with
respect to \(\lambda_{F}\), with the Lebesgue-Stieltjes integral
\(\int_{[\vect{a},\vect{b}]}^{}h\odif{\lambda_{F}}\) being equal to the
Riemann-Stieltjes integral \(\int_{[\vect{a},\vect{b}]}^{}h\odif{F}\).
\end{enumerate}
\begin{pf}
See \textcite{terhorst1984riemann}.
\end{pf}

\begin{note}
In the context of computing expectation of \(h(\vect{X})\) where \(\vect{X}\)
is a random vector under a probability measure \(\pr\), since we have
\(\lambda_{F}=\pr_{\vect{X}}\) by \Cref{thm:dist-fn-char-prob-meas-rd}, the
Lebesgue-Stieltjes integral \(\int_{[\vect{a},\vect{b}]}^{}h\odif{\lambda_{F}}\)
can indeed be expressed as \(
\int_{[\vect{a},\vect{b}]}^{}h(\vect{x})\odif{\pr_{\vect{X}}(\vect{x})}
\overset{\text{(notation)}}{=}\int_{[\vect{a},\vect{b}]}^{}h(\vect{x})\odif{F(\vect{x})}\).
It shares the same notation as the one for Riemann integral, and indeed can be
computed as a Riemann-Stieltjes integral (if exists) by
\Cref{prp:rs-ls-int-relate}. This serves as the main tool for computing such
Lebesgue-Stieltjes integral.
\end{note}

A famous example concerning the difference between Riemann and Lebesgue
integral is the \emph{Dirichlet function} \(h(x)=\indic_{\Q}(x)\) for all
\(x\in[0,1]\). Since it is bounded and discontinuous on \([0,1]\) (with
\(\lambda([0,1])=1\)), it follows by \Cref{prp:rs-ls-int-relate} that \(h\) is
not Riemann integrable. However, it is a simple function and is Lebesgue
integrable: We have \(\int_{[0,1]}^{}h\odif{\lambda}=1\cdot \lambda(\Q)+0\cdot
\lambda([0,1]\setminus \Q)=0\).
\end{proposition}
\item \textbf{Fubini-Tonelli theorem.} The main tool for computing multivariate
integrals is the \emph{Fubini-Tonelli theorem}; perhaps you have seen some
special cases of this result in your previous multivariable calculus course.

\begin{theorem}[Fubini-Tonelli]
\label{thm:fubini-tonelli}
Let \((\Omega_1,\mathcal{F}_1,\mu_1)\) and \((\Omega_2,\mathcal{F}_2,\mu_2)\)
be two measure spaces where \(\mu_1\) and \(\mu_2\) are \(\sigma\)-finite, and
consider the product space \((\Omega,\mathcal{F},\mu)=(\Omega_1\times
\Omega_2,\mathcal{F}_1\otimes\mathcal{F}_2,\mu_1\times \mu_2)\). Let
\(h:\Omega\to\eR\) be a
\((\mathcal{F}_1\otimes\mathcal{F}_2,\mathcal{B}(\eR))\)-measurable function.

If \(h\in L_{+}(\Omega,\mathcal{F},\mu)\) \emph{(Tonelli)} or \(h\in
L^{1}(\Omega,\mathcal{F},\mu)\) \emph{(Fubini)}, then
\(\omega_1\mapsto\vc{\int_{\Omega_2}^{}h(\omega_1,\omega_2)\odif{\mu_2(\omega_2)}}\)
and
\(\omega_2\mapsto\orc{\int_{\Omega_1}^{}h(\omega_1,\omega_2)\odif{\mu_1(\omega_1)}}\)
are \(\mathcal{F}_1\)-measurable and \(\mathcal{F}_2\)-measurable respectively,
and
\[
\int_{\Omega}^{}h\odif{\mu}
=\int_{\Omega_1}^{}
\left(\vc{\int_{\Omega_2}^{}h(\omega_1,\omega_2)\odif{\mu_2(\omega_2)}}\right)\odif{\mu_1(\omega_1)}
=\int_{\Omega_2}^{}
\left(\orc{\int_{\Omega_1}^{}h(\omega_1,\omega_2)\odif{\mu_1(\omega_1)}}\right)\odif{\mu_2(\omega_2)}.
\]
\end{theorem}
\begin{pf}
Omitted.
\end{pf}

\begin{remark}
\item \emph{Tonelli's theorem} refers to the result with condition \(h\in
L_{+}\), and \emph{Fubini's theorem} refers to the result with \(h\in L^1\).
The result above is a combination of these two theorems.
\item The Fubini-Tonelli theorem gives us conditions under which changing the
order of integration is allowed. The result can be extended to the case where
finitely many (not just two) integrals are involved by induction. 
\item A typical procedure in applying the Fubini-Tonelli theorem is as follows:
\begin{enumerate}[label={(\arabic*)}]
\item Apply \emph{Tonelli's theorem} to \(|h|\in L_{+}\) for calculating
\(\int_{\R^d}^{}|h(\vect{x})|\odif{F(\vect{x})}\).
\item If \(\int_{\R^d}^{}|h(\vect{x})|\odif{F(\vect{x})}<\infty\), then \(h\in L^{1}\).
After that, apply \emph{Fubini's theorem} to \(h\in L^{1}\) for calculating
\(\int_{\R^d}^{}h(\vect{x})\odif{F(\vect{x})}\) (the actual integral of interest).
\end{enumerate}
\end{remark}
\item \textbf{Expectation of products under independence.}
With the Fubini-Tonelli theorem, we can establish the following result which
can often substantially simplify the calculations of expectations under
independence.

\begin{proposition}
\label{prp:expv-prod-indp}
Let \(X_1,\dotsc,X_d\in L^{1}\) be independent. Then, we have \(\expv{X_1\dotsb
X_d}=\expv{X_1}\dotsb\expv{X_d}\).
\end{proposition}
\begin{pf}
First note that under the independence of \(X_1,\dotsc,X_d\), by
\Cref{thm:char-rv-ind} the joint distribution function \(F\) of
\(X_1,\dotsc,X_d\) is given by \(F(\vect{x})=\prod_{j=1}^{d}F_j(x_j)\), where
\(F_j\) is the distribution function of \(X_j\) for each \(j=1,\dotsc,d\).
Interpreting \(F\) and each \(F_j\) as distributions (push-forward measures), we
can then see that \(F\) can be seen as the product measure \(\prod_{j=1}^{d}F_j\).

Then, we follow the typical procedure in applying the Fubini-Tonelli theorem
suggested above.
\begin{enumerate}[label={(\arabic*)}]
\item By \emph{Tonelli's theorem}, we have
\begin{align*}
\expv{|X_1\dotsb X_d|}&=\expv{|X_1|\dotsb|X_d|}
=\int_{\R^d}^{}|x_1|\dotsb|x_d|\odif{F(\vect{x})}
\overset{\text{(Tonelli)}}{=}
\int_{\R}\dotsi\int_{\R}^{}|x_1|\dotsb|x_d|\odif{F_1(x_1)}\dotsb\odif{F_d(x_d)} \\
\overset{\text{(taking out constants)}}&{=}\prod_{j=1}^{d}\int_{\R}^{}|x_j|\odif{F_j(x_j)}=
\expv{|X_1|}\dotsb\expv{|X_d|}\overset{(X_1,\dotsc,X_d\in L^1)}{<}\infty.
\end{align*}
Hence, \(X_1\dotsb X_d\in L^{1}\).
\item Applying \emph{Fubini's theorem} gives
\begin{align*}
\expv{X_1\cdots X_d}&=\int_{\R^d}^{}x_1\dotsb x_d\odif{F(\vect{x})}
\overset{\text{(Fubini)}}{=}
\int_{\R}\dotsi\int_{\R}^{}x_1\dotsb x_d\odif{F_1(x_1)}\dotsb\odif{F_d(x_d)} \\
&=\prod_{j=1}^{d}\int_{\R}^{}x_j\odif{F_j(x_j)}=
\expv{X_1}\dotsb\expv{X_d}.
\end{align*}
\end{enumerate}
\end{pf}
\item \textbf{Expectation formula in terms of quantile/survival function.}
While the change of variables formula in \Cref{thm:change-of-var} is a common
tool for computing expectations, it is not the only one. Below, we will provide
two alternative methods for computing expectations: one is based on \emph{quantile
function} and another is based on \emph{survival function}.

\begin{proposition}
\label{prp:expv-quant-surv}
Let \((\Omega,\mathcal{F},\pr)\) be a probability space, \(X\sim F\), and
\(X\in L^1\). Then,
\(\expv{X}=\int_{0}^{1}F^{-1}(u)\odif{u}=\int_{0}^{\infty}\bar{F}(x)\odif{x}
-\int_{-\infty}^{0}F(x)\odif{x}\), where \(F^{-1}\) and \(\bar{F}\) denote
quantile function and survival function respectively.
\end{proposition}
\begin{note}
In the special case where \(X\) is nonnegative, we can simplify the formula in
terms of survival function to \(\expv{X}=\int_{0}^{\infty}\bar{F}(x)\odif{x}\),
which is very useful in \emph{survival analysis} (e.g., \(X\) represents
lifetime).
\end{note}

\begin{pf}
By quantile transform, we have \(X\eqd F^{-1}(U)\) with \(U\sim\unif{0}{1}\).
Hence, \(\expv{X}=\expv{F^{-1}(U)}=\int_{0}^{1}F^{-1}(u)\odif{u}\),
establishing the formula in terms of quantile function. The formula in
terms of survival function can be shown analytically through integration by
parts, but is best understood geometrically as follows:
\begin{center}
\begin{tikzpicture}[scale=0.9,
declare function={
f(\x)=(\x<=\fpeval{0.2*exp(2)})*ln(\x/0.2)
+and(x>\fpeval{0.2*exp(2)},x<=4)*2
+and(x>4, x<=5)*(-5+2*x);
}
]
\begin{axis}[domain=-0.5:5, axis lines=middle, samples=200, xmax=5.5,
ymax=5.5, ymin=-0.5, xtick={5}, xticklabels={1}, xticklabel style={yshift=0.6cm},
ytick=\empty, ylabel={\(x\)},xlabel={\(y\)},
xlabel style={anchor=west},
ylabel style={anchor=south},
]
\addplot[draw=none, name path=A]{0};
\addplot[draw=none, name path=B]{f(x)};
\addplot[ForestGreen, domain=4:5]{-5+2*x};
\addplot[ForestGreen, domain=\fpeval{0.2*exp(2)}:4]{2};
\addplot[ForestGreen, domain=0:\fpeval{0.2*exp(2)}]{ln(x/0.2)};
\addplot[violet, opacity=0.2] fill between[of=A and B, soft clip={domain=0.2:5}];
\addplot[brown, opacity=0.3] fill between[of=A and B, soft clip={domain=0:0.2}];
\draw[ForestGreen, fill] (4,2) circle [radius=0.8mm];
\node[ForestGreen] () at (1,2.5) {\(x=F^{-1}(y)\)};
\draw[dashed, ForestGreen] (4,2) -- (4,3);
\addplot[magenta, dotted]{x};
\node[magenta] () at (3,3) {\(y=x\)};
\end{axis}
\end{tikzpicture}
\begin{tikzpicture}[scale=0.9,
declare function={
f(\x)=(\x<=2)*(0.2*exp(x))
+and(x>2,x<=3)*4
+and(x>3, x<=5)*(2.5+x/2);
}
]
\begin{axis}[domain=-0.5:5, axis lines=middle, samples=200,
ymax=5.5, ytick={5}, yticklabels={1},
xtick=\empty, ylabel={\(y\)},xlabel={\(x\)},
xlabel style={anchor=west},
ylabel style={anchor=south},
]
\addplot[draw=none, name path=A]{0};
\addplot[draw=none, name path=B]{5};
\addplot[draw=none, name path=C]{f(x)};
\addplot[violet, opacity=0.2] fill between[of=B and C, soft clip={domain=0:5}];
\addplot[brown, opacity=0.3] fill between[of=A and C, soft clip={domain=-0.5:0}];
\addplot[blue, domain=-0.5:2]{0.2*exp(x)};
\addplot[blue, domain=3:5]{2.5+x/2};
\draw[blue, fill] (2,4) circle [radius=0.8mm];
\addplot[blue, domain=2:3]{4};
\node[blue] () at (2.5,1) {\(y=F(x)\)};
\draw[dashed, blue] (2,\fpeval{0.2*exp(2)}) -- (2,4);
\addplot[magenta, dotted]{x};
\node[magenta] () at (3,3) {\(y=x\)};
\end{axis}
\end{tikzpicture}
\end{center}
For the graph of distribution function (right), the \vc{violet} area
corresponds to the integral
\(\int_{0}^{\infty}\bar{F}(x)\odif{x}=\int_{0}^{\infty}1-F(x)\odif{x}\), while
the \brc{brown} area corresponds to the integral
\(\int_{-\infty}^{0}F(x)\odif{x}\).
\end{pf}
\end{enumerate}
\subsection{Variance, Covariance, and Correlation}
\begin{enumerate}
\item Apart from expectation, you should have also learnt other kinds of
probabilistic quantities like \emph{variance}, \emph{covariance}, and
\emph{correlation} in your first probability course. Here, we will briefly
review them and study some related results.
\item \textbf{Definitions.} Let \((\Omega,\mathcal{F},\pr)\) be a probability
space and \(X,Y\in L^2\). Then the \defn{variance} and \defn{standard
deviation} of \(X\) are \(\vari{X}:=\expv{(X-\expv{X})^{2}}\) and
\(\sd{X}:=\sqrt{\vari{X}}\) respectively. The \defn{covariance} and \defn{correlation}
of \(X\) and \(Y\) are \(\cov{X,Y}:=\expv{(X-\expv{X})(Y-\expv{Y})}\) and
\(\corr{X,Y}:=\frac{\cov{X,Y}}{\sqrt{\vari{X}\vari{Y}}}\) respectively.

\begin{note}
The correlation \(\corr{X,Y}\) serves as a measure of \emph{linear} dependence
between \(X\) and \(Y\). Some other measures of dependence between \(X\) and
\(Y\) are available, such as the \defn{Kendall's tau} \\
\(\tau=\corr{\indicset{F_X(X)\le F_X(X')},\indicset{F_X(X)\le F_X(X')}}\)
and the \defn{Spearman's rho} \(\rho_S=\corr{F_X(X),F_Y(Y)}\), where \(F_X\)
and \(F_Y\) are distribution functions of \(X\) and \(Y\) respectively, and
\(X',Y'\) are \emph{independent copies} of \(X,Y\) respectively, i.e.,
\(X,X'\iid F_X\) and \(Y,Y'\iid F_Y\). These measures can also capture some
nonlinear dependence between \(X\) and \(Y\).

While the expression of Kendall's tau is more complicated, it turns out to be
easier to compute than Spearman's rho.
\end{note}
\item\label{it:var-cov-corr-prop} \textbf{Properties of variance, covariance,
and correlation.} Let \((\Omega,\mathcal{F},\pr)\) be a probability space and
\(X,Y\in L^2\).
\begin{enumerate}
\item \emph{(well-known formula for variance)} \(\vari{X}=\expv{X^{2}}-\expv{X}^{2}\).
\item \emph{(well-known formula for covariance)} \(\cov{X,Y}=\expv{XY}-\expv{X}\expv{Y}\).
\item \(\cov{X,X}=\vari{X}\).
\item \emph{(symmetry)} \(\cov{Y,X}=\cov{X,Y}\).
\item \(\cov{X,c}=0\) for all \(c\in\R\).
\item \emph{(criterion for zero variance)} \(\vari{X}=0\) iff \(X\eqas \expv{X}\).
\item \(\vari{aX+bY}=a^{2}\vari{X}+b^{2}\vari{Y}+2ab\cov{X,Y}\) for all \(a,b\in\R\).

\begin{note}
If we have \(Y\equiv 1\), then this formula reduces to \(\vari{aX+b}=a^{2}\vari{X}\).
\end{note}
\item \emph{(independence implies uncorrelated)} If \(X\) and \(Y\) are
independent, then \(\cov{X,Y}=\corr{X,Y}=0\) (in this case we say that \(X\)
and \(Y\) are \defn{uncorrelated}).

\begin{warning}
It should be well known that the converse does \underline{not} hold. For
instance, if we have \(X\sim\unif{-1}{1}\) and \(Y=X^2\), then
\(\cov{X,Y}=\expv{X^{3}}-\expv{X}\expv{X^{2}}=0-0=0\) (and also
\(\corr{X,Y}=0\)) but definitely \(X\) and \(Y\) are not independent.
\end{warning}
\end{enumerate}
\begin{pf}
\begin{enumerate}
\item \(\vari{X}=\expv{(X-\expv{X})^{2}}=\expv{\orc{X^{2}}-2X\vc{\expv{X}}+\mgc{\expv{X}^{2}}}
=\expv{\orc{X^{2}}}-2\vc{\expv{X}}\expv{X}+\mgc{\expv{X^{2}}}=\expv{X^{2}}-\expv{X}^{2}\).
\item \(\cov{X,Y}=\expv{(X-\expv{X})(Y-\expv{Y})}=\expv{XY-\expv{X}Y-X\expv{Y}+\expv{X}\expv{Y}}
=\expv{XY}-2\expv{X}\expv{Y}+\expv{X}\expv{Y}=\expv{XY}-\expv{X}\expv{Y}\).
\item \(\cov{X,X}\overset{\text{(b)}}{=}\expv{X^{2}}-\expv{X}^{2}\overset{\text{(a)}}{=}\vari{X}\).
\item \(\cov{Y,X}\overset{\text{(b)}}{=}\expv{YX}-\expv{Y}\expv{X}
=\expv{XY}-\expv{X}\expv{Y}\overset{\text{(b)}}{=}\cov{X,Y}\).
\item \(\cov{X,c}\overset{\text{(b)}}{=}\expv{cX}-\expv{c}\expv{X}=c\expv{X}-c\expv{X}=0\).
\item \(\vari{X}=\expv{(X-\expv{X})^{2}}=0\overset{\text{\labelcref{it:nonneg-leb-int-prop}}}{\iff}
(X-\expv{X})^{2}\eqas 0\iff X\eqas \expv{X}\).
\item \(\vari{aX+bY}=\expv{\bigl((aX+bY)-\expv{aX+bY}\bigr)^{2}}
=\expv{\big(a(X-\expv{X})+b(Y-\expv{Y})\big)^{2}}
=\expv{a^{2}(X-\expv{X})^{2}}
+\expv{2ab(X-\expv{X})(Y-\expv{Y})}
+\expv{b^{2}(Y-\expv{Y})^{2}}
=a^{2}\vari{X}+2ab\cov{X,Y}+b^{2}\vari{Y}
\).
\item Since \(X\) and \(Y\) are independent, \(\expv{XY}=\expv{X}\expv{Y}\) by
\Cref{prp:expv-prod-indp}. Hence,
\(\cov{X,Y}\overset{\text{(b)}}{=}\expv{XY}-\expv{X}\expv{Y}=0\), which
implies by definition that \(\corr{X,Y}=0\).
\end{enumerate}
\end{pf}
\item \textbf{Cauchy-Schwarz inequality.} We have already mentioned
the \emph{Cauchy-Swartz inequality} in \labelcref{it:holder-cs-ineq}, as a special
case of the H\"older inequality. Now, we will express it in terms of the
probabilistic quantities introduced here and prove the inequality, together
with an associated result about correlation.

\begin{proposition}[Cauchy-Swartz inequality and bounds on correlation]
\label{prp:cs-ineq-corr-bound}
Let \((\Omega,\mathcal{F},\pr)\) be a probability space and \(X,Y\in L^{2}\). Then,
\begin{enumerate}
\item \emph{(Cauchy-Swartz inequality)}
\(|\expv{XY}|\le(\expv{X^{2}}\expv{Y^{2}})^{1/2}\) where the equality holds iff
\(Y\eqas mX\) for some \(m\in\R\).
\item \emph{(bounds on correlation)} \(-1\le\corr{X,Y}\le 1\), and we have
\(\corr{X,Y}=1\) (\(-1\) resp.)\ iff \(Y\eqas mX+c\) where \(m>0\) (\(m<0\)
resp.).
\end{enumerate}
\end{proposition}
\begin{pf}
\begin{enumerate}
\item Let \(Z_t:=tX+Y\) for every \(t\in\R\). Then, we have
\(0\le\expv{Z_t^{2}}=t^{2}\expv{X^{2}}+2t\expv{XY}+\expv{Y^{2}}=:at^{2}+bt+c\).
So \(at^{2}+bt+c\) is a polynomial in \(t\) with at most one root (as it is
always nonnegative). Hence, its discriminant satisfies \(\Delta=b^{2}-4ac\le
0\), which implies that \(4\expv{XY}^{2}-4\expv{X^{2}}\expv{Y^{2}}\le 0\), and
hence \(|\expv{XY}|\le(\expv{X^{2}}\expv{Y^{2}})^{1/2}\).

Also, we have
\begin{align*}
\text{the equality holds}&\iff b^2-4ac=0 \\
&\iff at^{2}+bt+c=0\text{ for some unique \(t\in\R\)} \\
&\iff \expv{Z_{t}^{2}}=0\text{ for such \(t\in\R\)} \\
\overset{\text{\labelcref{it:nonneg-leb-int-prop}}}&{\iff} Z_t\eqas 0\text{ for such \(t\in\R\)} \\
&\iff Y\eqas -tX\text{ for such \(t\in\R\)}.
\end{align*}
\item Applying the Cauchy-Swartz inequality to \(\widetilde{X}:=X-\expv{X}\)
and \(\widetilde{Y}:=Y-\expv{Y}\) yields
\[
|\cov{X,Y}|\overset{\text{(by construction)}}{=}
\left|\expv{\widetilde{X}\widetilde{Y}}\right|
\overset{\text{(Cauchy-Swartz)}}{\le}\left(\expv{\widetilde{X}^{2}}\expv{\widetilde{Y}^{2}}\right)^{1/2}
\overset{\text{(by construction)}}{=}
(\vari{X}\vari{Y})^{1/2},
\]
where the equality holds iff \(\widetilde{Y}\eqas m\widetilde{X}\) for some
\(m\in\R\) iff \(Y\eqas mX+c\) for such \(m\in\R\), with \(c=\expv{Y}-m\expv{X}\).

This implies that
\(|\corr{X,Y}|=\frac{|\cov{X,Y}|}{\sqrt{\vari{X}\vari{Y}}}\le 1\), and the
equality holds (i.e., \(\corr{X,Y}=\pm 1\)) iff \(Y\eqas mX+c\); in such case,
we have \(\corr{X,Y}=\frac{\cov{X}{mX+c}}{\sqrt{\vari{X}\vari{mX+c}}}
=\frac{m\vari{X}}{\sqrt{m^{2}\vari{X}^{2}}}=\frac{m}{|m|}\), which equals \(1\)
(\(-1\) resp.)\ iff \(m>0\) (\(m<0\) resp.).
\end{enumerate}
\end{pf}
\item \textbf{Hoeffding's lemma.} Another less well-known result about
covariance is the \emph{Hoeffding's lemma}, which can be used to determine the
maximum/minimum possible values of correlations given the two marginals (it
turns out that some values of correlation \emph{are not attainable} for
certain margins!).

\begin{lemma}[Hoeffding's lemma]
\label{lma:hoeffding}
Let \((\Omega,\mathcal{F},\pr)\) be a probability space,
and let \((X_1,X_2)\sim F\) with margins \(F_1,F_2\) respectively and
\(X_1,X_2\in L^2\). Then,
\[\cov{X_1,X_2}=\int_{-\infty}^{\infty}\int_{-\infty}^{\infty}
F(x_1,x_2)-F_1(x_1)F_2(x_2)\odif{x_1}\odif{x_2}.\]
\end{lemma}
\begin{pf}
Let \((X_1',X_2')\) be an \emph{independent copy} of \((X_1,X_2)\), i.e.,
\((X_1,X_2),(X_1',X_2')\iid F\). The result then follows from the following
chain of equalities:
\begin{align*}
2\cov{X_1,X_2}&=\expv{(X_1-\expv{X_1})(X_2-\expv{X_2})}+\expv{(X_1'-\expv{X_1'})(X_2'-\expv{X_2'})} \\
\overset{\text{(independence)}}&{=}
\expv{\bigl((X_1-\expv{X_1})-(X_1'-\expv{X_1'})\bigr)
\cdot\bigl((X_2-\expv{X_2})-(X_2'-\expv{X_2'})\bigr)} \\
\overset{(\expv{X_j}=\expv{X_j'})}&{=}\expv{\vc{(X_1-X_1')}\orc{(X_2-X_2')}} \\
&=\expv{\vc{\left(\int_{-\infty}^{\infty}
\indicset{X_1'\le x_1}-\indicset{X_1\le x_1}\odif{x_1}\right)}
\orc{\left(\int_{-\infty}^{\infty}\indicset{X_2'\le x_2}-\indicset{X_2\le x_2}\odif{x_1}\right)}
} \\
&=\expv{\int_{-\infty}^{\infty}\int_{-\infty}^{\infty}
\vc{\left(\indicset{X_1'\le x_1}-\indicset{X_1\le x_1}\right)}
\orc{\left(\indicset{X_2'\le x_2}-\indicset{X_2\le x_2}\right)}
\odif{x_1}\odif{x_2}} \\
\overset{\text{(Fubini)}}&{=}\int_{-\infty}^{\infty}\int_{-\infty}^{\infty}
\expv{
\vc{\left(\indicset{X_1'\le x_1}-\indicset{X_1\le x_1}\right)}
\orc{\left(\indicset{X_2'\le x_2}-\indicset{X_2\le x_2}\right)}
}
\odif{x_1}\odif{x_2} \\
&=2\int_{-\infty}^{\infty}\int_{-\infty}^{\infty}
F(\vc{x_1},\orc{x_2})-\vc{F_1(x_1)}\orc{F_2(x_2)}
-\vc{F_1(x_1)}\orc{F_2(x_2)}+F(\vc{x_1},\orc{x_2})
\odif{x_2} \\
&=2\int_{-\infty}^{\infty}\int_{-\infty}^{\infty}
F(x_1,x_2)-F_1(x_1)F_2(x_2)\odif{x_1}\odif{x_2}.
\end{align*}
\end{pf}

\begin{note}
To apply Hoeffding's lemma for determining the maximum/minimum possible values
of correlations given two marginals, we can use the Sklar's theorem
(\Cref{thm:sklar}) in conjunction with the Fr\'echet-Hoeffding bounds
(\Cref{thm:frechet-hoeffding-bounds}), which provide bounds for \(F(x_1,x_2)\),
namely \(W(F_1(x_1),F_2(x_2))\le F(x_1,x_2)\le M(F_1(x_1),F_2(x_2))\),
by writing \(F(x_1,x_2)=C(F_1(x_1),F_2(x_2))\) where \(C\) is a copula of
\(F\). 

Once the margins are fixed, the upper and lower bounds here are fixed (so do
the variances of \(X_1\) and \(X_2\)). Hence, we can then bound the covariance
(and thus also correlation) by the Hoeffding lemma as follows:
\begin{align*}
\int_{-\infty}^{\infty}\int_{-\infty}^{\infty}
W(F_1(x_1),F_2(x_2))-F_1(x_1)F_2(x_2)\odif{x_1}\odif{x_2}
&\le\cov{X_1,X_2} \\
&\le
\int_{-\infty}^{\infty}\int_{-\infty}^{\infty}
M(F_1(x_1),F_2(x_2))-F_1(x_1)F_2(x_2)\odif{x_1}\odif{x_2}.
\end{align*}
\end{note}
\end{enumerate}
\subsection{Multivariate Versions of Probabilistic Quantities}
\label{subsect:multivar-prob-quant}
\begin{enumerate}
\item After studying some probabilistic quantities, here we are going to study
their \emph{multivariate versions} (serving as generalizations of the concepts
to higher dimensions), which may have been discussed in your previous linear
regression course.
\item \textbf{Definitions.} Let \(X_1,\dotsc,X_d\in L^{1}\). Then,
\(\expv{\vect{X}}:=(\expv{X_1},\dotsc,\expv{X_d})\) is the \defn{mean vector}
or \defn{expectation} of \(\vect{X}\).

Let \(X_1,\dotsc,X_d,Y_1,\dotsc,Y_p\in L^{2}\). Then,
\(\cov{\vect{X},\vect{Y}}
:=\begin{bmatrix}\cov{X_i,Y_j}\end{bmatrix}_{i,j=1}^{d,p}\) is the
\defn{cross-variance matrix} of \(\vect{X}=(X_1,\dotsc,X_d)\) and
\(\vect{Y}=(Y_1,\dotsc,Y_p)\). The \defn{covariance matrix} of \(\vect{X}\) is
\(\cov{\vect{X}}:=\cov{\vect{X},\vect{X}}\) and the \defn{correlation matrix} of \(\vect{X}\)
is \(\corr{\vect{X}}:=\begin{bmatrix}\corr{X_i,Y_j}\end{bmatrix}_{i,j=1}^{d,d}\).
\begin{note}
Here the notation \(\begin{bmatrix}a_{ij}\end{bmatrix}_{i,j=1}^{m,n}\) denotes
the \(m\times n\) matrix \(\begin{bmatrix}
a_{11}&\cdots&a_{1n}\\
\vdots&\ddots&\vdots\\
a_{m1}&\cdots&a_{mn}\\
\end{bmatrix}\).
\end{note}
\item\label{it:mult-var-prob-quant-prop} \textbf{Properties of multivariate
probabilistic quantities.}
\begin{enumerate}
\item\label{it:mean-vec-linearity} \emph{(linearity)} Let \(\vect{X}=(X_1,\dotsc,X_d)\)
with \(X_1,\dotsc,X_d\in L^{1}\).
Then, \(\expv{A\vect{X}+\vect{b}}=A\expv{\vect{X}}+\vect{b}\), for all \(A\) and
\(\vect{b}\) such that the matrix-vector operations are defined.
\item\label{it:lin-cross-cov} \emph{(effects of linear operations on cross-covariance matrix)} Let
\(\vect{X}=(X_1,\dotsc,X_d)\) and \(\vect{Y}=(Y_1,\dotsc,Y_p)\) with
\(X_1,\dotsc,X_d,Y_1,\dotsc,Y_p\in L^{2}\). Then,
\(\cov{A\vect{X}+\vect{b},C\vect{Y}+\vect{d}}=A\cov{\vect{X},\vect{Y}}C^{T}\)
for all \(A,\vect{b},C,\vect{d}\) such that the matrix-vector operations are
defined.

\item\label{it:cov-sums} \emph{(covariance matrix of a sum)} \(\vect{X}=(X_1,\dotsc,X_{\vc{d}})\)
and \(\vect{Y}=(Y_1,\dotsc,Y_{\vc{d}})\) with
\(X_1,\dotsc,X_d,Y_1,\dotsc,Y_d\in L^{2}\). Then,
\(\cov{\vect{X}+\vect{Y}}=\cov{\vect{X}}+\cov{\vect{Y}}
+\cov{\vect{X},\vect{Y}}+\cov{\vect{X},\vect{Y}}^{T}\).
\end{enumerate}
\begin{pf}
We shall use the notations \(\vect{a}_j\) and \(A_{ij}\) to denote the \(j\)th
entry of the vector \(\vect{a}\) and the \((i,j)\)th entry of the matrix \(A\)
respectively.
\begin{enumerate}
\item Entrywise, we have
\[
\expv{A\vect{X}+\vect{b}}_{j}=\expv{\sum_{k=1}^{d}a_{jk}X_k+b_j}
=\sum_{k=1}^{d}a_{jk}\expv{X_k}+b_j
=(A\expv{\vect{X}}+\vect{b})_{j}.
\]
\item Entrywise, we have
\begin{align*}
\cov{A\vect{X}+\vect{b},C\vect{Y}+\vect{d}}_{ij}
&=\cov{(A\vect{X}+\vect{b})_{i},(C\vect{Y}+\vect{d})_{j}}
=\cov{\sum_{k=1}^{d}a_{ik}X_k+b_i,\sum_{\ell=1}^{p}c_{j\ell}Y_{\ell}+d_j} \\
&=\expv{\left(\sum_{k=1}^{d}\vc{a_{ik}(X_k-\expv{X_k})}\right)
\left(\sum_{\ell=1}^{p}\orc{c_{j\ell}(Y_{\ell}-\expv{Y_{\ell}})}\right)} \\
&=\expv{\sum_{k=1}^{d}\sum_{\ell=1}^{p}\vc{a_{ik}(X_k-\expv{X_k})}
\orc{c_{j\ell}(Y_{\ell}-\expv{Y_{\ell}}})} \\
&=\sum_{k=1}^{d}\sum_{\ell=1}^{p}a_{ik}c_{j\ell}
\expv{\vc{(X_k-\expv{X_k})}\orc{(Y_{\ell}-\expv{Y_{\ell}}})}
=\sum_{k=1}^{d}\sum_{\ell=1}^{p}a_{ik}c_{j\ell}\cov{X_k,Y_{\ell}} \\
&=(A\cov{\vect{X},\vect{Y}}C^{T})_{ij}.
\end{align*}

\item Entrywise, we have
\begin{align*}
\cov{\vect{X}+\vect{Y}}_{ij}&=\cov{(\vect{X}+\vect{Y})_{i},(\vect{X}+\vect{Y})_{j}}
=\cov{X_i+Y_i,X_j+Y_j} \\
&=\expv{\left(X_i+Y_i-\expv{X_i+Y_i}\right)\left(X_j+Y_j-\expv{X_j+Y_j}\right)} \\
&=\expv{\left(X_i-\expv{X_i}+Y_i-\expv{Y_i}\right)\left(X_j-\expv{X_j}+Y_j\expv{Y_j}\right)} \\
&=\expv{(X_i-\expv{X_i})(X_j-\expv{X_j})}+\expv{(X_i-\expv{X_i})(Y_j-\expv{Y_j})} \\
&\quad+\expv{(Y_i-\expv{Y_i})(X_j-\expv{X_j})}+\expv{(Y_i-\expv{Y_i})(Y_j-\expv{Y_j})} \\
&=(\cov{\vect{X},\vect{X}})_{ij}+(\cov{\vect{X},\vect{Y}})_{ij}+(\cov{\vect{Y},\vect{X}})_{ij}
+(\cov{\vect{Y},\vect{Y}})_{ij} \\
&=(\cov{\vect{X}})_{ij}+(\cov{\vect{X},\vect{Y}})_{ij}+(\cov{\vect{X},\vect{Y}})_{\vc{ji}}
+(\cov{\vect{Y}})_{ij} \\
&=(\cov{\vect{X}}+\cov{\vect{Y}}
+\cov{\vect{X},\vect{Y}}+\cov{\vect{X},\vect{Y}}^{T})_{ij}.
\end{align*}
\end{enumerate}
\end{pf}
\item \textbf{Implications of the properties.} The properties in
\labelcref{it:mult-var-prob-quant-prop} are quite general and can be used to
derive some more familiar properties as special cases:
\begin{enumerate}
\item \emph{(\labelcref{it:mean-vec-linearity} with \(A=\vect{a}^{T}\))}
\(\expv{\vect{a}^{T}\vect{X}}=\vect{a}^{T}\expv{\vect{X}}\).
\item \emph{(\labelcref{it:lin-cross-cov} with \(C=A\) and \(\vect{d}=\vect{b}\))}
\(\cov{A\vect{X}+\vect{b}}=A\cov{\vect{X}}A^{T}\)
\item \emph{(\labelcref{it:lin-cross-cov} with \(C=A=\vect{a}^{T}\) and \(\vect{d}=\vect{b}=\vect{0}\))}
\(\vari{\vect{a}^{T}\vect{X}}=\vect{a}^{T}\cov{\vect{X}}\vect{a}\) (a scalar).
\begin{enumerate}
\item Setting \(\vect{a}=(1,\dotsc,1)\) gives the \emph{variance of sum} formula:
\[
\vari{\sum_{i=1}^{d}X_i}=\sum_{i=1}^{d}\sum_{j=1}^{d}\cov{X_i,X_j}
=\sum_{i=1}^{d}\vari{X_i}+2\sum_{1\le i<j\le d}^{}\cov{X_{i},X_{j}}.
\]
If we have further that \(X_i\) and \(X_j\) are uncorrelated for all
\(i\ne j\), then it reduces to \(\vari{\sum_{i=1}^{d}X_i}=\sum_{i=1}^{d}\vari{X_i}\).
\item Setting \(\vect{a}=(1/d,\dotsc,1/d)\) and assuming that \(X_i\) and \(X_j\)
are uncorrelated for all \(i\ne j\), we have \(\vari{(\sum_{i=1}^{d}X_i)/d}
=(\sum_{i=1}^{d}\vari{X_i})/d^{2}\).

If we have further that \(X_1,\dotsc,X_d\) are identically distributed, then
it reduces to \(\vari{(\sum_{i=1}^{d}X_i)/d}=(\vari{X_1})/d\).
\end{enumerate}
\end{enumerate}
\item \textbf{Characterization of covariance matrix.} To close
\Cref{subsect:multivar-prob-quant}, we will introduce a characterization of
covariance matrix. To prove such characterization, the following result from
linear algebra is helpful.
\begin{lemma}[Cholesky decomposition]
\label{lma:cholesky-decomp}
Let \(\Sigma\in\R^{d\times d}\) be a symmetric and positive semi-definite
matrix.  Then, we can write \(\Sigma=AA^{T}\) for some unique \(A\in\R^{d\times
d}\), which is a lower triangular matrix with nonnegative diagonal elements
(which are further positive if \(\Sigma\) is \emph{positive definite}); such
\(A\) is known as the \defn{Cholesky factor}.
\end{lemma}
\begin{note}
A symmetric matrix \(\Sigma\in\R^{d\times d}\) is \defn{positive semi-definite}
if \(\vect{x}^{T}\Sigma\vect{x}\ge 0\) for all \(\vect{x}\in\R^d\), and is
\defn{positive definite} if \(\vect{x}^{T}\Sigma\vect{x}>0\) for all
\(\vect{x}\in\R^d\setminus \{\vect{0}\}\). From these definitions, it should be
clear that a positive definite matrix is always positive semi-definite also,
but not vice versa.
\end{note}
\begin{proposition}[Characterization of covariance matrix]
\label{prp:cov-matx-char}
A real symmetric matrix \(\Sigma\in\R^{d\times d}\) is a covariance matrix iff
it is positive semi-definite.
\end{proposition}
\begin{pf}
``\(\Rightarrow\)'': Assume \(\Sigma\) is the covariance matrix of a random
vector \(\vect{X}\). Then for all \(\vect{a}\in\R^d\) we have
\(\vect{a}^{T}\Sigma\vect{a}=\vari{\vect{a}^{T}\vect{X}}\ge 0\), so \(\Sigma\)
is positive definite.

``\(\Leftarrow\)'': Assume \(\Sigma\) is positive semi-definite with the
Cholesky factor \(A\). Let \(\vect{X}=(X_1,\dotsc,X_d)\) with
\(X_1,\dotsc,X_d\iid\ndist{0,1}\).\footnote{Indeed, any other random variables
that are pairwise uncorrelated and have unit variance each would also work.} Then,
its covariance matrix is \(\cov{\vect{X}}=I_d\), and thus
\(\cov{A\vect{X}}=A\cov{\vect{X}}A^{T}=AA^{T}=\Sigma\). This means that
\(\Sigma\) is the covariance matrix of \(A\vect{X}\).
\end{pf}

\begin{remark}
\item \emph{(characterization of correlation matrix)} This result also implies
a characterization of \emph{correlation matrix}, namely that a real symmetric
matrix \(P\) is a correlation matrix iff it is positive semi-definite \vc{with
every diagonal entry being \(1\)}. This is because the correlation matrix of
\((X_1,\dotsc,X_d)\) is the same as the covariance matrix of
\((X_1/\sd{X_1},\dotsc,X_d/\sd{X_d})\), whose diagonal entries are always
\(1\).
\item \emph{(invertibility of covariance matrix)} From linear algebra,
\(\Sigma\in\R^{d\times d}\) is invertible iff it is positive definite (not semi-definite).
Hence, some covariance matrices can be non-invertible and we need positive
definiteness to ensure the invertibility.
\end{remark}
\end{enumerate}
\subsection{The Lebesgue-Radon-Nikodym Theorem}
\label{subsect:leb-rn}
\begin{enumerate}
\item In \Cref{subsect:leb-rn}, we will explore results about \emph{changes of
measures}, which are applied in \emph{importance sampling} in statistics (see
\labelcref{it:rn-deriv-prop}) and \emph{risk-neutral pricing} in financial
economics. To start with, we shall study the three types of relationships
between different measures, namely \emph{dominating}, \emph{equivalent}, and
\emph{singular}.
\item \textbf{Dominating, equivalent, and singular.} Let \(\mu\) and \(\nu\) be
measures on a measurable space \((\Omega,\mathcal{F})\). Then:
\begin{itemize}
\item \(\mu\) \defn{dominates} \(\nu\) (or \(\nu\) is \defn{absolutely
continuous} with respect to \(\mu\)), denoted by \(\nu\ll\mu\), if
\(\mu(A)=0\implies \nu(A)=0\) for all \(A\in\mathcal{F}\).
\item \(\mu\) and \(\nu\) are \defn{equivalent} if \(\nu\ll\mu\) and
\(\mu\ll\nu\), i.e., \(\mu(A)=0\iff \nu(A)=0\) for all \(A\in\mathcal{F}\)
\emph{(sharing the same null sets)}
\item \(\mu\) and \(\nu\) are \defn{singular}, denoted by \(\mu\perp\nu\), if
there exists \(A\in\mathcal{F}\) such that \(\mu(A)=0\) \emph{(\(\mu\) lives on
\(A^{c}\))} and \(\nu(A^{c})=0\) \emph{(\(\nu\) lives on \(A\))}.

\begin{note}
The singularity \(\mu\perp\nu\) implies that for every \(B\in\mathcal{F}\) with
\(\mu(B)>0\) (from here we deduce that \(B\subseteq A^c\)), we have
\(\nu(B)=0\) (since \(0\le\nu(B)\le\nu(A^c)=0\)), and vice versa. Symbolically,
we can write \(\mu(B)>0\implies \nu(B)=0\) and \(\nu(B)>0\implies \mu(B)=0\) for
all \(B\in\mathcal{F}\).
\end{note}
\end{itemize}
\item \textbf{Some lemmas.} To prove the \emph{Lebesgue-Radon-Nikodym theorem},
the following lemmas are needed.
\begin{lemma}[Preservation of domination/singularity upon summation]
\label{lma:meas-sum-dom-sing}
Let \(\nu_n\) be a measure on a measurable space \((\Omega,\mathcal{F})\) for
each \(n\in\N\). For the measure \(\nu:=\sum_{n=1}^{\infty}\nu_n\) on
\((\Omega,\mathcal{F})\) (check that it is indeed a measure!), we have:
\begin{enumerate}
\item \emph{(preserving domination)} If \(\nu_n\ll \mu\) for every \(n\in\N\), then \(\nu\ll\mu\).
\item \emph{(preserving singularity)} If \(\nu_n\perp\mu\) for every
\(n\in\N\), then \(\nu\perp\mu\).
\end{enumerate}
\end{lemma}
\begin{pf}
\begin{enumerate}
\item For every \(A\in\mathcal{F}\) with \(\mu(A)=0\), we have \(\nu(A)=\sum_{n=1}^{\infty}\mu_n(A)=
\sum_{n=1}^{\infty}0=0\) since \(\nu_n\ll\mu\) for every \(n\in\N\). Thus, \(\nu\ll\mu\).
\item For each \(n\in\N\), since \(\nu_n\perp\mu\), there exists
\(A_n\in\mathcal{F}\) such that \(\nu_n(A_n)=0\) and \(\mu(A_n^{c})=0\). Now
let \(A:=\bigcap_{n=1}^{\infty}A_n\in\mathcal{F}\). Then, we have
\[
0\le\nu(A)=\sum_{n=1}^{\infty}\nu_n(A)\overset{\text{(monotonicity)}}{\le}
\sum_{n=1}^{\infty}\nu_n(A_n)=0
\]
and
\[
0\le\mu(A^{c})\underset{(A^c=\bigcup_{n=1}^{\infty}A_n^{c})}{\overset{\text{(\(\sigma\)-subadditivity)}}{\le}}
\sum_{n=1}^{\infty}\mu(A_n^c)=0.
\]
Hence, \(\nu\perp\mu\).
\end{enumerate}
\end{pf}
\begin{lemma}[Relationship between finite measures]
\label{lma:fin-meas-relate}
Let \(\mu\) and \(\nu\) be \vc{finite} measures on \((\Omega,\mathcal{F})\).
Then we have either \(\mu\perp\nu\) or \(\nu|_{A}\ge\varepsilon\mu|_{A}\) for
some \(\varepsilon>0\) and \(A\in\mathcal{F}\) where \(\mu(A)>0\).
\end{lemma}
\begin{pf}
It follows from applying the \emph{Hahn's decomposition theorem}
\parencite[Theorem~3.3]{folland1999real} on the \emph{signed measure}
\(\nu-(1/n)\mu\). For more details, see \textcite[Lemma~3.7]{folland1999real}.
\end{pf}
\item \textbf{The Lebesgue-Radon-Nikodym theorem.}
\begin{theorem}[Lebesgue-Radon-Nikodym theorem]
\label{thm:leb-rn}
Let \(\mu\) and \(\nu\) be \(\sigma\)-finite measures on a measurable space
\((\Omega,\mathcal{F})\). Then:
\begin{enumerate}
\item \emph{(Lebesgue decomposition)} There exists unique \(\sigma\)-finite
measures \(\nu_a\) and \(\nu_s\) on \((\Omega,\mathcal{F})\) such that
\(\nu=\nu_a+\nu_s\) where \(\nu_a\ll\mu\) \emph{(\underline{a}bsolutely
continuous with respect to \(\mu\))} and \(\nu_s\perp\mu\) \emph{(\underline{s}ingular with \(\mu\))}.
\item \emph{(Radon-Nikodym theorem)} There exists a \(\mu\)-a.e.\ unique and
integrable (wrt \(\mu\)) \(f:\Omega\to[0,\infty)\) such that
\(\nu_a(A)=\int_{A}^{}f\odif{\mu}\) for all \(A\in\mathcal{F}\).

\begin{remark}
\item \emph{(meaning of a.e.\ unique)} Here ``\(\mu\)-a.e.\ unique'' suggests
that if there is another function \(g\) satisfying such conditions, then
we have \(f=g\) \(\mu\)-a.e.
\item \emph{(terminologies)} In such case, we say that \(\nu_a\) has
\defn{density} \(f\) with respect to \(\mu\), denoted by
\(\odif{\nu_a}=f\odif{\mu}\) or \(f=\odv{\nu_a}{\mu}\) (the notations are
inspired by the formula
\(\nu_a(A)=\vc{\int_{A}^{}\odif{\nu_a}=\int_{A}^{}f\odif{\mu}}\)).  Such
density \(f\) is also known as the \defn{Radon-Nikodym derivative} of \(\nu_a\)
with respect to \(\mu\) (unique a.e.).
\end{remark}
\end{enumerate}
\end{theorem}
\begin{pf}
We will prove both results together.
\subsubsection*{Step 1}
We start by proving the special case where \(\mu\) and \(\nu\) are
\emph{finite} (so \(\mu(\Omega)<\infty\) and \(\nu(\Omega)<\infty\)).  The
first step is to consider the collection
\(\mathcal{A}:=\{\vc{g:\Omega\to[0,\infty]}: \text{\(g\) is integrable wrt
\(\mu\), \(\int_{A}^{}g\odif{\mu}\le\nu(A)\) for all \(A\in\mathcal{F}\)}\}\).

\textbf{Showing that \(\mathcal{A}\) is nonempty and closed under maxima (of finitely many elements).}
\begin{itemize}
\item \emph{Nonempty:} Since \(\mathcal{A}\) contains the zero function, it is nonempty.
\item \emph{Closed under maxima:} Fix any \(g_1,g_2\in\mathcal{A}\). We
want to show that \(\max\{g_1,g_2\}\in\mathcal{A}\) (which would then imply by
induction that \(g_1,\dotsc,g_n\in\mathcal{A}\implies \max\{g_1,\dotsc,g_n\}\in\mathcal{A}\)).

Let \(B:=\{\omega\in\Omega:g_1(\omega)<g_2(\omega)\}\in\mathcal{F}\). Since
\begin{align*}
\int_{A}^{}\max\{g_1,g_2\}\odif{\mu}
&=\int_{A}^{}\max\{g_1,g_2\}\indic_{B}\odif{\mu}+\int_{A}^{}\max\{g_1,g_2\}\indic_{B^c}\odif{\mu} \\
&=\int_{A\cap B}^{}g_2\odif{\mu}+\int_{A\cap B^c}^{}g_1\odif{\mu} \\
\overset{(g_1,g_2\in\mathcal{A})}&{\le}\nu(A\cap B)+\nu(A\cap B^c)
=\nu(A),
\end{align*}
we have \(\max\{g_1,g_2\}\in\mathcal{A}\).
\end{itemize}
\textbf{Constructing the density \(f\).}
Let \(s:=\sup_{g\in\mathcal{A}}\int_{\Omega}^{}g\odif{\mu}\). For every
\(g\in\mathcal{A}\), we have
\(\int_{\vc{\Omega}}^{}g\odif{\mu}\le\nu(\vc{\Omega})\), so \(s\le \nu(\Omega)<\infty\).
By definition of supremum, there exists a sequence \(\{g_n\}\subseteq \mathcal{A}\) such that
\(\int_{\Omega}^{}g_n\odif{\mu}\to s\) as \(n\to\infty\).

Now, for each \(n\in\N\), let
\(f_n:=\max\{g_1,\dotsc,g_n\}\overset{\text{(closed under
maxima)}}{\in}\in \mathcal{A}\), and let \(f:=\sup_{n\in\N}g_n\). Then by
construction we have \(f_n\nearrow f\) pointwisely. Since each \(g_n\) is
nonnegative, we have \(f_n\in L_{+}\) for every \(n\in\N\). Thus, by MCT,
\(\int_{\Omega}^{}f\odif{\mu}=\lim_{n\to\infty}\int_{\Omega}^{}f_n\odif{\mu}\)
and \(f\) is integrable since \(s<\infty\).

Moreover, we can establish the following properties of \(f\):
\begin{itemize}
\item \underline{\(f\in \mathcal{A}\)}: Fix any \(A\in\mathcal{F}\). Since
\(f_n\in\mathcal{A}~\forall n\in\N\), we have
\(\int_{A}^{}f_n\odif{\mu}\le\nu(A)~\forall n\in\N\). Thus,
\[
\int_{A}^{}f\odif{\mu}=\int_{\Omega}^{}f\indic_{A}\odif{\mu}
\overset{\text{(MCT)}}{=}\lim_{n\to\infty}\int_{\Omega}^{}f_n\indic_{A}\odif{\mu}
=\lim_{n\to\infty}\int_{A}^{}f_n\odif{\mu}\le\nu(A),
\]
which means that \(f\in \mathcal{A}\).
\item \underline{\(\int_{\Omega}^{}f\odif{\mu}=s\)}: Note that
\[
\int_{\Omega}^{}f\odif{\mu}=\lim_{n\to\infty}\int_{\Omega}^{}f_n\odif{\mu}\overset{(f_n\ge
g_n)}{\ge}\lim_{n\to\infty}\int_{\Omega}^{}g_n\odif{\mu}=s
\]
and
\[
\int_{\Omega}^{}f\odif{\mu}=\lim_{n\to\infty}\int_{\Omega}^{}f_n\odif{\mu}
\underset{(\int_{\Omega}^{}f_n\odif{\mu}\le s~\forall n\in\N)}
{\overset{(f_n\in\mathcal{A}~\forall n\in\N)}{\le}}s.
\]
Thus, \(\int_{\Omega}^{}f\odif{\mu}=s\).
\item \underline{\(f:\Omega\to[0,\infty)\)}: Since \(f_n\nearrow f\)
pointwisely, we already know that \(f\) is a function from \(\Omega\) to
\([0,\infty]\).  Also, \(f\) is integrable, so we have \(f<\infty\) a.e.\ by
\labelcref{it:leb-int-ae-fin}. Hence by redefining \(f\) to take nonnegative
real values (say \(0\)) on a null set (where \(f\) takes the value \(\infty\) originally)
if needed, we may assume that \(f\) is a function from \(\Omega\) to
\([0,\infty)\). Due to this redefinition, such density \(f\) can at most be
\(\mu\)-a.e.\ unique (we will show that \(f\) is indeed a.e.\ unique below).
\end{itemize}
\textbf{Showing the existence of such \(\sigma\)-finite \(\nu_a\) and \(\nu_s\).}
We define \(\nu_a(A):=\int_{A}^{}f\odif{\mu}~\forall A\in\mathcal{F}\) and
\(\nu_s(A):=\nu(A)-\nu_a(A)\overset{(f\in \mathcal{A})}{\ge}0\). By
\labelcref{it:nonneg-leb-int-prop}, \(\nu_a\) is a measure on
\((\Omega,\mathcal{F})\). It is also straightforward to show that \(\nu_s\) is
a measure on \((\Omega,\mathcal{F})\). Regarding their \(\sigma\)-finiteness, consider:
\begin{itemize}
\item \(\nu_a\): Since \(f\) is integrable, we have
\(\nu_a(\Omega)=\int_{\Omega}^{}f\odif{\mu}<\infty\), so \(\nu_a\) is
finite, hence \(\sigma\)-finite.
\item \(\nu_s\): Since \(\nu(\Omega)<\infty\) and \(\nu_a(\Omega)\), we have
\(\mu_s(\Omega)<\infty\), and so \(\nu_s\) is again finite, hence \(\sigma\)-finite.
\end{itemize}
It then remains to show that \(\nu_a\ll\mu\) and \(\nu_s\perp\mu\):
\begin{itemize}
\item \underline{\(\nu_a\ll\mu\)}: For every \(A\in\mathcal{F}\) with \(\mu(A)=0\),
by \labelcref{it:quasi-int-prop} we know \(\nu_a(A)=0\). Thus, \(\nu_a\ll\mu\).
\item \underline{\(\nu_s\perp\mu\)}: Assume to the contrary that
\(\nu_s\not\perp\mu\). Then by \Cref{lma:fin-meas-relate}, there must exist
\(\varepsilon>0\) and \(A\in\mathcal{F}\) with \(\mu(A)>0\) such that
\(\nu_s|_{A}\ge\varepsilon\mu|_{A}\). This implies that for all \(B\in\mathcal{F}\) we have
\[
\int_{B}^{}\varepsilon\indic_{A}\odif{\mu}=\varepsilon\mu(\underbrace{A\cap B}_{\subseteq A})
=\varepsilon\mu|_{A}(A\cap B)
\le\nu_s(A\cap B)\le\overset{\text{(monotonicity)}}{\le}\nu_s(B)=\nu(B)-\int_{B}^{}f\odif{\mu}.
\]
Rearranging this inequality then gives
\(\int_{B}^{}(f+\varepsilon\indic_{A})\odif{\mu}\le\mu(B)\), which implies that
\(f+\varepsilon\indic_{A}\in\mathcal{A}\), and therefore
\(\int_{\Omega}^{}f+\varepsilon\indic_{A}\odif{\mu}\le s\).

But on the other hand, we have \(\int_{\Omega}^{}f\odif{\mu}=s\) as shown
previously, so
\(\int_{\Omega}^{}f+\varepsilon\indic_{A}\odif{\mu}=s+\varepsilon\mu(A)>s\),
contradiction.
\end{itemize}
\textbf{Showing the uniqueness of \(\nu_a\) and \(\nu_s\), and the a.e.\ uniqueness of \(f\).}
Suppose that we have \(\nu=\nu_a'+\nu_s'\) with \(\nu_a'(A)=\int_{A}^{}f'\odif{\mu
}~\forall A\in\mathcal{F}\), where \(\nu_a',\nu_s',f'\) satisfy the conditions
mentioned in the result.

Then, rearranging \(\nu_a+\nu_s=\nu=\nu_a'+\nu_s\) gives
\(\nu_s-\nu_s'=\nu_a'-\nu_a\), so we can write
\(\vc{\nu_s(B)=\nu_s'(B)}=\int_{B}^{}f'-f\odif{\mu}=\orc{\nu_a'(B)-\nu_a(B)}\)
for all \(B\in\mathcal{F}\). Next, consider two cases:
\begin{itemize}
\item \emph{Case 1: \(\mu(B)>0\).} Since \(\nu_s\perp\mu\) and
\(\nu_s'\perp\mu\), we have \(\vc{\nu_s(B)-\nu_s'(B)}=0-0=0\), so
\(\orc{\nu_a'(B)-\nu_a(B)}=0\) also.
\item \emph{Case 2: \(\mu(B)=0\).} By \labelcref{it:quasi-int-prop}, we know
\(\int_{B}^{}f'-f\odif{\mu}=0\), so \(\vc{\nu_s(B)-\nu_s'(B)}=\orc{\nu_a'(B)-\nu_a(B)}=0\).
\end{itemize}
Therefore, we have \(\vc{\nu_s(B)=\nu_s'(B)}\) and \(\orc{\nu_a'(B)=\nu_a(B)}\)
for all \(B\in\mathcal{F}\), establishing the uniqueness of \(\nu_a\) and
\(\nu_s\).  Furthermore, since \(\int_{B}^{}f'-f\odif{\mu}=0\) for all
\(B\in\mathcal{F}\), by \labelcref{it:equiv-ae-equal-int-fn} we know \(f'=f\)
\(\mu\)-a.e., establishing the a.e.\ uniqueness of \(f\).

\subsubsection*{Step 2}
We have completed the proof for the case where \(\mu\) and \(\nu\) are finite.
Next, we are going to extend it to the case where they are \(\sigma\)-finite.

\textbf{Constructing pairwise disjoint \(A_1,A_2,\dotsc\in\mathcal{F}\) that ``simultaneously
satisfy'' the \(\sigma\)-additivity of \(\mu\) and \(\nu\).}
Our goal here is to construct pairwise disjoint
\(A_1,A_2,\dotsc\in\mathcal{F}\) such that
\(\biguplus_{n=1}^{\infty}A_n=\Omega\) with \(\mu(A_n)<\infty\) and
\(\nu(A_n)<\infty\) for each \(n\in\N\). This can be done by intersecting the
sets arising respectively from the \(\sigma\)-additivity of \(\mu\) and \(\nu\).

Due to the \(\sigma\)-additivity of \(\mu\) and \(\nu\), there exist
\(A_{\mu,1},A_{\mu,2},\dotsc\in\mathcal{F}\) and \(A_{\nu,1},A_{\nu,2},\dotsc\in\mathcal{F}\)
such that \(\bigcup_{m=1}^{\infty}A_{\mu,m}=\Omega\) and
\(\bigcup_{\ell=1}^{\infty}A_{\nu,\ell}=\Omega\) with \(\mu(A_{\mu,m})<\infty\) and
\(\nu(A_{\nu,\ell})<\infty\) for all \(m,n\in\N\). WLOG (by performing
\emph{disjointification} on them if needed), we may assume that
\(A_{\mu,1},A_{\mu,2},\dotsc\in\mathcal{F}\) and \(A_{\nu,1},A_{\nu,2},\dotsc\in\mathcal{F}\)
are respectively pairwise disjoint.

Now, consider
\(\Omega=\Omega\cap\Omega=(\biguplus_{m=1}^{\infty}A_{\mu,m})\cap(\biguplus_{\ell=1}^{\infty}A_{\nu,\ell})=\biguplus_{m=1}^{\infty}\biguplus_{\ell=1}^{\infty}(A_{\mu,m}\cap
A_{\nu,\ell})\). Note that there are countably many \((A_{\mu,m}\cap
A_{\nu,\ell})\)'s, so after relabelling them to \(A_1,A_2,\dotsc\in\mathcal{F}\), we
have constructed the desired sets.

\textbf{Constructing finite measures and applying the result from step 1 to them.}
Fix any \(n\in\N\). Define \(\mu_n(B):=\mu(B\cap A_n)\) and
\(\nu_n(B):=\nu(B\cap A_n)\) for all \(B\in\mathcal{F}\). By construction of
\(A_n\)'s, \(\mu_n\) and \(\nu_n\) are both finite. Hence, applying the result
from step 1 to them gives
\[
\nu_n(B)=\nu_{n,a}(B)+\nu_{n,s}(B)=\int_{B}^{}f_n\odif{\mu_n}+\nu_{n,s}(B)
\quad\forall B\in\mathcal{F},
\]
where \(\nu_{n,a},\nu_{n,s},f_n\) satisfy the conditions mentioned in the result,
and particularly we know from step 1 that \(\nu_{n,a}\) and \(\nu_{n,s}\) are
indeed \emph{finite}.

\textbf{Showing the existence of such \(\nu_a\), \(\nu_s\), and \(f\).} By
construction of \(\mu_n\) and \(\nu_n\), we have
\(\mu_n(A_n^c)=\nu_n(A_n^c)=0\). So, we have
\(\int_{A_n^c}^{}f_n\odif{\mu_n}=0\) by \labelcref{it:quasi-int-prop}, and hence
\(\nu_{n,s}(A_n^c)=\nu_n(A_n^c)-\int_{A_n^c}^{}f_n\odif{\mu_n}=0-0=0\).
Furthermore, we may assume that \(f_n|_{A_n^c}=0\) (after redefinition if
needed) without affecting the \(\mu_n\)-a.e.\ uniqueness, since
\(\mu_n(A_n^c)=0\).

Now, let \(\nu_a(B):=\int_{B}^{}f\odif{\mu}~\forall B\in\mathcal{F}\) with
\(f:=\sum_{n=1}^{\infty}f_n\), after redefining on a null set where it is
originally \(\infty\) if needed), and let
\(\nu_s:=\sum_{n=1}^{\infty}\nu_{n,s}\).  We then establish the target
properties as follows:
\begin{itemize}
\item \emph{\(f\) is integrable:} TODO

Similar to step 1, we can show that \(f\) is a function from \(\Omega\) to
\([0,\infty)\) by redefining on a null set where it originally takes the value
\(\infty\) if needed.
\item \emph{\(\nu_a\ll\mu\) and \(\nu_s\perp\mu\):} By
\Cref{lma:meas-sum-dom-sing}, since \(\nu_{n,s}\perp\mu~\forall n\in\N\), we
have \(\nu_s\perp\mu\). Also, we know by \labelcref{it:quasi-int-prop} that
\(\nu_a\ll\mu\).
\item \emph{\(\nu_a\) and \(\nu_s\) are \(\sigma\)-finite:} TODO
\item \emph{\(\nu(B)=\nu_a(B)+\nu_s(B)~\forall B\in\mathcal{F}\):} Consider
\begin{align*}
\nu(B)&=\nu\left(B\cup\biguplus_{n=1}^{\infty}A_n\right)
=\nu\left(\biguplus_{n=1}^{\infty}B\cap A_n\right)
=\sum_{n=1}^{\infty}\nu(B\cap A_n)
=\sum_{n=1}^{\infty}\nu_n(B) \\
&=\sum_{n=1}^{\infty}\left(\int_{B}^{}f_n\odif{\mu_n}+\nu_{n,s}(B)\right)
\underset{(f_{n}|_{A_n^c}=0)}{\overset{(\mu_n|_{A_n}=\mu|_{A_{n}})}{=}}
\sum_{n=1}^{\infty}\int_{B}^{}f_n\odif{\mu}+\nu_{s}(B) \\
\overset{\text{(\Cref{cor:leb-int-sum-commu})}}&{=}
\int_{B}^{}\sum_{n=1}^{\infty}f_n\odif{\mu}+\nu_{s}(B)
=\int_{B}^{}f\odif{\mu}+\nu_{s}(B)
=\nu_a(B)+\nu_s(B).
\end{align*}
\end{itemize}
\textbf{Showing the uniqueness of \(\nu_a\) and \(\nu_s\), and the a.e.\ uniqueness of \(f\).}
The argument is exactly the same as the one in step 1.
\end{pf}

Sometimes the following result, which is a corollary of the
Lebesgue-Radon-Nikodym theorem, is referred to as the \emph{Radon-Nikodym
theorem}.  For studying changes of measures, quite often this special case is
utilized instead of the more general Lebesgue-Radon-Nikodym theorem.
\begin{corollary}[Radon-Nikodym theorem]
\label{cor:rn-thm}
If \(\mu\) and \(\nu\) are \(\sigma\)-finite measures on a measurable space
\((\Omega,\mathcal{F})\) with \(\nu\ll\mu\), then there exists a \(\mu\)-a.e.\
unique and integrable (wrt \(\mu\)) \(f:\Omega\to[0,\infty)\) such that
\(\nu(A)=\int_{A}^{}f\odif{\mu}\) for all \(A\in\mathcal{F}\) (which is the
Radon-Nikodym derivative \(\odv{\nu}{\mu}\)).
\end{corollary}
\begin{pf}
With \(\nu\ll\mu\), we know from the Lebesgue-Radon-Nikodym theorem that the
Lebesgue decomposition must be given by \(\nu=\nu_a\) (\(\nu_s=0\)), and so the
result follows.
\end{pf}
\item\label{it:rn-deriv-prop} \textbf{Properties of Radon-Nikodym derivatives.}
Here, we are going to show some properties of Radon-Nikodym derivatives, which
are quite consistent with the notation \(\odv{\nu}{\mu}\).

Let \(\mu\), \(\nu\), \(\lambda\) be \(\sigma\)-finite measures on
\((\Omega,\mathcal{F})\) with \(\nu\ll\mu\) and \(\mu\ll\lambda\). Then:
\begin{enumerate}
\item\label{it:rn-deriv-int-sub} \emph{(``integration by substitution'')} If \(g\in
L^{1}(\Omega,\mathcal{F},\nu)\), then \(g\odv{\nu}{\mu}\in
L^{1}(\Omega,\mathcal{F},\mu)\) and
\(\int_{\Omega}^{}g\odif{\nu}=\int_{\Omega}^{}g\odv{\nu}{\mu}\odif{\mu}\).
\item\label{it:rn-deriv-chain-rule} \emph{(``chain rule'')} \(\nu\ll\lambda\) and
\(\odv{\nu}{\vc{\lambda}}=\odv{\nu}{\mu}\odv{\mu}{\vc{\lambda}}\) \(\vc{\lambda}\)-a.e.
\end{enumerate}
\begin{pf}
\begin{enumerate}
\item We apply the standard argument on \(g\).
\begin{enumerate}[label={(\arabic*)}]
\item Fix any indicator function \(g=\indic_{A}\) where \(A\in\mathcal{F}\). We
then have
\[
\int_{\Omega}^{}g\odif{\nu}\overset{\text{(simple)}}{=}\nu(A)\underset{(\nu\ll\mu)}{\overset{\text{(Radon-Nikodym)}}{=}}
=\int_{A}^{}\odv{\nu}{\mu}\odif{\mu} =\int_{\Omega}^{}g\odv{\nu}{\mu}\odif{\mu}
\]
By the linearity of integral, the equation also holds for all simple functions.
\item Fix any \(g\in L_{+}(\Omega,\mathcal{F},\nu)\). By \Cref{lma:apx-seq}, there exists a sequence
\(\{g_n\}\) of nonnegative simple functions on \(\Omega'\) such that
\(g_n\nearrow g\), which implies that \(g_n\odv{\nu}{\mu}\nearrow g\odv{\nu}{\mu}\)
with \(g_n\odv{\nu}{\mu}\in L_{+}\) for each \(n\in\N\). Hence,
\[
\int_{\Omega}^{}g\odif{\nu}\overset{\text{(MCT)}}{=}
\lim_{n\to\infty}\int_{\Omega}^{}g_n\odif{\nu}
\overset{(1)}{=}\lim_{n\to\infty}\int_{\Omega}^{}g_n\odv{\nu}{\mu}\odif{\mu}
\overset{\text{(MCT)}}{=}\int_{\Omega}^{}g\odv{\nu}{\mu}\odif{\mu}.
\]
\item Fix any \(g\in L^{1}(\Omega,\mathcal{F},\nu)\). Then we have
\[
\int_{\Omega}^{}g\odif{\nu}=\int_{\Omega}^{}g^{+}\odif{\nu}-\int_{\Omega}^{}g^{-}\odif{\nu}
\overset{(2)}{=}\int_{\Omega}^{}g^{+}\odv{\nu}{\mu}\odif{\mu}
-\int_{\Omega}^{}g^{-}\odv{\nu}{\mu}\odif{\mu}
=\int_{\Omega}^{}g\odv{\nu}{\mu}\odif{\mu}.
\]
This also implies that \(g\odv{\nu}{\mu}\in L^{1}(\Omega,\mathcal{F},\mu)\) as
\(\int_{\Omega}^{}g^{+}\odv{\nu}{\mu}\odif{\mu}=\int_{\Omega}^{}g^{+}\odif{\nu}<\infty\)
and \(\int_{\Omega}^{}g^{-}\odv{\nu}{\mu}\odif{\mu}=\int_{\Omega}^{}g^{-}\odif{\nu}<\infty\).
\end{enumerate}
\item For all \(A\in\mathcal{F}\), we have
\(\lambda(A)=0\overset{(\mu\ll\lambda)}{\implies }\mu(A)=0
\overset{(\nu\ll\lambda)}{\implies }\nu(A)=0\), and hence \(\nu\ll\lambda\).

Next, we have
\[
\int_{A}^{}\odv{\nu}{\lambda}\odif{\lambda}
\overset{\text{(Radon-Nikodym)}}{=}\nu(A)
\overset{\text{(Radon-Nikodym)}}{=}\int_{A}^{}\odv{\nu}{\mu}\odif{\mu}
=\int_{\Omega}^{}\vc{\indic_{A}\odv{\nu}{\mu}}\odif{\orc{\mu}}
\overset{\text{(a)}}{=}\int_{\Omega}^{}\vc{\indic_{A}\odv{\nu}{\mu}}\odv{\orc{\mu}}{\lambda}\odif{\mu}
=\int_{A}^{}\odv{\nu}{\mu}\odv{\mu}{\lambda}\odif{\lambda}
\]
for all \(A\in\mathcal{F}\), so the result follows by
\labelcref{it:equiv-ae-equal-int-fn}.
\end{enumerate}
\end{pf}

\begin{remark}
\item \emph{(relationship between Radon-Nikodym derivative and density function)}
Consider a probability space \((\R^d,\mathcal{B}(\R^d),\pr_{\vect{X}})\). If
the distribution \(\pr_{\vect{X}}\) or \(F\) is absolutely continuous with
respect to the Lebesgue measure \(\lambda\), then by Radon-Nikodym theorem we have
\(F(B)=\int_{B}^{}\odif{F}=\int_{B}^{}f\odif{\lambda}\) for all \(B\in\mathcal{B}(\R^d)\),
where \(f=\odv{F}{\lambda}\) is the Radon-Nikodym derivative.

Particularly, setting \(B=(\vect{-\infty},\vect{x}]\) gives
\(F(\vect{x})=F((\vect{-\infty},\vect{x}]
=\int_{(\vect{-\infty},\vect{x}]}^{}f\odif{\lambda}
=\int_{(\vect{-\infty},\vect{x}]}^{}f(\widetilde{\vect{x}})\odif{\lambda(\widetilde{\vect{x}})}
\overset{\text{(notation)}}{=}\int_{(\vect{-\infty},\vect{x}]}^{}f(\widetilde{\vect{x}})\odif{\widetilde{\vect{x}}}\),
which suggests that the Radon-Nikodym derivative is indeed the density function
of \(F\). In such case, recall that \(F\) is said to be \emph{absolutely
continuous} --- this relates with the terminology here. Indeed, a distribution
function \(F\) is absolutely continuous iff its corresponding distribution
(also denoted by \(F\)) is absolutely continuous wrt the Lebesgue measure \(\lambda\).

Applying \labelcref{it:rn-deriv-int-sub}, we also obtain
\(\int_{\R^d}^{}g(\vect{x})\odif{F(\vect{x})}=\int_{\R^d}^{}g(\vect{x})f(\vect{x})\odif{\vect{x}}\),
if \(g\in L^{1}(\R^d,\mathcal{B}(\R^d),F)\).
\item \emph{(importance sampling)}
\labelcref{it:rn-deriv-int-sub} can be expressed more succinctly as
``\(\expvmu{\nu}{g}=\expvmu{\mu}{g\odv{\nu}{\mu}}\) if \(\nu\ll\mu\)''.  This
formula is applied in \emph{importance sampling} in statistics. To see this,
consider \(\vect{X}\sim F\) and \(\widetilde{\vect{X}}\sim H\), with \(f\) and
\(h\) being the densities of \(F\) and \(H\) (interpreted as distribution
functions/distributions) respectively. Then the formula suggests that, for
every density \(h\) satisfying that \(h(\vect{x})=0\implies f(\vect{x})=0\),
\[
\expvmu{F}{g(\vect{X})}=\int_{\R^d}^{}g(\vect{x})f(\vect{x})\odif{\vect{x}}
=\int_{\R^d}^{}\underbrace{\frac{g(\vect{x})f(\vect{x})}{h(\vect{x})}h(\vect{x})}
_{:=0\text{ if \(h(\vect{x})=0\)}}
\odif{\vect{x}}
=\expvmu{H}{\frac{g(\widetilde{\vect{X}})f(\widetilde{\vect{X}})}{h(\widetilde{\vect{X}})}},
\]
which is utilized in importance sampling.
\end{remark}
\end{enumerate}
