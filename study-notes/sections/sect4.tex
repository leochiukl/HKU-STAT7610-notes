\section{Ordinary Conditional Probability, Independence, and Dependence}
\label{sect:cond-prob-indep-dep}
\begin{enumerate}
\item In \Cref{sect:cond-prob-indep-dep}, we will delve into notions that are
\emph{exclusive} to probability theory, namely ordinary conditional
probability, independence, and dependence. The availability of these concepts
is a main difference that separates the closely related \emph{measure theory}
and \emph{probability theory}.
\end{enumerate}
\subsection{Ordinary Conditional Probability}
\label{subsect:ordinary-cond-prob}
\begin{enumerate}
\item The \emph{ordinary conditional probability} is indeed just the
conditional probability you have seen in your first probability course. We add
the term ``ordinary'' here since we are going to introduce a much more general
and abstract \faIcon[regular]{dizzy} concept of conditioning (see
\Cref{sect:cond-exp}).

Let \((\Omega,\mathcal{F},\pr)\) be a probability space, \(B\in\mathcal{F}\) be
an event with \rc{\(\prob{B}>0\)}. Then, the \defn{(ordinary) conditional
probability of \(A\) given \(B\)}, denoted by \(\prob{A|B}\), is given by
\(\prob{A|B}=\prob{A\cap B}/\prob{B}\) for all \(A\in\mathcal{F}\).
\begin{note}
The requirement that \(\prob{B}>0\) is indeed somewhat restrictive, e.g., it
would not permit us to condition on \(\{X=x\}\) where \(X\) is an absolutely
continuous random variable, since \(\prob{X=x}=0\) always. However, often one
wants to do such kind of conditioning. This issue leads to the development of
the more general and abstract concept of conditioning mentioned above.
\end{note}

Let us verify that \(\prob{\cdot|B}\) is indeed a probability measure on
\(\mathcal{F}\) in the following:

\begin{pf}
\begin{enumerate}[label={(\arabic*)}]
\item The nonnegativity follows from the observation that both \(\prob{A\cap
B}\) and \(\prob{B}\) are always nonnegative (the latter is always positive
indeed).
\item We have \(\prob{\Omega|B}=\prob{\Omega\cap B}/\prob{B}=\prob{B}/\prob{B}=1\).
\item Fix any pairwise disjoint \(A_1,A_2,\dotsc\in\mathcal{F}\). Then,
\begin{align*}
\prob{\left.\biguplus_{i=1}^{\infty}A_i\right|B}
&=\frac{\prob{\left(\biguplus_{i=1}^{\infty}A_i\right)\cap B}}{\prob{B}}
=\frac{\prob{\biguplus_{i=1}^{\infty}(A_i\cap B)}}{\prob{B}}
=\frac{\sum_{i=1}^{\infty}\prob{A_i\cap B}}{\prob{B}} \\
&=\sum_{i=1}^{\infty}\frac{\prob{A_i\cap B}}{\prob{B}}
=\sum_{i=1}^{\infty}\prob{A_i|B}.
\end{align*}
\end{enumerate}
\end{pf}
\item \textbf{What would happen if \(\prob{B}=0\)?} From the definition of
the ordinary conditional probability above, it is clear that \(\prob{A\cap
B}/\prob{B}\) is \emph{undefined} if \(\prob{B}=0\). Consequently, we must not
define the conditional probability \(\prob{A|B}\) using the ratio formula above.
In this case, we would not define \(\prob{A|B}\) using a formulaic approach.
Rather, we would just require it to satisfy \(\prob{A|B}\prob{B}=\prob{A\cap
B}\) without actually providing an explicit definition of \(\prob{A|B}\).

With this equality, we can then obtain that \(\prob{A|B}\prob{B}=0\).

\begin{pf}
Note that
\(\prob{A|B}\prob{B}=\prob{A\cap B}\overset{\text{(monotonicity)}}{\le}\prob{B}=0\) and
\(\prob{A|B}\prob{B}=\prob{A\cap B}\ge 0\).
\end{pf}

Of course, merely having the equality \(\prob{A|B}\prob{B}=\prob{A\cap B}\)
still leaves a lot of room for discussions on what conditioning on such
zero-probability event \(B\) means.  To deal with this somewhat complicated
issue of ``conditioning on zero-probability events'', it is standard to utilize
the machinery from \Cref{sect:cond-exp}.

\item \textbf{Two main results about ordinary conditional probability.} You
should have already learnt the following two properties about conditional
probability in your first probability course, and they are stated and proved
here again for reference. The first one is the \emph{law of total probability},
which may be seen as a probability version of the \emph{law of total measure}.
\begin{theorem}[Law of total probability]
\label{thm:law-of-total-prob}
Let \((\Omega,\mathcal{F},\pr)\) be a probability space, and
\(\{B_1,B_2,\dotsc\}\subseteq \mathcal{F}\) be a partition of \(\Omega\). Then
\[
\prob{A}=\sum_{i=1}^{\infty}\prob{A|B_i}\prob{B_i}.
\]
\begin{note}
We have \(\prob{A|B_i}\prob{B_i}=0\) for every \(i\) where \(\prob{B_i}=0\).
\end{note}
\end{theorem}
\begin{pf}
By the law of total measure (\labelcref{it:meas-prop}), we have
\[
\prob{A}=\sum_{i=1}^{\infty}\prob{A\cap B_i}=\sum_{i=1}^{\infty}\prob{A|B_i}\prob{B_i}.
\]
\end{pf}

The next one is the \emph{Bayes' theorem}, which illustrates a key idea used in
the \emph{Bayesian methodology} in statistics.

\begin{theorem}[Bayes' theorem]
\label{thm:bayes}
Let \((\Omega,\mathcal{F},\pr)\) be a probability space, and
\(A\in\mathcal{F}\) be an event with \(\prob{A}>0\). Then
\[
\prob{B|A}=\frac{\prob{A|B}\prob{B}}{\prob{A}}
\]
for all \(B\in\mathcal{F}\).
Furthermore, if \(\{B_1,B_2,\dotsc\}\subseteq \mathcal{F}\) is a partition of
\(\Omega\), then we have
\[
\prob{B_i|A}=\frac{\prob{A|B_i}\prob{B_i}}{\sum_{j=1}^{\infty}\prob{A|B_j}\prob{B_j}}.
\]
\end{theorem}
\begin{pf}
The second formula follows from the first by applying the law of total
probability on the denominator \(\prob{A}\):
\(\prob{A}=\sum_{j=1}^{\infty}\prob{A|B_j}\prob{B_j}\), so we will only prove
the first:
\[
\prob{B|A}=\frac{\prob{B\cap A}}{\prob{A}}=\frac{\prob{A\cap B}}{\prob{A}}
=\frac{\prob{A|B}\prob{B}}{\prob{A}}.
\]
\end{pf}
\end{enumerate}
\subsection{Independence}
\label{subsect:independence}
\begin{enumerate}
\item \textbf{Definitions of independence.} The conditional probability
\(\prob{A|B}\) can be viewed as the probability assessment of the event \(A\)
when knowing that the event \(B\) occurs.  But if the event \(A\) does not in
any way depend on the occurrence of event \(B\), the probability assessment
should then be the same with or without the knowledge of the occurrence of
\(B\): \(\prob{A|B}\overset{\text{(should be)}}{=}\prob{A}\), which can be
rewritten as \(\prob{A\cap B}=\prob{A}\prob{B}\). This motivates the definition
of \emph{independence} of two events, and we have more general definitions of
independence in the following.

Let \((\Omega,\mathcal{F},\pr)\) be a probability space, and \(I\subseteq \R\)
be an arbitrary index set. Then:
\begin{enumerate}
\item \emph{(Independence of events)} The events \(A_1,\dotsc,A_n\in\mathcal{F}\) are \defn{independent} if
\(\prob{\bigcap_{i\in I}^{}A_i}=\prod_{i\in I}^{}\prob{A_i}\) for
\underline{every} subset \(I\subseteq \{1,\dotsc,n\}\). \begin{warning} Not
just \(I=\{1,\dotsc,n\}\)!  \end{warning}

\begin{note}
Here we have \(\bigcap_{i\in \varnothing}^{}A_i:=\Omega\) and
\(\prod_{i\in\varnothing}^{}\prob{A_i}:=1\).
\end{note}
\item \emph{(Independence of a collection of events)} A collection
\(\{A_i\}_{i\in I}\subseteq \mathcal{F}\) of events is \defn{independent} if
\(A_{i_1},\dotsc,A_{i_n}\) are independent for all
\(\{i_1,\dotsc,i_n\}\subseteq I\) with \(n\in\N\) being arbitrary.
\item \emph{(Independence of families)} The families
\(\mathcal{A}_1,\dotsc,\mathcal{A}_n\subseteq \mathcal{F}\) of events are
\defn{independent} if \(A_1,\dotsc,A_n\) are independent for all
\(A_1\in\mathcal{A}_1,\dotsc,A_n\in\mathcal{A}_n\).
\item \emph{(Independence of a collection of families)} A collection
\(\{\mathcal{A}_i\}_{i\in I}\) of families of events, with
\(\mathcal{A}_i\subseteq \mathcal{F}\) for all \(i\in I\), is
\defn{independent} if \(\mathcal{A}_{i_1},\dotsc,\mathcal{A}_{i_n}\) are
independent for all \(\{i_1,\dotsc,i_n\}\subseteq I\) with \(n\in\N\) being
arbitrary.
\item \emph{(Independence of measurable functions)} Measurable functions
\(X_1,\dotsc,X_d\), all defined on the probability space
\((\Omega,\mathcal{F},\pr)\), are \defn{independent}, if the families
\(\sigma(X_{i_1}),\dotsc,\sigma(X_{i_d})\) are independent.
\item \emph{(Independence of a collection of measurable functions)} A
collection \(\{X_i\}_{i\in I}\) of measurable functions, all defined on the
probability space \((\Omega,\mathcal{F},\pr)\), is \defn{independent} if the
collection \(\{\sigma(X_i)\}_{i\in I}\) is independent.

Particularly:
\begin{itemize}
\item If \(X_i\) is a random variable for every \(i\in I\), then the collection
is independent if \(X_{i_1}^{-1}(B_{i_1}),\dotsc,X_{i_n}^{-1}(B_{i_n})\) are
independent
\(\forall~B_{i_1},\dotsc,B_{i_n}\in\mathcal{B}(\R),\quad
\forall~\{i_{1},\dotsc,i_{n}\}\subseteq I,\quad
\forall~n\in\N\).
\item If \(X_i:\Omega\to\R^{d_i}\) is a random vector (written as \(\vect{X}_i\) in the
following) for every \(i\in I\), then the collection is independent if
\(\vect{X}_{i_1}^{-1}(B_{i_1}),\dotsc,\vect{X}_{i_n}^{-1}(B_{i_n})\) are independent
\(\forall~B_{i_1}\in\mathcal{B}(\R^{d_1}),\dotsc,B_{i_n}\in\mathcal{B}(\R^{d_n}),\quad
\forall~\{i_{1},\dotsc,i_{n}\}\subseteq I,\quad
\forall~n\in\N\).
\end{itemize}
\end{enumerate}
\item \textbf{Results about independence.} The following results can simplify
the process of verifying independence. Throughout, let
\((\Omega,\mathcal{F},\pr)\) be a probability space.
\begin{enumerate}
\item \label{it:ind-via-gen} \emph{(Independence via generators)} If
\(\{\mathcal{A}_{i}\}_{i\in I}\) is a collection of independent \(\pi\)-systems
on \(\Omega\), then the collection \(\{\sigma(\mathcal{A}_i)\}_{i\in I}\) is
independent.
\item \label{it:ind-comp} \emph{(Independence of complements)} If \(\{A_{i}\}_{i\in I}\subseteq \mathcal{F}\)
is independent, then so is \(\{A_{i}^c\}_{i\in I}\).
\item \label{it:grouping-lma} \emph{(Grouping lemma)} If \(\{\mathcal{A}_{i}\}_{i\in
I}\subseteq \mathcal{F}\) is a collection of independent \(\pi\)-systems, and
\(I=\biguplus_{j\in J}^{}I_j\), then the collection \(\{\sigma(\bigcup_{i\in
I_j}^{}\mathcal{A}_i)\}_{j\in J}\) is independent.
\begin{center}
\begin{tikzpicture}
\draw[fill, violet] (1,0) circle [radius=1mm];
\draw[fill, violet] (2,0) circle [radius=1mm];
\draw[fill, violet] (3,0) circle [radius=1mm];
\draw[fill, orange] (4,0) circle [radius=1mm];
\draw[fill, orange] (5,0) circle [radius=1mm];
\draw[blue] (0.5,-0.5) rectangle (5.5,0.5);
\node[] () at (3,1) {(independent)};
\draw[very thick, decorate,decoration={mirror, calligraphic brace, amplitude=5pt, raise=5pt}] (1,0) -- (3,0);
\draw[very thick, decorate,decoration={mirror, calligraphic brace, amplitude=5pt, raise=5pt}] (4,0) -- (5,0);
\draw[-Latex] (2,-0.4) -- (2.8,-1);
\draw[-Latex] (4.5,-0.4) -- (3.2,-1);
\node[] () at (3,-1.3) {independent};
\end{tikzpicture}
\end{center}
\end{enumerate}
\begin{pf}
\begin{enumerate}
\item Fix any \(n\in\N\), and any \(\{i_1,\dotsc,i_n\}\subseteq I\).  Then, fix
any \(A_{i_k}\in\mathcal{A}_{i_k}\) for every \(k=2,\dotsc,n\). Let
\(\mathcal{D}=\{A\in\mathcal{F}:\prob{A\cap A_{i_2}\cap\dotsb\cap
A_{i_n}}=\prob{A}\prod_{k=2}^{n}\prob{A_{i_k}}\}\).

\textbf{Showing that \(\mathcal{D}\) is a Dynkin system on \(\Omega\).}
\begin{enumerate}[label={(\arabic*)}]
\item By the definition of independence of a collection of families, we know that
\(\mathcal{A}_{i_2},\dotsc,\mathcal{A}_{i_n}\) are independent. Hence, we have
\[
\prob{\vc{\Omega}\cap A_{i_2}\cap\dotsb\cap A_{i_n}}
=\prob{A_{i_2}\cap\dotsb\cap A_{i_n}}
\overset{\text{(independence)}}{=}\prod_{k=2}^{n}\prob{A_{i_k}}
=\prob{\vc{\Omega}}\prod_{k=2}^{n}\prob{A_{i_k}},
\]
meaning that \(\Omega\in\mathcal{D}\).
\item Fix any \(A\in\mathcal{D}\). Then we have
\begin{align*}
\prob{A^c\cap\bigcap_{k=2}^{n}A_{i_k}}
&=\prob{\left.\bigcap_{k=2}^{n}A_{i_k}\right\backslash A}
=\prob{\left.\bigcap_{k=2}^{n}A_{i_k}\right\backslash\left(A\cap\bigcap_{k=2}^{n}A_{i_k}\right)} \\
&=\prob{\bigcap_{k=2}^{n}A_{i_k}}-\vc{\prob{A\cap\bigcap_{k=2}^{n}A_{i_k}}} \\
\overset{\vc{(A\in\mathcal{D})}}
&{=}
\prod_{k=2}^{n}\prob{A_{i_k}}-\vc{\prob{A}\prod_{k=2}^{n}\prob{A_{i_k}}}
=\prob{A^c}\prod_{k=2}^{n}\prob{A_{i_k}},
\end{align*}
thus \(A^c\in\mathcal{D}\).
\item Fix any pairwise disjoint \(A_1,A_2,\dotsc\in\mathcal{D}\). Then we have
\begin{align*}
\prob{\left(\biguplus_{i=1}^{\infty}A_i\right)\cap\bigcap_{k=2}^{n}A_{i_k}}
\overset{\text{(distributivity)}}&{=}
\prob{\biguplus_{i=1}^{\infty}\left(A_i\cap\bigcap_{k=2}^{n}A_{i_k}\right)}
=\sum_{i=1}^{\infty}\prob{A_i\cap\bigcap_{k=2}^{n}A_{i_k}} \\
\overset{(A_1,A_2,\dotsc\in\mathcal{D})}&{=}
\sum_{i=1}^{\infty}\left(\prob{A_i}\vc{\prod_{k=2}^{n}\prob{A_{i_k}}}\right)
=\vc{\prod_{k=2}^{n}\prob{A_{i_k}}}\sum_{i=1}^{\infty}\prob{A_i} \\
&=\prob{\biguplus_{i=1}^{\infty}A_i}\prod_{k=2}^{n}\prob{A_{i_k}},
\end{align*}
so \(\biguplus_{i=1}^{\infty}A_i\in\mathcal{D}\).
\end{enumerate}

\textbf{Applying Dynkin's \(\pi\)-\(\lambda\) theorem to show the independence
of \(\sigma(\mathcal{A}_{i_1}),\mathcal{A}_{i_2},\dotsc,\mathcal{A}_{i_n}\).}
By assumption, \(\mathcal{A}_{i_1},\dotsc,\mathcal{A}_{i_n}\) are independent.
So, particularly, for all \(\vc{A_{i_1}}\in\mathcal{A}_{i_1}\), we have
\(\prob{\vc{A_{i_1}}\cap A_{i_2}\cap\dotsb\cap
A_{i_n}}=\prob{\vc{A_{i_1}}}\prod_{k=2}^{n}\prob{A_{i_k}}\), meaning that
\(\vc{A_{i_1}}\in\mathcal{D}\). Therefore, we have \(\mathcal{A}_{i_1}\subseteq
\mathcal{D}\). Since \(\mathcal{A}_{i_1}\) is a \(\pi\)-system, by Dynkin's
\(\pi\)-\(\lambda\) theorem, we have \(\sigma(\mathcal{A}_{i_1})\subseteq
\mathcal{D}\).  It follows that
\(\sigma(\mathcal{A}_{i_1}),\mathcal{A}_{i_2},\dotsc,\mathcal{A}_{i_n}\) are
independent.

\textbf{Completing the proof by applying the argument repetitively.} By
repeating the argument above (change
\(\text{``\(\mathcal{A}_{i_1}\)''}\to\text{``\(\mathcal{A}_{i_2}\)}\)'' and
\(\text{``\(\mathcal{A}_{i_2},\dotsc,\mathcal{A}_{i_n}\)''}
\to\text{``\(\sigma(\mathcal{A}_{i_1}),\mathcal{A}_{i_3},\dotsc,\mathcal{A}_{i_n}\)''}\),
...), we deduce that
\(\sigma(\mathcal{A}_{i_1}),\sigma(\mathcal{A}_{i_2}),\dotsc,\sigma(\mathcal{A}_{i_n})\)
are independent.
\item Assume \(\{A_i\}_{i\in I}\subseteq \mathcal{F}\) is independent. Then,
consider the family \(\{\mathcal{A}_i\}_{i\in I}\) with \(\mathcal{A}_i\) being
the singleton \(\{A_i\}\) for all \(i\in I\). Note that
\(\{\mathcal{A}_i\}_{i\in I}\) contains independent \(\pi\)-systems, so
\(\{\sigma(\mathcal{A}_i)\}_{i\in I}\) is independent by
\labelcref{it:ind-via-gen}. Since
\(A_i^c\in\sigma(\mathcal{A}_i)=\{\varnothing,A_i,A_i^c,\Omega\}\) for all
\(i\in I\), it follows that \(\{A_i^c\}_{i\in I}\) is independent.
\item For every \(j\in J\), let \(\mathcal{A}_{I_j}:=\{
\bigcap_{i\in\widetilde{I}_{j}}^{}A_i:A_i\in \mathcal{A}_i~\forall i\in \widetilde{I}_{j},\quad
\widetilde{I}_{j}\subseteq I_{j},\quad |\widetilde{I}_{j}|<\infty\}\), which is
a \(\pi\)-system by construction. Since \(\{\mathcal{A}_{i}\}_{i\in I}\) is a
collection of independent \(\pi\)-systems, by considering the definition we
deduce that \(\{\mathcal{A}_{I_j}\}_{j\in J}\) is also a collection of
independent \(\pi\)-systems. Then, by \labelcref{it:ind-via-gen}, the
collection \(\{\sigma(\mathcal{A}_{I_j})\}_{j\in J}\) is independent.

Now, to complete the proof, we are going to show that \(\sigma(\bigcup_{i\in
I_j}^{}\mathcal{A}_i)=\sigma(\mathcal{A}_{I_j})\):
\begin{itemize}
\item ``\(\subseteq\)'': Fix any \(i\in I_j\). Note that
\[\mathcal{A}_i=\{A_i:A_i\in\mathcal{A}_i\}
=\left\{\bigcap_{i\in\widetilde{I}_{j}}^{}A_i:A_i\in \mathcal{A}_i~\forall i\in \widetilde{I}_{j},\quad \widetilde{I}_{j}=\{i\}\right\}
\subseteq \mathcal{A}_{I_j}
\subseteq \sigma(\mathcal{A}_{I_j}),\]
thus \(\bigcup_{i\in I_j}^{}\mathcal{A}_i\subseteq \sigma(\mathcal{A}_{I_j})\).
Then, by the minimality we have \(\sigma\left(\bigcup_{i\in
I_j}^{}\mathcal{A}_i\right)\subseteq \sigma(\mathcal{A}_{I_j})\).
\item ``\(\supseteq\)'': Fix any \(\widetilde{I}_{j}\subseteq I_j\) with \(|\widetilde{I}_{j}|<\infty\),
and any \(A_i\in\mathcal{A}_i\) for all \(i\in\widetilde{I}_{j}\). Then, we have
\(A_i\in\bigcup_{i\in I_j}^{}\mathcal{A}_i\subseteq \sigma(\bigcup_{i\in I_j}^{}\mathcal{A}_i)\)
for all \(i\in\widetilde{I}_{j}\). Hence, by the closedness under intersections
for \(\sigma(\bigcup_{i\in I_j}^{}\mathcal{A}_i)\), we have \(
\bigcap_{i\in\widetilde{I}_{j}}^{}A_i\in\sigma(\bigcup_{i\in I_j}^{}\mathcal{A}_i)\).

This implies that \(\mathcal{A}_{I_j}\subseteq \sigma(\bigcup_{i\in
I_j}^{}\mathcal{A}_i)\), and hence
\(\sigma(\mathcal{A}_{I_j})\subseteq \sigma(\bigcup_{i\in
I_j}^{}\mathcal{A}_i)\) by the minimality.
\end{itemize}
\end{enumerate}
\end{pf}
\item \textbf{Zero-one laws.} Generally, a \defn{zero-one law} refers to any
result which asserts that the probability of an event has to be
\underline{zero} or \underline{one}, under certain conditions. Here we will go
through some popular zero-one laws, namely the \emph{Borel-Cantelli lemmas} and
the \emph{Kolmogorov's zero-one law}.

\begin{theorem}[Borel-Cantelli lemmas]
\label{thm:bc-lma}
Let \((\Omega,\mathcal{F},\pr)\) be a probability space, and
\(\{A_i\}_{i\in\N}\subseteq \mathcal{F}\) be a collection of events.
\begin{enumerate}
\item \emph{(First Borel-Cantelli lemma)} If \(\sum_{i=1}^{\infty}\prob{A_i}<\infty\), then
\(\prob{\{\omega\in A_n\text{ io}\}}=0\).
\item \emph{(Second Borel-Cantelli lemma)} If \(\{A_i\}_{i\in\N}\) is
independent and \(\sum_{i=1}^{\infty}\prob{A_i}=\infty\), then
\(\prob{\{\omega\in A_n\text{ io}\}}=1\).
\end{enumerate}
\end{theorem}
\begin{pf}
\begin{enumerate}
\item Note that
\[
\prob{\{\omega\in A_n\text{ io}\}}
\overset{\text{\labelcref{it:liminfsup-interpret}}}{=}
\prob{\limsup_{n\to \infty}A_n}
=\prob{\bigcap_{n=1}^{\infty}\bigcup_{k=n}^{\infty}A_k}
\overset{\text{(monotonicity)}}{\le}\prob{\bigcup_{k=n}^{\infty}A_k}
\overset{\text{(\(\sigma\)-subadditivity)}}{\le}\sum_{k=n}^{\infty}\prob{A_k}.
\]
By assumption, we have \(\sum_{k=1}^{\infty}\prob{A_k}=S<\infty\), and hence
\[
\lim_{n\to \infty}\sum_{k=n}^{\infty}\prob{A_k}
=\lim_{n\to \infty}\left(S-\sum_{k=1}^{n-1}\prob{A_k}\right)
=S-S=0.
\]
So, taking the limit \(n\to\infty\) on the inequality above yields
\(\prob{\{\omega\in A_n\text{ io}\}}\le 0\). Together with the
nonnegativity of \(\pr\), we must have \(\prob{\{\omega\in A_n\text{
io}\}}=0\).
\item With \(\sum_{i=1}^{\infty}\prob{A_i}=\infty\), we also have
\(\vc{\sum_{k=n}^{\infty}\prob{A_k}=\infty}\) for all \(n\in\N\). Hence,
\begin{align*}
\prob{\{\omega\in A_n\text{ io}\}}
\overset{\text{\labelcref{it:liminfsup-interpret}}}&{=}
\prob{\limsup_{n\to \infty}A_n}
\overset{\text{\labelcref{it:liminfsup-relate}}}{=}1-\prob{\liminf_{n\to \infty}A_n^c} \\
\overset{\text{\labelcref{it:lim-mono-sets}}}&{=}1-\prob{\lim_{n\to\infty}\inf_{k\ge n}A_k^c}
=1-\prob{\lim_{n\to\infty}\bigcap_{k=n}^{\infty}A_k^c} \\
\overset{\text{(continuity from below)}}&{=}1-\lim_{n\to\infty}\prob{\bigcap_{k=n}^{\infty}A_k^c}
\underset{\text{(\(\{A_k\}_{k=n}^{\infty}\) are independent)}}
{\overset{\text{\labelcref{it:ind-comp}}}{=}}1-\lim_{n\to\infty}\prod_{k=n}^{\infty}\prob{A_k^c} \\
\overset{(1-x\le e^{-x})}&{\ge}1-\lim_{n\to\infty}\prod_{k=n}^{\infty}e^{-\prob{A_k}}
=1-\lim_{n\to\infty}\underbrace{e^{-\vc{\sum_{k=n}^{\infty}\prob{A_k}}}}_{0}
=1.
\end{align*}
\end{enumerate}
\end{pf}

The next result will be the \emph{Kolmogorov's zero-one law}. But before
stating that, we first introduce a preliminary concept: \emph{tail
\(\sigma\)-algebra}. Let \((\Omega,\mathcal{F})\) be a measurable space and
\(\{\mathcal{A}_n\}_{n\in\N}\subseteq \mathcal{P}(\mathcal{F})\) be a
collection of \(\pi\)-systems. Then, the \defn{tail \(\sigma\)-algebra (of
\(\{\mathcal{A}_n\}_{n\in\N}\))} , denoted by \(\mathcal{T}\) or
\(\mathcal{T}(\{\mathcal{A}_n\}_{n\in\N})\), is given by
\(\mathcal{T}=\bigcap_{n=1}^{\infty}\sigma(\bigcup_{k=n}^{\infty}\mathcal{A}_k)
=\bigcap_{n=1}^{\infty}\sigma(\mathcal{A}_n,\mathcal{A}_{n+1},\dotsc)\) (check
that it is a \(\sigma\)-algebra on \(\Omega\)!). Every event in \(\mathcal{T}\)
is said to be a \defn{tail event}.
\begin{intuition}
For the \underline{tail} \(\sigma\)-algebra
\(\mathcal{T}=\bigcap_{n=1}^{\infty}\sigma(\mathcal{A}_n,\mathcal{A}_{n+1},\dotsc)\),
it is obtained by intersecting the \(\sigma\)-algebras generated by the
``\underline{tails}'' from \(\{\mathcal{A}_n\}_{n\in\N}\).
\end{intuition}
\begin{theorem}[Kolmogorov's zero-one law]
\label{thm:kolmogorov-zero-one}
Let \((\Omega,\mathcal{F},\pr)\) be a probability space,
\(\{\mathcal{A}_n\}_{n\in\N}\subseteq \mathcal{P}(\mathcal{F})\) be a
collection of independent \(\pi\)-systems, and \(\mathcal{T}\) be the tail
\(\sigma\)-algebra of \(\{\mathcal{A}_n\}_{n\in\N}\). Then,
\(\prob{A}=0\text{ or }1\) for all \(A\in\mathcal{T}\).
\end{theorem}
\begin{pf}
Let \(\mathcal{T}_n:=\sigma(\bigcup_{k=1}^{n-1}\mathcal{A}_k)\) for every \(n\in\N\), with
\(\mathcal{T}_{\infty}:=\sigma(\bigcup_{k=1}^{\infty}\mathcal{A}_k)\).

\textbf{Showing the independence of \(\bigcup_{n=1}^{\infty}\mathcal{T}_n\) and \(\mathcal{T}\).}
Since \(\{\mathcal{A}_{n}\}\) is independent, by \labelcref{it:grouping-lma},
\(\vc{\mathcal{T}_n}=\sigma(\bigcup_{k=1}^{n-1}\mathcal{A}_k)\) and
\orc{\(\sigma(\bigcup_{k=n}^{\infty}\mathcal{A}_k)\)} are independent
\(\sigma\)-algebras for all \(n\in\N\).

Noting that \(\mathcal{T}=\bigcap_{n=1}^{\infty}\sigma(\bigcup_{k=n}^{\infty}\mathcal{A}_k)
\subseteq\orc{\sigma(\bigcup_{k=n}^{\infty}\mathcal{A}_k)}\) for all \(n\in\N\),
we conclude that \vc{\(\mathcal{T}_n\)} and \(\mathcal{T}\) are independent for all \(n\in\N\).
Thus, by considering the definition we have the independence of
\(\bigcup_{n=1}^{\infty}\mathcal{T}_n\) and \(\mathcal{T}\).

\textbf{Showing the independence of \(\sigma\left(\bigcup_{n=1}^{\infty}\mathcal{T}_n\right)\) and \(\mathcal{T}\).}
Next, we will show that \(\bigcup_{n=1}^{\infty}\mathcal{T}_n\) is a
\(\pi\)-system: Fix any \(A_1,A_2\in\bigcup_{n=1}^{\infty}\mathcal{T}_n\). Then
we know \(A_1\in\mathcal{T}_{n_1}\) and \(A_2\in\mathcal{T}_{n_2}\) for some \(n_1,n_2\in\N\).
With \(\mathcal{T}_n\nearrow\), we then have \(A_1,A_2\in\mathcal{T}_{\max\{n_1,n_2\}}\).
By the closedness under intersections for for the \(\sigma\)-algebra \(\mathcal{T}_{\max\{n_1,n_2\}}\),
\(A_1\cap A_2\in\mathcal{T}_{\max\{n_1,n_2\}}\subseteq \bigcup_{n=1}^{\infty}\mathcal{T}_n\).

Of course, \(\mathcal{T}\) is also a \(\pi\)-system, since it is closed under
intersections.  Hence, by \labelcref{it:ind-via-gen},
\(\sigma\left(\bigcup_{n=1}^{\infty}\mathcal{T}_n\right)\) and
\(\sigma(\mathcal{T})\) are independent, which implies the independence of
\(\sigma\left(\bigcup_{n=1}^{\infty}\mathcal{T}_n\right)\) and \(\mathcal{T}\subseteq \sigma(\mathcal{T})\).

\textbf{Showing the independence of \(\mathcal{T}_{\infty}\) and
\(\mathcal{T}\).} We first note the following result: \(\mathcal{A}\subseteq
\mathcal{B}\implies \sigma(\mathcal{A})\subseteq \sigma(\mathcal{B})\). This
holds because with \(\mathcal{A}\subseteq \mathcal{B}\subseteq
\sigma(\mathcal{B})\), by the minimality we would have
\(\sigma(\mathcal{A})\subseteq \sigma(\mathcal{B})\). Using this result, we will
show that \(\sigma(\bigcup_{n=1}^{\infty}\mathcal{T}_n)=\mathcal{T}_{\infty}\),
and hence establish the independence of \(\mathcal{T}_{\infty}\) and
\(\mathcal{T}\).
\begin{itemize}
\item ``\(\subseteq\)'': Since \(\mathcal{T}_n\subseteq \mathcal{T}_{\infty}\) for every \(n\in\N\),
we have \(\bigcup_{n=1}^{\infty}\mathcal{T}_n\subseteq \mathcal{T}_{\infty}\), thus
\(\sigma(\bigcup_{n=1}^{\infty}\mathcal{T}_n)\subseteq \mathcal{T}_{\infty}\).
\item ``\(\supseteq\)'': For every \(n\in\N\), we have \(\mathcal{A}_n\subseteq
\sigma(\bigcup_{k=1}^{n}\mathcal{A}_n)=\mathcal{T}_{n+1}\). Thus,
\(\bigcup_{n=1}^{\infty}\mathcal{A}_n\subseteq
\bigcup_{n=1}^{\infty}\mathcal{T}_{n}\), which implies that
\(\mathcal{T}_{\infty}=\sigma(\bigcup_{n=1}^{\infty}\mathcal{A}_n) \subseteq
\sigma(\bigcup_{n=1}^{\infty}\mathcal{T}_n)\) by the result above.
\end{itemize}
\textbf{Showing that every \(A\in\mathcal{T}\) is independent of itself.}
Note that \(\mathcal{T}=\bigcap_{n=1}^{\infty}\sigma(\bigcup_{k=n}^{\infty}\mathcal{A}_k)
\subseteq \overset{(n=1)}{\sigma(\bigcup_{k=1}^{\infty}\mathcal{A}_k)}
=\mathcal{T}_{\infty}\). So, the independence of \(\mathcal{T}_{\infty}\) and
\(\mathcal{T}\) implies that every \(\mathcal{A}\in\mathcal{T}\) is independent of itself,
which means that \(\prob{A}=\prob{A\cap A}=\prob{A}\prob{A}=\prob{A}^{2}\), or
\(\prob{A}=0\text{ or }1\), for every \(A\in\mathcal{T}\).
\end{pf}
\begin{remark}
\item \emph{(Application of Kolmogorov's zero-one law to independent events)}
By taking \(\mathcal{A}_n=\{A_n\}\) for every \(n\in\N\), we know that given a
collection \(\{A_n\}_{n\in\N}\subseteq \mathcal{F}\) of independent events, we have
\(\prob{A}=0\text{ or }1\) for every \(A\in\mathcal{T}=\bigcap_{n=1}^{\infty}\sigma(\{A_n,A_{n+1},\dotsc\})\).
\item \emph{(Application of Kolmogorov's zero-one law to independent random
variables)} Suppose we are given a collection \(\{X_n\}_{n\in\N}\) of
independent random variables, all defined on \((\Omega,\mathcal{F},\pr)\).
Then, by taking \(\mathcal{A}_n=\sigma(X_n)\) for every \(n\in\N\), we have
\(\prob{A}=0\text{ or }1\) for every
\(A\in\mathcal{T}=\bigcap_{n=1}^{\infty}\vc{\sigma(\sigma(X_n),\sigma(X_{n+1}),\dotsc)}\).

Particularly, we can indeed write \(\vc{\sigma(\sigma(X_n),\sigma(X_{n+1}),\dotsc)}
=\mgc{\sigma(X_n,X_{n+1},\dotsc)}\), where the latter is the \(\sigma\)-algebra
generated by \(\{X_k\}_{k=n}^{\infty}\), defined by
\(\sigma(\bigcup_{k=n}^{\infty}\sigma(X_k))\), which is the same as
\(\sigma(\sigma(X_n),\sigma(X_{n+1}),\dotsc)\).
\end{remark}
\item \textbf{Characterization of independence of random variables.} By
definition, to verify the independence of a collection \(\{X_i\}_{i\in I}\) of
random variables, we need to check that
\(X_{i_1}^{-1}(B_{i_1}),\dotsc,X_{i_n}^{-1}(B_{i_n})\) are independent
\(\forall~B_{i_1},\dotsc,B_{i_n}\in\mathcal{B}(\R),\quad
\forall~\{i_{1},\dotsc,i_{n}\}\subseteq I,\quad \forall~n\in\N\), which can
take quite a lot of work. The following result provides us an alternative and
often more convenient route for showing such independence:

\begin{theorem}[Characterization of independence of random variables]
\label{thm:char-rv-ind}
Let \((\Omega,\mathcal{F},\pr)\) be a probability space, and
\(X_i:\Omega\to\R\) be a random variable for every \(i\in I\). Then, the
collection \(\{X_i:i\in I\}\) is independent iff for all \(n\in\N\) and \(\{i_1,\dotsc,i_n\}\subseteq I\),
\(\prob{X_{i_1}\le x_{i_1},\dotsc,X_{i_n}\le x_{i_n}}=\prod_{k=1}^{n}\prob{X_{i_k}\le x_{i_k}}\).
\end{theorem}
\begin{pf}
``\(\Rightarrow\)'': Assume \(\{X_i:i\in I\}\) is independent. Then by
definition, \(\{\sigma(X_i)\}_{i\in I}\) is independent. This implies that
\(\forall~i_1,\dotsc,i_n\in\R\quad\forall~n\in\N\), the events
\(X_{i_1}^{-1}((-\infty,x_{i_1}]),\dotsc,X_{i_n}^{-1}((-\infty,x_{i_n}])\) are
independent, and thus
\[
\prob{X_{i_1}\le x_{i_1},\dotsc,X_{i_n}\le x_{i_n}}
=\prob{\bigcap_{k=1}^{n}X_{i_k}^{-1}((-\infty,x_{i_k}])}
=\prod_{k=1}^{n}\prob{X_{i_k}^{-1}((-\infty,x_{i_k}]}
=\prod_{k=1}^{n}\prob{X_{i_k}\le x_{i_k}}.
\]
``\(\Leftarrow\)'': For every \(k=1,\dotsc,n\), let
\(\mathcal{A}_{i_k}:=\{X_{i_k}^{-1}((-\infty,x])\}_{x\in\R}\).  By the property
of preservation of intersection for preimages (\labelcref{it:preimg-prop}),
\(\mathcal{A}_{i_k}\)'s are all \(\pi\)-systems.  By assumption, we know
\(\mathcal{A}_{i_1},\dotsc,\mathcal{A}_{i_n}\) are independent, thus
\(\sigma(\mathcal{A}_{i_1}),\dotsc,\sigma(\mathcal{A}_{i_n})\) are independent
by \labelcref{it:ind-via-gen}.

Noting that for every \(k=1,\dotsc,n\),
\begin{align*}
\sigma(\mathcal{A}_{i_k})&=\sigma(\vc{\{X_{i_k}^{-1}((-\infty,x])\}_{x\in\R}})
\overset{\text{(notation)}}{=}\sigma(\vc{X_{i_k}^{-1}(\{(-\infty,x]\}_{x\in\R})}) \\
\overset{\text{(\Cref{lma:sig-alg-gen-preimg-comm})}}&{=}
X_{i_k}^{-1}(\sigma(\{(-\infty,x]\}_{x\in\R}))
\overset{\text{(\Cref{prp:borel-rd-generators})}}{=}X_{i_k}^{-1}(\mathcal{B}(\R))
=\sigma(X_{i_k}),
\end{align*}
we conclude that \(\sigma(X_{i_1}),\dotsc,\sigma(X_{i_n})\) are independent.
Hence, by definition, \(\{X_i\}_{i\in I}\) is independent.
\end{pf}

\begin{remark}
\item \emph{(Special case for finitely many random variables --- factorization
of distribution function)} With \(I=\{1,\dotsc,d\}\), random variables
\(X_1,\dotsc,X_d\) with \(\vect{X}=(X_1,\dotsc,X_d)\sim F\) and margins
\(F_1,\dotsc,F_d\) are independent iff \[
\vc{F(\vect{x})}=\prob{\vect{X}\le\vect{x}}=
\prod_{j=1}^{d}\prob{X_j\le x_j}\vc{=\prod_{j=1}^{d}F_j(x_j)},\quad\text{for all \(\vect{x}\in\R^d\)}.
\]
(This condition indeed implies that \(\prob{X_{i_1}\le
x_{i_1},\dotsc,X_{i_n}\le x_{i_n}}=\prod_{k=1}^{n}\prob{X_{i_k}\le x_{i_k}}\)
for every \(\{i_1,\dotsc,i_n\}\subseteq I\), by letting some components of
\(\vect{x}\) to \(\infty\).)
\item \emph{(Special case for finitely many random variables --- factorization
of density function)} Assuming further that \(F\) has continuous partial
derivatives with respect to each component in \(\supp{F}\), we know that \(F\) is
absolutely continuous with density given by
\(f(\vect{x})=\frac{\partial^{d}}{\partial x_d\cdots\partial x_1}F(\vect{x})\). In such case,
\(X_1,\dotsc,X_d\) are independent iff
\[
f(\vect{x})=\frac{\partial^{d}}{\partial x_d\cdots\partial x_1}F(\vect{x})
=\prod_{j=1}^{d}\pdv{}{x_j}F_j(x_j)
=\prod_{j=1}^{d}f_j(x_j),\quad\text{for all \(\vect{x}\in\supp{F}\)}.
\]
(For discrete \(\vect{X}=(X_1,\dotsc,X_d)\), we have similarly that
\(X_1,\dotsc,X_d\) are independent iff the joint mass function is
\(f(\vect{x})=\prod_{j=1}^{d}f_j(x_j)\) for all
\(\vect{x}\in\prod_{j=1}^{d}\supp{F_j}\).)

\item \emph{(Survival function version of the result)} We indeed also have a
``survival function version'' of the result: \(\{X_i:i\in I\}\) is independent
iff for all \(n\in\N\) and \(\{i_1,\dotsc,i_n\}\subseteq I\),
\(\prob{X_{i_1}>x_{i_1},\dotsc,X_{i_n}>x_{i_n}}=\prod_{k=1}^{n}\prob{X_{i_k}>x_{i_k}}\).
This can be shown by adapting the proof slightly: change
\((-\infty,x_{i_k}]\to (x_{i_k},\infty)\) for every \(k\) and \((-\infty,x]\to
(x,\infty)\).
\end{remark}
\item \textbf{Functions of independent random variables are also independent.}
An important result regarding the independence of random variable is that
\emph{functions of (disjoint sets of) independent random variables are also
independent}, which can simplify the checking of independence of random
variables a lot. The precise formulation of this statement is as
follows.

\begin{theorem}[Functions of independent random variables are also independent]
\label{thm:fun-ind-rv-ind}
If \(\{X_{ij}\}_{j=1,i=1}^{d_i,\infty}\) is a collection of independent random variables,
and \(h_i:\R^{d_i}\to\R\) is
\((\mathcal{B}(\R^{d_i}),\mathcal{B}(\R))\)-measurable for every \(i\in\N\),
then the collection \(\{Y_i\}\) of random variables, with
\(Y_i:=h_i(X_{i1},\dotsc,X_{id_i})\) for every \(i\in\N\), is independent.
\end{theorem}
\begin{pf}
By assumption, the collection \(\{\sigma(X_{ij})\}_{j=1,i=1}^{d_i,\infty}\) of
\(\sigma\)-algebras is independent, so by \labelcref{it:grouping-lma} the
collection \(\left\{\sigma\left(\bigcup_{j=1}^{d_i}\sigma(X_{ij})\right)\right\}_{i\in\N}\) 
is independent.

\textbf{Showing that \(\mathcal{F}_i\) is a \(\sigma\)-algebra.} Fix any
\(i\in\N\). Let \(\vect{X}_i=(X_{i1},\dotsc,X_{id_i})\). Then, we will show
that \[\mathcal{F}_i:=\left\{B\in\mathcal{B}(\R^{d_i}):\vect{X}_i^{-1}(B)
\in\sigma\left(\bigcup_{j=1}^{d_i}\sigma(X_{ij})\right)\right\}\] is a
\(\sigma\)-algebra:
\begin{enumerate}[label={(\arabic*)}]
\item For \(\varnothing\in\mathcal{B}(\R^{d_i})\), we have \(\vect{X}_i^{-1}(\varnothing)=\varnothing\in
\sigma\left(\bigcup_{j=1}^{d_i}\sigma(X_{ij})\right)\), thus \(\varnothing\in\mathcal{F}_i\).
\item Fix any \(B\in\mathcal{F}_i\). Then, by the closedness under complements, we
have \(B^c\in \mathcal{B}(\R^{d_i})\), and also
\[
\vect{X}_i^{-1}(B^c)
\overset{\text{\labelcref{it:preimg-prop}}}{=}(\vect{X}_i^{-1}(B))^{c}
\in\sigma\left(\bigcup_{j=1}^{d_i}\sigma(X_{ij})\right).
\]
Hence, \(B^c\in\mathcal{F}_i\).
\item Fix any \(B_1,B_2,\dotsc,\in\mathcal{F}_i\). Then by the closedness under
countable unions, we have \(\bigcup_{n=1}^{\infty}B_n\in \mathcal{B}(\R^{d_i})\)
and also
\[
\vect{X}_i^{-1}\left(\bigcup_{n=1}^{\infty}B_n\right)
\overset{\text{\labelcref{it:preimg-prop}}}{=}\bigcup_{n=1}^{\infty}\vect{X}_i^{-1}(B_n)
\in\sigma\left(\bigcup_{j=1}^{d_i}\sigma(X_{ij})\right).
\]
\end{enumerate}
\textbf{Showing that \(\mathcal{F}_i=\mathcal{B}(\R^{d_i})\).}
By \labelcref{it:borel-rd-prod-sig}, \(\mathcal{B}(\R^{d_i})\) is the product
\(\sigma\)-algebra \(\bigotimes_{j=1}^{d_i}\mathcal{B}(\R)=
\sigma(\{\prod_{j=1}^{d_i}B_j:B_j\in\mathcal{B}(\R)~\forall
j=1,\dotsc,d_i\})\). Using this, we will then show that
\(\mathcal{F}_i=\mathcal{B}(\R^{d_i})\):
\begin{itemize}
\item ``\(\subseteq\)'': It is immediate by the definition of \(\mathcal{F}_i\).
\item ``\(\supseteq\)'': Fix any \(B=\prod_{j=1}^{d_i}B_j\) where
\(B_j\in\mathcal{B}(\R)\) for every \(j=1,\dotsc,d_i\). Then,
\(\vect{X}_i^{-1}(B) =\bigcap_{j=1}^{d_i}X_{ij}^{-1}(B_j)\). Since
\(X_{ij}^{-1}(B_j)\in X_{ij}^{-1}(\mathcal{B}(\R)) =\sigma(X_{ij})\subseteq
\sigma(\bigcup_{j=1}^{d_i}\sigma(X_{ij}))\) for every \(j=1,\dotsc,d_i\), by
the closedness under intersections we have \(\vect{X}_i^{-1}(B)\in
\sigma(\bigcup_{j=1}^{d_i}\sigma(X_{ij}))\). By definition of \(\mathcal{F}_i\), this means
\(B\in\mathcal{F}_i\). 

Hence, we have \(\{\prod_{j=1}^{d_i}B_j:B_j\in\mathcal{B}(\R)~\forall
j=1,\dotsc,d_i\}\subseteq \mathcal{F}_i\). By the minimality, we then have
\(\mathcal{B}(\R^{d_i})=\bigotimes_{j=1}^{d_i}\mathcal{B}(\R)\subseteq\mathcal{F}_i\).
\end{itemize}
\textbf{Showing that \(Y_i\) is a random variable (with \(\mathcal{F}=\sigma\left(\bigcup_{j=1}^{d_i}\sigma(X_{ij})\right)\)).}
Since \(\vect{X}_i^{-1}(\mathcal{B}(\R^{d_i}))=\vect{X}_{i}^{-1}(\mathcal{F}_i)
\subseteq \sigma\left(\bigcup_{j=1}^{d_i}\sigma(X_{ij})\right)\), we know \(\vect{X}_i\) is
\(\left(\sigma\left(\bigcup_{j=1}^{d_i}\sigma(X_{ij})\right),\mathcal{B}(\R^{d_i})\right)\)-measurable.
So, by \labelcref{it:meas-compo-meas}, \(Y_i=h_i\circ \vect{X}_i\) is
\(\left(\sigma\left(\bigcup_{j=1}^{d_i}\sigma(X_{ij})\right),\mathcal{B}(\R)\right)\)-measurable,
implying that \(Y_i\) is a random variable.

\textbf{Showing the independence of \(\{Y_i\}_{i\in\N}\).}
For every \(i\in\N\), we have
\begin{align*}
\sigma(Y_i)&=Y_i^{-1}(\mathcal{B}(\R))
\overset{\text{(similar to the proof of \labelcref{it:meas-compo-meas})}}{=}
\vect{X}_{i}^{-1}\left(h_i^{-1}(\mathcal{B}(\R))\right)
\overset{\text{\labelcref{it:preimg-prop}}}{\subseteq}\vect{X}_{i}^{-1}(\mathcal{B}(\R^{d_i}))
\subseteq \sigma\left(\bigcup_{j=1}^{d_i}\sigma(X_{ij})\right).
\end{align*}
As shown previously, the collection
\(\left\{\sigma\left(\bigcup_{j=1}^{d_i}\sigma(X_{ij})\right)\right\}_{i\in\N}\)
is independent. Hence, this subset inclusion implies that the collection
\(\{\sigma(Y_i)\}_{i\in\N}\) is independent also, which means that
\(\{Y_i\}_{i\in\N}\) is independent.
\end{pf}
\item \textbf{Mathematical construction of independent random variables.} In
probability theory and statistics, we often work with independent random
variables. Particularly, in statistics we talk about \emph{independent and
identically distributed} (iid) random variables a lot. An implicit assumption
made here is that we can always construct \emph{independent} random variables,
regardless of their underlying (marginal) distributions. While this is quite
reasonable and intuitive, it does take some work to mathematically show that
such construction is possible:

\begin{proposition}[Construction of independent random variables]
\label{prp:construct-ind-rvs}
Let \(F_1,\dotsc,F_d\) be arbitrary distribution functions on \(\R\). Then
there exist a probability space \((\Omega,\mathcal{F},\pr)\), and independent
random variables \(X_1\sim F_1,\dotsc,X_d\sim F_d\) defined on this probability
space.
\end{proposition}
\begin{pf}
Since \(F_1,\dotsc,F_d\) are distribution functions,
\((\Omega,\mathcal{F},\pr)=(\R^{d},\mathcal{B}(\R^{d}),\prod_{j=1}^{d}\lambda_{F_j})\)
is a probability space, by \Cref{thm:dist-fn-char-prob-meas-rd}. Now fix any
\(j=1,\dotsc,d\). Since the projection \(\pi_{j}\) is continuous, and thus
measurable by \labelcref{it:cts-fn-meas}, the function \(X_j:\Omega\to\R\) given by
\(X_j(\vect{\omega}):=\pi_j(\vect{\omega})=\omega_j\quad\forall~
\vect{\omega}\in\R^d\) is a random variable.  Hence, by
\labelcref{it:ran-vec-vec-rvs}, the function \(\vect{X}:\Omega\to\R^d\) given by
\(\vect{X}(\vect{\omega}):=(X_1(\vect{\omega}),\dotsc,X_d(\vect{\omega}))
=\vect{\omega}\quad\forall~\vect{\omega}\in\R^d\) is a
random vector.

Now, since we have
\begin{align*}
F(\vect{x})&=\prob{\vect{X}\le\vect{x}}=\prob{\{\vect{\omega}\in\Omega:\vect{\omega}\le\vect{x}\}}
=\prob{\bigcap_{j=1}^{d}\{\omega_j\in\R:\omega_j\le x_j\}} \\
&=\prob{\prod_{j=1}^{d}(-\infty,x_j]}
\overset{(\pr=\prod_{j=1}^{d}\lambda_{F_j})}{=}\prod_{j=1}^{d}
\underbrace{\lambda_{F_j}((-\infty,x_j])}_{\Delta_{(-\infty,x_j]}F_j=F_j(x_j)}
=\prod_{j=1}^{d}F_j(x_j)
\end{align*}
for all \(\vect{x}\in\R^d\), we conclude by \Cref{thm:char-rv-ind} that
\(X_1,\dotsc,X_d\) are independent.

Furthermore, letting \(x_k\to\infty\) for every \(k\ne j\) on the equation
above gives \(\prob{X_j\le x_j}=F_j(x_j)\). Thus, \(X_j\sim F_j\) for every
\(j=1,\dotsc,d\).
\end{pf}

\begin{remark}
\item \emph{(Independent and identically distributed random variables)} If
\(F_1=\dotsb=F_d=F\), then \(X_1,\dotsc,X_d\) are said to be \defn{independent
and identically distributed (iid)} from \(F\), or \defn{independent copies of \(X\sim F\)}.
This is denoted by \(X_1,\dotsc,X_d\iid F\).
\item \emph{(Infinite sequence of independent random variables)} Using
\emph{Kolmogorov's extension theorem}, we can also construct infinite sequences
of independent random variables.
\end{remark}
\item\label{it:mixture-dist} \textbf{Mixture distributions.} In actuarial
science, apart from working with independent random variables, often we deal
with \emph{mixture distributions}, e.g., aggregate loss
\(S:=\sum_{i=1}^{N}X_i\), where each \(X_i\) represents a loss amount, and
\(N\) is the (random) number of losses, taking values in \(\N\). Here \(S\)
possesses a mixture distribution: a distribution that is created by ``mixing''
two sources of randomness, one from \(X_i\)'s and another from \(N\).  To be
more precise, \(S\) should be interpreted as the sum of random variables
\(\sum_{i=1}^{n}X_i\) \emph{given} \(N=n\), for every \(n\in\N\); from here the
relationship between mixture distribution and \emph{conditioning} becomes more
transparent.

In general, let \(I\sim F_I\) be a random variable with \(\supp{F_I}=\N\), and
\(\vect{X}_i\sim F_i\) for every \(i\in\N\). Suppose that
\(I,\vect{X}_1,\vect{X}_2,\dotsc\) are independent. Then,
\(\vect{X}=\vect{X}_{I}\) (interpreted as \(\vect{X}_i\) given \(I=i\), or more
explicitly, \(\vect{X}_{I}(\omega):=\vect{X}_{I(\omega)}(\omega)\) for all
\(\omega\in\Omega\)), follows the \defn{mixture distribution} of
\(F_1,F_2,\dotsc\) with respect to \(F_{I}\).

\begin{note}
It is indeed possible to define mixture distribution in a similar way when
\(\supp{F_I}\) is uncountable (e.g., \(I\sim U(0,1)\)), but some technicalities
are involved so we would not discuss it in details here.
\end{note}

\item\label{it:mixture-df} \textbf{Distribution function for mixture distribution.}
For \(\vect{X}\) following the mixture distribution of \(F_1,F_2,\dotsc\) with
respect to \(F_{I}\), its distribution function admits a ``weighted average''
form:
\[
F(\vect{x})=\sum_{i=1}^{\infty}p_iF_i(\vect{x}),\quad\text{for all \(\vect{x}\in\R^d\)}
\]
where \(p_i:=\prob{I=i}\) for every \(i\in\N\).

\begin{pf}
For all \(\vect{x}\in\R^d\), we have
\[
F(\vect{x})=\prob{\vect{X}\le\vect{x}}
=\prob{\vect{X}_{I}\le\vect{x}}
=\sum_{i=1}^{\infty}p_i\prob{\vect{X}_{I}\le\vect{x}|I=i}
\overset{\text{(independence)}}{=}\sum_{i=1}^{\infty}p_i\prob{\vect{X}_i\le\vect{x}}
=\sum_{i=1}^{\infty}p_iF_i(\vect{x}).
\]
\end{pf}
\item \label{it:sampling} \textbf{Sampling.} Recall from
\Cref{subsect:univariate-trans} that some results concerning transformations of
random variables are discussed, and there we mentioned that they are helpful
for performing simulations. After learning about the concept of
\emph{independence}, we can study more details about it.

A collection of iid random vectors \(\vect{X}_1,\dotsc,\vect{X}_n\iid F\) is called a
\defn{random sample} from \(F\), which is often encountered in statistical
inference. Besides, for conducting simulation studies, we would need to
generate \emph{realizations} (something observable \faIcon{eye}) of
\(\vect{X}_1,\dotsc,\vect{X}_n\). One famous technique is known as the \emph{inversion
method}, which utilizes the quantile transform in \Cref{prp:quantile-trans}.

The \defn{inversion method} for sampling from a random variable \(X\sim F\) is
to first obtain (independent) realizations from \(U\sim \unif{0}{1}\) (many
algorithms are available for this), and then evaluate the quantile function
\(F^{-1}\) on each of them. This is theoretically justified because \(F^{-1}(U)\sim F\),
so the resulting values \(F^{-1}(u)\)'s can be seen as realizations of a
random vector with the distribution function \(F\).

To sample from the iid random variables \(X_1,\dotsc,X_n\), we can use the
inversion method and consider \(F^{-1}(U_1),\dotsc,F^{-1}(U_n)\iid F\) with
\(U_1,\dotsc,U_n\iid \unif{0}{1}\). Realizations of each \(U_i\) are first
obtained, and then transformed by the quantile function \(F^{-1}\).

\begin{note}
There are also some other more efficient methods for sampling, often equipped
with a \emph{variance reduction} component, which can increase the ``precision'' in
the sampling (details omitted here).
\end{note}
\item \textbf{Empirical distribution functions.} Given a random sample
\(\vect{X}_1,\dotsc,\vect{X}_n\), the \defn{empirical distribution
function} \(F_n\) is defined by
\(F_n(\vect{x})=\sum_{i=1}^{n}\frac{1}{n}\indic_{[\vect{X}_i,\vect{\infty})}(\vect{x})\),
which jumps by \(1/n\) at each \(\vect{X}_i\). Here
\(\vect{X}_1,\dotsc,\vect{X}_n\) are considered to be \underline{empirical} data,
and the empirical distribution function \(F_n\) can be viewed as an
estimation of the true underlying distribution function that generate such
\(\vect{X}_1,\dotsc,\vect{X}_n\). We also note that the empirical distribution
function indeed corresponds to a mixture distribution function, with
\(p_i=1/n\) and \(F_i(\vect{x})=\indic_{[\vect{X}_i,\vect{\infty})}(\vect{x})\)
for every \(i=1,\dotsc,n\).

Sampling can also be done on an empirical distribution function.
Let \(U\sim\unif{0}{1}\) and we write \(I=\lceil nU\rceil\), which follows the
discrete uniform distribution with support \(\{1,\dotsc,n\}\). Then, based on
the mixture distribution we have \(\vect{X}_{\lceil nU\rceil}\sim
F_n\). This hints a method for sampling from the empirical
distribution function \(F_n\) as follows. Given
\(\vect{X}_1,\dotsc,\vect{X}_n\):
\begin{enumerate}[label={(\arabic*)}]
\item Obtain realizations of \(U\).
\item Based on them, obtain realizations of \(I=\lceil nU\rceil\).
\item For each realization of \(I\), return \(\vect{X}_i\) if the realization
is \(i\).
\end{enumerate}
\end{enumerate}
\subsection{Dependence}
\label{subsect:dependence}
\begin{enumerate}
\item After discussing \emph{independence} in \Cref{subsect:independence}, it
is natural to consider another related concept: \emph{dependence}.  While the
concept of independence can be mathematically defined, the idea of
``dependence'' is much more unclear: When we speak of ``dependent'' random
variables, often we are unclear about how they actually depend on each other.
In \Cref{subsect:dependence}, we will study a way to describe the dependence
clearly and mathematically, through the usage of \emph{copula}.
\item \textbf{Copula and its characterization.} A \defn{(\(d\)-dimensional)
copula} \(C\) is a distribution function on \([0,1]^{d}\) with \(\unif{0}{1}\)
margins (oftentimes, we consider the case with \(d\ge 2\) so that we can
meaningfully talk about ``dependence''). This can be viewed as a
``probabilistic'' definition of copula, and the following provides an
``analytic'' characterization of it:
\begin{proposition}
\label{prp:copula-char}
A function \(C:[0,1]^{d}\to [0,1]\) is a \(d\)-dimensional copula iff
\begin{enumerate}[label={(\arabic*)}]
\item \emph{(Groundedness)} \(C(\vect{u})=0\) if \(u_j=0\) for some \(j=1,\dotsc,d\).
\item \emph{(\(\unif{0}{1}\) margins)} \(C(1,\dotsc,1,u_j,1,\dotsc,1)=u_j\) for
all \(u_j\in [0,1]\), for all \(j=1,\dotsc,d\).
\item \emph{(\(d\)-increasingness)} \(\Delta_{(\vect{a},\vect{b}]}C\ge 0\) for
all \({}_{d}0\le\vect{a}\le\vect{b}\le{}_{d}1\).
\end{enumerate}
\end{proposition}
\begin{pf}
``\(\Rightarrow\)'': It follows by the definition of distribution function.

``\(\Leftarrow\)'': The groundedness and \(d\)-increasingness properties for
being a distribution function are immediate; we only need to establish the
normalization and right-continuous properties for \(C\) to be a distribution
function (after being extended in the way mentioned in \labelcref{it:dist-fn}):
\begin{itemize}
\item \emph{Normalization:} We have \(\lim_{\vect{x}\to\vect{\infty}}C(\vect{x})
=C(1,\dotsc,1)\overset{\text{(\(\unif{0}{1}\) margins)}}{=}1\).
\item \emph{Right-continuity:} Applying \Cref{lma:lipschitz-ineq} to the
function \(C\) with \(\unif{0}{1}\) margins here, we have
\(|C(\vect{b})-C(\vect{a})| \le\sum_{j=1}^{d}|b_j-a_j|\), which implies the
uniform continuity of \(C\), and hence its right-continuity.
\end{itemize}
The function \(C\) also has \(\unif{0}{1}\) margins by assumption, so \(C\) is
a \(d\)-dimensional copula.
\end{pf}
\item \textbf{Sklar's theorem.} From the definition of copula alone, it is not
too clear why it can serve for describing dependence. Mathematically, this
function of copula is established by the \emph{Sklar's theorem}, which is
considered to be the foundational result that drives the study of copula:
\begin{theorem}[Sklar's theorem]
\label{thm:sklar}
\hfill
\begin{enumerate}
\item \emph{(Decomposition)} Given any distribution function \(F\) with margins \(F_1,\dotsc,F_d\),
there exists a copula \(C\) such that
\begin{equation}
\label{eq:copula-link}
F(\vect{x})=C(F_1(x_1),\dotsc,F_d(x_d))\quad\text{for all \(\vect{x}\in\R^d\)}.
\end{equation}
Furthermore, \(C\) is uniquely defined on \(\prod_{j=1}^{d}\ran{F_j}\), and is
given by \(C(\vect{u})=F\left(F_1^{-1}(u_1),\dotsc,F_d^{-1}(u_d)\right)\) for all
\(\vect{u}\in\prod_{j=1}^{d}\ran{F_j}\), where \(\ran{F_j}\) denotes the
\emph{range} of \(F_j\), namely \(\{F_j(x):x\in\R\}\) for all \(j=1,\dotsc,d\).
\item \emph{(Combination)} Given any copula \(C\) and univariate distribution
functions \(F_1,\dotsc,F_d\), the function \(F\) as defined in
\Cref{eq:copula-link} is a distribution function with margins
\(F_1,\dotsc,F_d\).
\end{enumerate}
\end{theorem}
\begin{pf}
We shall only prove for the case where \(F_1,\dotsc,F_d\) are continuous.
\begin{enumerate}
\item Let \(\vect{X}=(X_1,\dotsc,X_d)\sim F\) and
\(\vect{U}=(U_1,\dotsc,U_d):=(F_1(X_1),\dotsc,F_d(X_d))\). By probability transform, we know
\(\vect{U}\) has \(\unif{0}{1}\) margins, so the distribution function of \(U\)
is indeed a copula, which we denote by \(C\).

By \Cref{lma:strict-incr-support}, we know \(F_j\) is strictly increasing on
\(\supp{F_j}\), so by \labelcref{it:gen-inv-prop} we have
\(X_j=F_j^{-1}(F_j(X_j))=F_j^{-1}(U_j)\) on \(\supp{F_j}\) for all
\(j=1,\dotsc,d\). Therefore, using a similar argument as in the proof
\Cref{prp:prob-trans}, the distribution function \(F\) of \(X\) is
\begin{align*}
F(\vect{x})&=\prob{X_j\le x_j\text{ for all \(j=1,\dotsc,d\)}} \\
&=\prob{F_j^{-1}(U_j)\le x_j\text{ for all \(j=1,\dotsc,d\)}} \\
\overset{\text{\labelcref{it:gen-inv-prop}}}&{=}
\prob{U_j\le F_j(x_j)\text{ for all \(j=1,\dotsc,d\)}} \\
&=C(F_1(x_1),\dotsc,F_d(x_d)).
\end{align*}
This establishes \Cref{eq:copula-link}. Next, note that for all
\(\vect{u}\in\prod_{j=1}^{d}\ran{F_j}\) (which equals \([0,1]^{d}\) indeed due
to the continuous, groundedness, and normalization properties of each \(F_j\)),
we have
\[
C(\vect{u})\overset{\text{\labelcref{it:gen-inv-prop}}}{=}
C(F_1(\vc{F_1^{-1}(u_1)}),\dotsc,F_d(\vc{F_d^{-1}(u_d)})
=F(\vc{F_1^{-1}(u_1)},\dotsc,\vc{F_d^{-1}(u_d)}).
\]
\item Let \(\vect{U}=(U_1,\dotsc,U_d)\sim C\) and
\(\vect{X}:=(F_{1}^{-1}(U_1),\dotsc,F_d^{-1}(U_d))\). Observe that
\[
\prob{\vect{X}\le\vect{x}}\overset{\text{\labelcref{it:gen-inv-prop}}}{=}
\prob{U_1\le F_1(x_1),\dotsc,U_d\le F_d(x_d)}
=C(F_1(x_1),\dotsc,F_d(x_d))
\]
for all \(\vect{x}\in\R^d\), so the function \(F\) as defined in
\Cref{eq:copula-link} is a distribution function (of \(\vect{X}\)) with margins
\(F_1,\dotsc,F_d\) by quantile transform.
\end{enumerate}
\end{pf}

\begin{remark}
\item \emph{(Role of copula)} The decomposition property suggests that a
joint distribution function can always be decomposed into two parts: (i) its
margins \(F_1,\dotsc,F_d\) and (ii) a copula. From this, we can see that a
copula indeed reflects the dependence inherit in the joint distribution
function \(F\).
\item \emph{(Construction of distribution functions via copula)} The
combination property provides a method to construct joint distribution
functions flexibly, according to dependence structures we want.
\item We say that \(\vect{X}\sim F\) \defn{has copula \(C\)} if
\Cref{eq:copula-link} holds, where \(F_1,\dotsc,F_d\) are the margins of \(F\).
In case the margins are all continuous, we know by \Cref{thm:sklar} that such
copula \(C\) would be uniquely determined (on \([0,1]^{d}\)).
\end{remark}
\item \textbf{Invariance principle.} Apart from Sklar's theorem, another
important result about copula is the \emph{invariance principle}, which asserts
that the copula is invariant upon strictly increasing transformations; this is
natural since such transformations do not change the ``ordering'' and hence
should not influence the dependence. This is stated more precisely below.

\begin{theorem}[Invariance principle]
\label{thm:invariance}
Consider a random vector \(\vect{X}=(X_1,\dotsc,X_d)\sim F\) with continuous
margins \(F_1,\dotsc,F_d\) and copula \(C\). If \(T_j\) is strictly increasing on
\(\supp{F_j}\) for every \(j=1,\dotsc,d\), then \((T_1(X_1),\dotsc,T_d(X_d))\)
has the copula \(C\) as well, which is also uniquely determined.
\end{theorem}
\begin{pf}
Fix any \(j=1,\dotsc,d\). On \(\supp{F_j}\), the function \(T_j\) is strictly
increasing, so it has at most countably many discontinuities, because at each
discontinuity (jump in this case) there is a rational number in between. Hence,
we have that \(T_j\) is right-continuous on the set \(\supp{F_j}\) with at most
countably many discontinuities removed. Since \(\prob{X_j\in \supp{F_j}}=1\)
and there are only at most countably discontinuities, the probability that
\(X_j\) takes values in the resulting set after such removal would still be \(1\).
So, we may assume \(T_j\) is strictly increasing and right-continuous in the
following probabilistic argument.

By the right-continuity of \(T_j\), we have
\(T_j(T_j^{-1}(x_j))\overset{\text{\labelcref{it:gen-inv-prop}}}{=}x_j\) for
all \(x_j\in\ran{T_j}\). Hence, the distribution function \(G_j\) of
\(T_j(X_j)\) is given by
\begin{align*}
G_j(x_j)&=\prob{T_j(X_j)\le \vc{x_j}}
\overset{\text{\labelcref{it:gen-inv-prop}}}{=}
\prob{T_j(X_j)\le \vc{T_j(T_j^{-1}(x_j))}} \\
\overset{\text{(\(T_j\) strictly increasing)}}&{=}
\prob{X_j\le T_j^{-1}(x_j)}
=F_j(T_j^{-1}(x_j))
\end{align*}
for all \(x_j\in\ran{T_j}\). Since \(T_j(X_j)\) cannot possibly take values
outside \(\ran{T_j}\), the equation indeed holds for all \(x_j\in\R\).

Since \(T_j\) is strictly increasing, by \labelcref{it:gen-inv-prop}
\(T_j^{-1}\) is continuous on \(\ran{T_j}\). Also, by assumption \(F_j\) is
continuous, so the composition \(G_j=F_j\circ T_j^{-1}\) is continuous on
\(\ran{T_j}\). Since \(G_j\) remains constant outside
\(\supp{G_j}=\supp{T_j(X_j)}\subseteq \ran{T_j}\), we conclude that \(G_j\) is continuous
on \(\R\).

Therefore, the joint distribution function of \((T_1(X_1),\dotsc,T_d(X_d))\) is
\begin{align*}
\prob{T_j(X_j)\le x_j\text{ for all }j=1,\dotsc,d}
\overset{\text{(\(G_j\) continuous)}}&{=}
\prob{T_j(X_j)<x_j\text{ for all }j=1,\dotsc,d} \\
\overset{\text{(\(T_j\) right-continuous, \labelcref{it:gen-inv-prop})}}&{=}
\prob{X_j<T_j^{-1}(x_j)\text{ for all }j=1,\dotsc,d} \\
\overset{\text{(\(F_j\) continuous)}}&{=}
\prob{X_j\le T_j^{-1}(x_j)\text{ for all }j=1,\dotsc,d} \\
\overset{\text{(Sklar's theorem)}}&{=}
C(F_{1}(T_{1}^{-1}(x_1)),\dotsc,F_{d}(T_{d}^{-1}(x_d))
=C(G_{1}(x_1),\dotsc,G_d(x_d))
\end{align*}
for all \(\vect{x}\in\R^d\).

Since \(T_j(X_j)\sim G_j\) for all \(j=1,\dotsc,d\), by definition we know
\((T_1(X_1),\dotsc,T_d(X_d))\) has copula \(C\). Furthermore, by Sklar's
theorem it is uniquely determined since \(G_1,\dotsc,G_d\) are all continuous.
\end{pf}

\item \textbf{Copula from probability transforms.} By applying probability
transforms and the invariance principle, we can obtain the following result,
which justifies the practice of exploring the dependence between the samples
after applying the respective distribution function on each of them (to
``scale'' the data appropriately and ensure a ``fair'' comparison).
\begin{corollary}
\label{cor:prob-trans-copula}
Let \(\vect{X}=(X_1,\dotsc,X_d)\) be a random vector with continuous margins
\(F_1,\dotsc,F_d\). Then \(\vect{X}\) has a unique copula \(C\) iff
we have \((F_1(X_1),\dotsc,F_d(X_d))\sim C\).
\end{corollary}
\begin{pf}
Let \(\vect{U}=(U_1,\dotsc,U_d):=(F_1(X_1),\dotsc,F_d(X_d))\).

``\(\Rightarrow\)'': Knowing that \(F_j\) is strictly increasing on
\(\supp{F_j}\) for every \(j=1,\dotsc,d\), the invariance principle suggests
that \(\vect{U}\) has the unique copula \(C\). Also, by the probability
transform, the margins of \(\vect{U}\) are all \(\unif{0}{1}\). Hence, by
Sklar's theorem (combination), the distribution function of \(\vect{U}\) is
\(C\).

``\(\Leftarrow\)'': By \labelcref{it:gen-inv-prop}, we have
\(\vect{X}=\left(F_1^{-1}(F_1(X_1)),\dotsc,F_d^{-1}(F_d(X_d))\right)
=(F_{1}^{-1}(U_1),\dotsc,F_{d}^{-1}(U_d))\).
\end{pf}
\item \textbf{Three common copulas.} Three particular types of dependence
structures are often of interest, namely \emph{independence}, \emph{extremely
positive dependence (comonotonicity)}, and \emph{extremely negative dependence
(countermonotonicity)}. They are described by the following three copulas
respectively:
\begin{enumerate}
\item \defn{independence copula}:
\(C(\vect{u})=\Pi(\vect{u}):=\prod_{j=1}^{d}u_j\) for all
\(\vect{u}\in[0,1]^{d}\).
\item \defn{comonotone copula}:
\(C(\vect{u})=M(\vect{u}):=\min\{u_1,\dotsc,u_d\}\) for all
\(\vect{u}\in[0,1]^{d}\).
\item \defn{countermonotone copula} (when \(d=2\)):
\(C(\vect{u})=W(\vect{u}):=\max\{(\sum_{j=1}^{d}u_j-d+1,0\}\) for all
\(\vect{u}\in[0,1]^{d}\).
\begin{note}
Using a similar argument as in \labelcref{it:comp-incr-not-imply-d-incr}, one
can show that \(W\) is never a valid copula for every \(d\ge 3\).
\end{note}
\end{enumerate}
To prove that they are indeed valid copulas (only \(d=2\) for the
countermonotone copula), consider the following:

\begin{pf}
\begin{enumerate}
\item We have \(\vect{U}\sim \Pi\) for \(\vect{U}=(U_1,\dotsc,U_d)\) with
\(U_1,\dotsc,U_d\iid \unif{0}{1}\), since:
\[
\prob{\vect{U}\le\vect{u}}=\prod_{j=1}^{d}\prob{U_j\le u_j}
=\prod_{j=1}^{d}u_j=\Pi(\vect{u})
\]
for all \(\vect{u}\in[0,1]^{d}\).
\item We have \(\vect{U}\sim M\) for \(\vect{U}=(U,,\dotsc,U)\) with
\(U\sim\unif{0}{1}\), since:
\[
\prob{\vect{U}\le\vect{u}}=\prob{U\le u_1,\dotsc,U\le u_d}
=\prob{U\le\min\{u_1,\dotsc,u_d\}}
=\min\{u_1,\dotsc,u_d\}=M(\vect{u})
\]
for all \(\vect{u}\in[0,1]^{d}\).
\item When \(d=2\), we have \(\vect{U}\sim W\) for \(\vect{U}=(U,1-U)\) with
\(U\sim\unif{0}{1}\), since:
\[
\prob{\vect{U}\le\vect{u}}=\prob{U\le u_1,1-U\le u_2}
=\prob{1-u_2\le U\le u_1}
\overset{\text{(by cases)}}{=}
\max\{u_1-(1-u_2),0\}
=W(\vect{u})
\]
for all \(\vect{u}\in[0,1]^{2}\).
\end{enumerate}
\end{pf}

\item\label{it:indp-comono-countermono-cop-interpret} \textbf{Interpretations
of independence, comonotone, and countermonotone copulas.} For
\((X_1,\dotsc,X_d)\sim F\) with continuous margins \(F_1,\dotsc,F_d\) and
copula \(C\), we can interpret the previously introduced independence,
comonotone, and countermonotone copulas as follows:
\begin{enumerate}
\item \emph{(Independence copula)} \(C=\Pi\) iff \(X_1,\dotsc,X_d\) are
\underline{independent}.
\item \emph{(Comonotone copula)} \(C=M\) iff \(X_j\eqas T_j(X_1)\), where
\(T_j=F_j^{-1}\circ F_1\) is strictly increasing on \(\supp{F_1}\), for all
\(j=2,\dotsc,d\).
\item \emph{(Countermonotone copula)} \(C=W\) iff \(X_2\eqas T(X_1)\), where
\(T=F_2^{-1}\circ (1-F_1)\) is strictly decreasing on \(\supp{F_1}\).
\end{enumerate}
\begin{pf}
(Sketch)
\begin{enumerate}
\item Apply \Cref{thm:char-rv-ind}.
\item Apply the invariance principle for ``\(\Leftarrow\)''.
\item Apply the invariance principle for ``\(\Leftarrow\)'' also.
\end{enumerate}
\end{pf}
\item \textbf{Fr\'echet-Hoeffding bounds.} For the comonotone copula \(M\) and
countermonotone copula \(W\), apart from describing the extremely positive
dependence and extremely negative dependence, they also serve for bounds on
\emph{any} copula:
\begin{theorem}[Fr\'echet-Hoeffding bounds]
\label{thm:frechet-hoeffding-bounds}
For every \(d\)-dimensional copula \(C\), we have \(W(\vect{u})\le
C(\vect{u})\le M(\vect{u})\) for all \(\vect{u}\in[0,1]^{d}\).
\end{theorem}
\begin{pf}
Fix any \(\vect{u}\in[0,1]^{d}\) and any \(d\)-dimensional copula \(C\).

We first show the inequality \(W\le C\). Note that
\[
1-C(\vect{u})=C({}_{d}1)-C(\vect{u})\overset{\text{(\Cref{lma:lipschitz-ineq})}}{\le}
\sum_{j=1}^{d}(1-u_j)=d-\sum_{j=1}^{d}u_j,
\]
and thus \(C(\vect{u})\ge \vc{(\sum_{j=1}^{d}u_j)-d+1}\). Since
\(C(\vect{u})\ge 0\) (as a distribution function), this implies
\(C(\vect{u})\ge \max\{\vc{(\sum_{j=1}^{d}u_j)-d+1},0\}=W(\vect{u})\).

Next, we show the inequality \(C\le M\). Since \(C\) is \(d\)-increasing, it is
also componentwise increasing by \labelcref{it:f-vol-nonneg}. Thus, applying
this increasingness for every component except the \(j\)th one gives
\[
C(\vect{u})\le C(1,u_2,\dotsc,u_d)\le\dotsb\le C(1,\dotsc,1,u_j,1,\dotsc,1)
\overset{\text{(\(\unif{0}{1}\) margins)}}{=}u_j
\]
for every \(j=1,\dotsc,d\). This then implies that
\(C(\vect{u})\le\min\{u_1,\dotsc,u_d\}=M(\vect{u})\).
\end{pf}

\begin{remark}
\item In view of this result, \(M\) and \(W\) are called the \defn{Fr\'echet-Hoeffding upper bound}
and \defn{Fr\'echet-Hoeffding lower bound} respectively.
\item Even if \(W\) is not a valid copula for every \(d\ge 3\), it can still
serve as a lower bound, as this result suggests.
\end{remark}
\end{enumerate}
